\chapter{Spinors, Their Handedness}
\label{ch-spinors-hand}

This chapter is based on an AI output
and Ref.\cite{birdtracks-book}.


\section{In 1+3 dim}

Consider $SO(1,3)$ so $n=1+3=4$

$\gamma^\mu_{ab}$


$a,b\in\{1,2,3,4\}$

$\mu=1, 2, \ldots, n$

{\bf Mostly-plus metric (M+M) (a.k.a. East Coast metric)} is $g_{\mu\nu}= (-1, 1,1,1)$

{\bf Mostly-minus metric (M-M) (a.k.a. West Coast metric)} is $g_{\mu\nu}= (1, -1,-1,-1)$

\beq
[\gamma_\mu, \gamma_\nu]_+ = 2g^{\mu\nu}
\quad \text{ for } \mu, \nu=0, 1,2,3
\label{eq-ciff}
\eeq

\beq
\begin{array}{ll}
\text{M--M}: &
\gamma_0^\dagger = \gamma_0,
\quad \gamma_i^\dagger = -\gamma_i
\quad\text{ for } i=1,2,3
\\
\text{M+M}: &
\gamma_0^\dagger = -\gamma_0,
\quad \gamma_i^\dagger = \gamma_i
\quad\text{ for } i=1,2,3
\end{array}
\eeq
The last equation is true iff

\beq
\begin{array}{ll}
\text{M--M:} &
\gamma_\mu^\dagger =\gamma_0 \gamma_\mu \gamma_0
\\
\text{M+M:} &
\gamma_\mu^\dagger =-\gamma_0 \gamma_\mu \gamma_0
\end{array}
\eeq



In 1+3 dimensional relativistic quantum field theory, the {\bf chirality matrix} 
$\gamma_5$ is defined in terms of the four Dirac gamma matrices $\gamma_\mu$, where $\mu=0,1,2,3$. 
as follows

\beq
\begin{array}{ll}
\text{M--M:} & \gamma_5 = i \gamma_0 \gamma_1 \gamma_2 \gamma_3 = \frac{i}{4!} \epsilon^{\mu\nu\rho\sigma} \gamma_\mu \gamma_\nu \gamma_\rho \gamma_\sigma
\\
\text{M+M:} &\gamma_5 = -i \gamma_0 \gamma_1 \gamma_2 \gamma_3 
\end{array}
\eeq 


In the \qt{Dirac representation}, 

\beq
\gamma_5 =\pm
\left[
\begin{array}{cc}
0 & I_2 \\
I_2 & 0
\end{array}
\right]
\eeq

The  properties of $\gamma_5$ might change by a sign
depending on which metric
we are using. Henceforth, we 
will use the M--M 

$\gamma_5$ properties:

\begin{itemize}
 
\item Anticommutes with $\gamma_\mu$
\beq
\gamma_5\gamma_\mu=-\gamma_\mu\gamma_5
\quad
\text{for $\mu=0,1,2,3$}
\eeq
This follows because $g_{\mu\nu}$
is diagonal so, by the Clifford algebra
definition Eq.(\ref{eq-ciff})
(anti-cummutator of gammas), the
$\gamma_\mu$ anticommutes with 3 of the
4 gamma matrices in $\gamma_5$

\item Hermitian
\beq
\gamma_5^\dagger = \gamma_5
\eeq

This follows because

\beq
\gamma_5^\dagger = -i(-\gamma_3)(-\gamma_2)(-\gamma_1)(+\gamma_0)=  i\gamma_3\gamma_2\gamma_1\gamma_0=
(-1)^6
i(\gamma_0\gamma_1\gamma_2\gamma_3)=\gamma_5
\eeq

\item Square is 1
\beq
\gamma_5^2 = I.
\eeq

Therefore its eigenvalues are $+1$ (positive chirality)  and $-1$ (negative chirality). This follows because

\beq
(\gamma_5)^2=-\gamma_0\gamma_1\gamma_2\gamma_3\gamma_0\gamma_1\gamma_2\gamma_3
=-(-1)^6(\gamma_0)^2(\gamma_1)^2(\gamma_2)^2(\gamma_3)^2=
-g_{00}g_{11}g_{22}g_{33}=
1
\eeq


\item Traceless
\beq
\tr(\gamma_5)=0
\eeq
This follows immediately from
the expression for
$\gamma_5$ in the Dirac representation.
Alternatively, note that if spinor $\psi$
satisfies \beq \gamma_5\psi =+\psi\eeq (has positive chirality),
then spinor $\gamma_\mu\psi$ has negative 
chirality because
\beq \gamma_5(\gamma_\mu \psi)=-
\gamma_\mu \gamma_5\psi
=-\gamma_\mu \psi
\eeq
This means that the eigenvalues of 
$\gamma_5$ come in $\pm 1$ pairs.
Hence the trace of $\gamma_5$ must be zero.


\item
\beq
\tr(\gamma_5 \gamma_\mu \gamma_\nu \gamma_\rho \gamma_\sigma)
= 4i \epsilon^{\mu\nu\rho\sigma},
\eeq
with $\epsilon^{0123}=+1$

\item
If 
\beq
P_L = P_-=\frac{1}{2}(1 - \gamma_5), \qquad
P_R = P_+ =\frac{1}{2}(1 + \gamma_5).
\eeq
where $R$= right handed,
$L=$ left handed
then

\beq
\psi = \underbrace{\psi_L}_{P_L\psi} + \underbrace{\psi_R}_{P_R\psi}
\eeq

\item
If $\PP=\gamma_0$ (Parity Operator) and $\sigma^{\mu\nu} = \frac{i}{2}[\gamma_\mu,\gamma_\nu]$ (Lorentz generators)

\beq
[\gamma_5, \sigma^{\mu\nu}] = 0,\quad 
\gamma_5 \PP=- \PP\gamma_5\eeq
Thus $\gamma_5$ is {\bf pseudoscalar} under Lorentz transformations (i.e., invariant under proper Lorentz transformations but flips sign under parity).
\end{itemize}





\section{In $p+q$ dim}

Consider $SO(p,q)$ so $n=p+q$

$\gamma^\mu_{ab}$
$int(x)=$ integer part of $x\in\RR$

$a,b\in\{ 1, 2, \ldots, 2^{\intntwo}\}$

$\mu=1, 2, \ldots, n$

% Please add the following required packages to your document preamble:
% \usepackage[table,xcdraw]{xcolor}
% Beamer presentation requires \usepackage{colortbl} instead of \usepackage[table,xcdraw]{xcolor}
\begin{table}[h!]
\begin{tabular}{|
>{\columncolor[HTML]{FFFFC7}}l |l|l|l|l|l|l|l|l|l|l|}
\hline
$n$ & 1 & 2 & 3 & 4 & 5 & 6 & 7 & 8 & 9 & 10 \\ \hline
$\intntwo$ & 0 & 1 & 1 & 2 & 2 & 3 & 3& 4 & 4 & 5 \\ \hline
$d=2^{\intntwo}$ & 1 & 2 & 2 & 4 & 4 & 8 & 8 & 16 & 16 & 32 \\ \hline
\end{tabular}
\caption{$\gamma_\mu \in \CC^{d\times d}$}
\label{tab-gamma-dim}
\end{table}


\beq
[\gamma_\mu, \gamma_\nu]_+ = 2g^{\mu\nu}
\quad \text{ for } \mu, \nu=0, 1,2,\ldots, n
\label{eq-ciff-gen}
\eeq

\beq
\Gamma^{(3)}_{\lam\mu\nu}
=
\bcen
\xymatrix@R=1pc@C=1pc@R=1.5pc{
&&&
\\
&\ar[u]\cala_n \ar@2{-}[rr]
&\ar[u]
&\ar[u]
&
\\
&\ar@{-->}[l]\ul{\gamma}\ar[u]
&\ar@{-->}[l]\ul{\gamma}\ar[u]
&\ar@{-->}[l]\ul{\gamma}\ar[u]
&\ar@{-->}[l]
}
\ecen
=
\bcen
\xymatrix@R=1pc@C=1pc@R=1.5pc{
&&&
\\
&\ar[u]\cala_n^{\frac{1}{2}} \ar@2{-}[rr]
&\ar[u]
&\ar[u]
&
\\
&\cala_n^{\frac{1}{2}} \ar@2{-}[rr]
&
&
&
\\
&\ar@{-->}[l]\ul{\gamma}\ar[u]
&\ar@{-->}[l]\ul{\gamma}\ar[u]
&\ar@{-->}[l]\ul{\gamma}\ar[u]
&\ar@{-->}[l]
}
\ecen
\eeq

Generalize $\gamma_5$ from $SO(1,3)$ to
$\gfive$ for $SO(p,q)$.
\beqa
\gfive &=&
\frac{1}{\sqrt{n!}}
\bcen
\xymatrix@R=1pc@C=1pc@R=1.5pc{
&\cala_n^{\frac{1}{2}} \ar@2{-}[rr]
&
&
&
\\
&\ar@{-->}[l]\ul{\gamma}\ar[u]
&\ar@{-->}[l]\ul{\gamma}\ar[u]
&\ar@{-->}[l]\ul{\gamma}\ar[u]
&\ar@{-->}[l]
}
\ecen
\\
&=&
\frac{e^{i\phi}}{n!}
\eps^{\mu_1 \mu_2 \ldots \mu_n}
\gamma_{\mu_1}\gamma_{\mu_2}
\ldots\gamma_{\mu_n}
= e^{i\phi}
\gamma_1 \gamma_2\ldots \gamma_n
\eeqa


$\gfive$ properties
\begin{itemize}

\item Anticommutes  with $\gamma_\mu$ if $n$ is even. Commutes 
with $\gamma_\mu$ if $n$ is odd.


\beq 
\left\{
\begin{array}
{ll}\gamma_\mu \gfive =- \gfive \gamma_\mu
&\text{($n$ even)}
\\
\gamma_\mu \gfive = \gfive \gamma_\mu
&\text{($n$ odd)}
\end{array}
\right.
\eeq

This follows because $g_{\mu\nu}$
is diagonal so, by the Clifford algebra
definition Eq.(\ref{eq-ciff-gen})
(anti-cummutator of gammas), the
$\gamma_\mu$ anticommutes with $n-1$ of the
$n$ gamma matrices in $\gfive$

\item For $n$ odd, $\gfive$ is
a constant matrix. If also $\gfive^2=1$,
then $\gfive=\pm1$.

\item Square is one

If
\beq
e^{i\phi} = i^\frac{n(n-1)}{2}
\sqrt{\prod_{j=1}^k
g_{jj}}
\eeq
with $g_{jj}\in \{1, -1\}$, then


\beq
\gfive^2 = 1
\eeq
This follows because

\beqa
\gfive^2 &=&
e^{i2\phi} \gamma_1\gamma_2\cdots\gamma_n
\gamma_1\gamma_2\cdots\gamma_n
\\
&=&
e^{i2\phi}
(-1)^{\frac{n(n-1)}{2}}
(\gamma_{1})^2
(\gamma_{2})^2
\ldots
(\gamma_{n})^2
\\
&=&
e^{i2\phi}
(-1)^{\frac{n(n-1)}{2}}
\prod_{j=1}^k
g_{jj}
\eeqa


 
\item If $n$ is even and
\beq
P_\pm = \frac{1}{2}(1 \pm \gfive),
\eeq
then 

\beq
P_+^2=P_-^2=1,\quad  P_+P_-=0,\quad P_+ + P_-=1
\eeq

\beq 
\left\{
\begin{array}{ll}
\gamma_\mu P_+ = P_- \gamma_\mu
&\text{($n$ even)}
\\
\gamma_\mu P_+ = P_+ \gamma_\mu
&\text{($n$ odd)}
\end{array}
\right.
\eeq

\beq
\gamma_\mu = P_+ \gamma_\mu P_-
+
P_-\gamma_\mu P_+
\quad\text{($n$ even)}
\eeq
Note that when $n$ is odd,
$\gfive=\pm1$ so either $P_+$ or $P_-$ is zero
and the other is 1. Hence this decomposition
is only  valid for $n$ even.

The Lorentz group for $n$ odd has only one irreducible spinor representation.
In even $n$ it has two inequivalent spinor irreps (positive and negative chirality).

Thus chirality cannot exist in odd dimensions because the representation theory of the Lorentz group does not support it.

\end{itemize}


\section{Weyl and Majorana Spinors}

% Please add the following required packages to your document preamble:
% \usepackage[table,xcdraw]{xcolor}
% Beamer presentation requires \usepackage{colortbl} instead of \usepackage[table,xcdraw]{xcolor}
\begin{table}[h!]
\begin{tabular}{|l|l|l|l|}
\hline
\rowcolor[HTML]{FFFFC7} 
Spinor Type & Condition & \# of dofs & Exists in 4D? \\ \hline
Dirac & none & 8 & yes \\ \hline
Weyl & $\gfive \psi = \pm \psi$ & 4 & yes \\ \hline
Majorana & $\psi^c=\psi$ & 4 & yes \\ \hline
Majorana-Weyl & both & 2 & no \\ \hline
\end{tabular}
\caption{Different types of spinors, their number of dofs (real degrees of freedom) and whether they exist in 4 dim.}
\label{tab-maj-weyl}
\end{table}

\begin{itemize}
\item Weyl spinors

\beq
P_L = \frac{1}{2}(1-\gfive),\quad
P_R = \frac{1}{2}(1+\gfive)
\eeq

A {\bf left-handed Weyl spinor} satisfies
\beq
\psi_L = P_L\psi,\quad \gfive\psi_L=-\psi_L
\eeq

A {\bf a right-handed Weyl spinor} satisfies
\beq
\psi_R = P_R\psi,\quad \gfive \psi_R=+\psi_R.
\eeq

\item Majorana Spinors


Given a Dirac spinor $\psi$, its {\bf charge-conjugate} is

\beq
\psi^c = C \bar{\psi}^T
\quad \text{ where }
\bar{\psi} = \psi^\dagger\gamma^0
\eeq

The matrix $C$ is defined by

\beq
C\gamma^\mu C^{-1} = -(\gamma^\mu)^T,
\eeq
and exists in any dimension.

A {\bf Majorana spinor} is one satisfying 

\beq
\psi=\psi^c
\eeq

Whether Majorana spinors exist depends on the spacetime dimension and signature.
In 1+3 dimensions, Majorana spinors do exist because one can choose a representation where all gamma matrices are purely real or imaginary so that $\psi$ can be chosen real.

A Majorana spinor has 4 real components. (vs.  Dirac spinors which have  4 complex.

A Weyl spinor and Majorana spinor cannot both be imposed simultaneously in 1+3 dim.
But in some dimensions (e.g. 2 mod 8), Majorana–Weyl spinors **do** exist (important in 10D superstring theory).



\end{itemize}

