\chapter{Symplectic Groups}
\label{ch-symp-groups}


This chapter is based on Cvitanovic's Birdtracks book Ref. \cite{birdtracks-book}.

$n$ even
\beq
f = 
\left(
\begin{array}{cc}
0& I_{n/2}
\\
-I_{n/2}&0
\end{array}
\right)
\eeq

\beq
f^T=-f,\quad
f^Tf=I_{n},
\quad f^2 = 
-I_{n}
\eeq
$f^2$ is a primitive invariant matrix so it
must be proportional to
the identity. 

\beq
Sp(n)=\{G\in GL(n,\CC):
G^Tf G= f
\}
\eeq

\begin{claim}
If $G\in Sp(n)$, then $det(G)=1$.
\end{claim}
\proof

\beq
det(G^T f G)= det(f)
\eeq

\beq
det(f) = det(I_{n/2})det(-I_{n/2})= (-1)^{n/2}
\eeq

\beq
det(G^T f G) = det^2(G)
(-1)^{n/2}
\eeq


\beq
det^2(G)=1
\eeq

$det(G)=\pm 1$. $Sp(n)$
is connected, $I_n\in Sp(n)$  and $det(I_n)=1$.
Hence $det(G)=1$.

\qed

We assume that $n$ is even and the {\bf pseudo metric tensor} $f_{ab}$
is a primitive invariant
that satisfies:

\beq
f_{ab}=-f_{ba}
\eeq

\beq
f_{ac} f_{cb}=-\delta_a^b
\eeq
where $a, b\in \{1,2, \ldots, n\}$.

\hrule
{\bf  Net Arrow Convention:}
We will point the arrows in a birdtrack so that the birdtrack is a DAG. Cycles that make the birdtrack not acyclic will
have a segment in red. Without that
red segment, the birdtrack becomes  acyclic. The reason we follow this arrow convention is that it
promotes acyclic birdtracks which are more
akin to bnets. We will eschew undirected birdtracks for the same reason: bnets are directed.

{\bf Tensor Index Convention}: 
Since we will only use DAGs,
birdtracks will only use $f\indices{_a^b}$ and $f\indices{^a_b}$
We will assume that

\beq
[f\indices{^a_b}]=
[f\indices{_a^b}]=f
\quad
\text{(fully explicit $f$ (FEf) convention)}
\eeq
An alternative index convention
that we will not follow is to assume

\beq
f\indices{^a_b}= f\indices{_a^b}=-\delta_a^b, \quad
[f_{ab}]=[f^{ab}]=f
\quad
\text{(non-FEf convention)}
\eeq
To go from our FEf to a non-FEf convention, 
write  the birdtrack equation (BE) algebraically, and then,
in the BE, 
move indices up
or down so that $f$ tensors all have
either both indices up or both indices down and so that contracted indices are in up-down pairs.
For example, if the BE in the
FEf convention is:

\beq
M\indices{_a^b} f\indices{_b^c} = f\indices{_a^c}
\eeq
then replace that  BE by

\beq
M\indices{_a^b} f\indices{_b_c} = f\indices{_a_c}
\eeq
 



\hrule


\beq
f_{ab}=
\bcen
\xymatrix{
a
&\ar@[green][l]
f\ar@2{-}[d]
\\
b&\ar[l]
}
\ecen
=
\xymatrix{
a
&f\ar[r]\ar@[green][l]
&
b
}
\eeq

\beq
f_{ba} =f_{ab}^T=
\bcen
\xymatrix{
a
&\ar@[green][l]
f^T\ar@2{-}[d]
\\
b
&\ar[l]
}
\ecen
=
\bcen
\xymatrix{
&\ar@{<->}[d]\ar[l]&\ar[l]
f\ar@2{-}[d]
\\
&\ar[l]&\ar[l]
}
\ecen
\eeq

\beq
\myboxed{
f_{ab}=-f_{ba}
},\quad
\bcen
\xymatrix{
&\ar[l]
f\ar@2{-}[d]
\\
&\ar[l]
}
\ecen
=
-
\bcen
\xymatrix{
&\ar@{<->}[d]\ar[l]&\ar[l]
f\ar@2{-}[d]
\\
&\ar[l]&\ar[l]
}
\ecen
=
-
\bcen
\xymatrix{
&\ar[l]
f^T\ar@2{-}[d]
\\
&\ar[l]
}
\ecen
\eeq


$Sp(n)$ leaves invariant
the skew symmetric form

\beq
h(p,q)=f_{ab}p^a q^b
\eeq

\beq
h(Gp, Gq)=h(p,q)
\implies
G\indices{^{b'}_b}
G\indices{^{a'}_a}
f_{a'b'}
=f_{ab}
\implies
G^TfG = f
\eeq

\beq
\myboxed{f_{ac}^Tf^{cb}=\delta_a^b},\quad
\xymatrix{
&f^T\ar[l]\ar[r]
&f
&\ar[l]
}=\xymatrix{
&\ar[l]
}
\eeq

\beq
\myboxed{f_{ac}f^{cb}=-\delta_a^b},\quad
\xymatrix{
&f\ar[l]\ar[r]
&f
&\ar[l]
}=\xymatrix{
-
&\ar[l]
}
\eeq

Generator $(T_i)_{ab}$ :

\beq
(T_i)\indices{_a^b}=
\bcen
\xymatrix
{
&\ar@{~}[d]&
\\
&T_i\ar[l]
&
\ar[l]
}
\ecen
\eeq



\beq
\begin{array}{l}
\myboxed{(T_i)\indices{_a^c}
f_{cb}+
\underbrace{(T_i)\indices{_b^c}
f_{ac}}_{f_{ac}(T_i^T)
\indices{^c_b}}
=0}
\\
\bcen
\xymatrix{
&\ar@{~}[d]
&
\\
&\ar[l]T_i
&\ar[l]
f\ar@2{-}[d]
\\
&&\ar[ll]
}
\ecen
+
\bcen
\xymatrix{
&\ar@{~}@/_1pc/[dd]
&
\\
&&\ar[ll]
f\ar@2{-}[d]
\\
&\ar[l]  T_i
&\ar[l]
}
\ecen
=0
\end{array}
\eeq



\section{$V_{def}\otimes V_{def}$ Decomposition}


\beq
M\indices{_a^b_c^d}
=
\bcen
\xymatrix@R=1pc@C=1pc{
a
&
&
&\ar@/_1pc/[dl]
d
\\
&f\ar@[green]@/_1pc/[ul]
&f^T\ar@/_1pc/[dr]
&
\\
\ar@/_1pc/[ur]
b
&
&
&c
}
\ecen
\eeq




\beqa
\cala_2 M
&=&
\bcen
\xymatrix@R=1pc@C=1pc{
&\cala_2\ar@2{-}[dd]\ar[l]
&
&
&\ar@/_1pc/[dl]
\\
&
&f\ar@/_1pc/[ul]
&f^T\ar@/_1pc/[dr]
&
\\
\ar[r]
&\ar@/_1pc/[ur]
&
&
&
}
\ecen
\\
&=&
\frac{1}{2}
\left[
\bcen
\xymatrix@R=1pc@C=1pc{
&\ar[l]
&
&
&\ar@/_1pc/[dl]
\\
&
&f\ar@/_1pc/[ul]
&f^T\ar@/_1pc/[dr]
&
\\
\ar[r]
&\ar@/_1pc/[ur]
&
&
&
}
\ecen
-
\bcen
\xymatrix@R=1pc@C=1pc{
&\ar@{<->}[dd]\ar[l]
&
&
&\ar@/_1pc/[dl]
\\
&
&f\ar@/_1pc/[ul]
&f^T\ar@/_1pc/[dr]
&
\\
\ar[r]
&\ar@/_1pc/[ur]
&
&
&
}
\ecen
\right]
\\
&=& M
\eeqa
Hence $M$ is ant-symmetric so 
only anti-symmetric
 space decomposes.

\beq
\tr(M)=
\bcen
\xymatrix@R=1pc@C=1pc{
a
&
&
&\ar@/_1pc/[dl]
d
\ar@[red]@{-}@/_1pc/[lll]
\\
&f\ar@/_1pc/[ul]
&f^T\ar@/_1pc/[dr]
&
\\
\ar@/_1pc/[ur]
b
&
&
&c
\ar@[red]@{-}@/^1pc/[lll]
}
\ecen
=\tr(ff^T)= n
\eeq

\beqa
\tr(\cala_2)
&=&
\frac{1}{2}
\left[
\bcen
\xymatrix@R=1.5pc@C=1.5pc{
&
&\ar[ll]
\ar@[red]@{-}@/_.8pc/[ll]
\\
&
&\ar[ll]
\ar@[red]@{-}@/^.8pc/[ll]
}
\ecen
-
\bcen
\xymatrix@R=1pc@C=1.5pc{
&\ar@{<->}[d]\ar[l]
&\ar[l]
\ar@[red]@{-}@/_.8pc/[ll]
\\
&\ar[l]
&\ar[l]
\ar@[red]@{-}@/^.8pc/[ll]
}
\ecen
\right]
\\
\nonumber
\\
&=&
\frac{1}{2}(n^2-n)=\frac{1}{2}n(n-1)
\eeqa

\beqa
\tr(\cals_2)
&=&
\frac{1}{2}
\left[
\bcen
\xymatrix@R=1pc@C=1.5pc{
&
&\ar[ll]
\ar@[red]@{-}@/_.8pc/[ll]
\\
&
&\ar[ll]
\ar@[red]@{-}@/^.8pc/[ll]
}
\ecen
+
\bcen
\xymatrix@R=1.5pc@C=1.5pc{
&\ar@{<->}[d]\ar[l]
&\ar[l]
\ar@[red]@{-}@/_.8pc/[ll]
\\
&\ar[l]
&\ar[l]
\ar@[red]@{-}@/^.8pc/[ll]
}
\ecen
\right]
\\
\nonumber
\\
&=&
\frac{1}{2}(n^2+n)=\frac{1}{2}n(n+1)
\eeqa

\beq
M^2=nM
\eeq
Hence, $(M-n)M=0$
so $M$ has two eigenvalues $\lam=0,n$.

Next we will use the following equation from Chapter \ref{ch-reducibility}
\footnote{Note that this equation
projects to zero all eigenvalues except one.}
to obtain a projection (PO)
operator for each eigenvalue

\beq 
P_i = \sum_{j\neq i}
\frac{M-\lam_j}{\lam_i-\lam_j}
\eeq

\begin{enumerate}
\item Singlet (Anti-symmetric $\cala_2 P_S=0$) PO
\beq
P_S = \frac{1}{n}
f\indices{_a^b}
(f^T)\indices{_c^d}
=
\frac{1}{n}
\bcen
\xymatrix@R=1pc@C=1pc{
a
&
&
&\ar@/_1pc/[dl]
d
\\
&f\ar@/_1pc/[ul]
&f^T\ar@/_1pc/[dr]
&
\\
\ar@/_1pc/[ur]
b
&
&
&c
}
\ecen
\eeq

\beq
\tr(P_S)= \frac{1}{n}\tr(M)=1
\eeq


\item Traceless Anti-symmetric PO\footnote{Traceless here refers to 
$P\indices{_a^a_c^d}V\indices{_d^c}=(PV)\indices{_a^a}=0$ for any vector $V\indices{_d^c}$. It does not refer 
to
$P\indices{_a^b_b^a}=0$}

\beqa
P_{TA}&=&
\frac{1}{2}
(\delta\indices{_a^d}\delta\indices{_b^c}
- \delta\indices{_b^d}\delta\indices{_a^c})
- \frac{1}{n}
f\indices{_a^b}
(f^T)\indices{_c^d}
\\
&=&
\bcen
\xymatrix{
a
&\cala_2\ar@2{-}[d]\ar[l]
&\ar[l] d
\\
b
&\ar[l]
&\ar[l]
c
}
\ecen
-\frac{1}{n}
\bcen
\xymatrix@R=1pc@C=1pc{
a
&
&
&\ar@/_1pc/[dl]
d
\\
&f\ar@/_1pc/[ul]
&f^T\ar@/_1pc/[dr]
&
\\
\ar@/_1pc/[ur]
b
&
&
&c
}
\ecen
\eeqa

\beq
\tr(P_{TA})= \frac{1}{2}(n^2-n-2)=
\frac{1}{2}(n+1)(n-2)
\eeq

\item Symmetric PO

\beqa
P_{SYM}&=&
\frac{1}{2}
(\delta\indices{_a^d}\delta\indices{_b^c}
+ \delta\indices{_b^d}\delta\indices{_a^c})
\\
&=&
\bcen
\xymatrix{
a
&\cals_2\ar@2{-}[d]\ar[l]
&\ar[l] d
\\
b
&\ar[l]
&\ar[l]
c
}
\ecen
\eeqa

\beq
\tr(P_{SYM}) = 
\frac{1}{2}(n^2+n)=
\frac{1}{2}n(n+1)
\eeq
\end{enumerate}

Clearly, 

\beq
P_{SYM}^2 =P_{SYM}, \quad P_S^2=P_S
\eeq
Note that 

\beq
P_{TA}P_S =(\cala_2 -\frac{1}{n}M)(\frac{1}{n}M)=P_S-P_S=0
\eeq
Hence
\beq
P_{TA}^2
=(\cala_2 -P_{S})^2
= P_{TA}
\eeq

$P_{SYM}$ is the only of the 3 PO that
is an invariant tensor so

\beq
\bcen
\xymatrix@R=1pc@C=1pc{
&
&
&\ar@/_1pc/[dl]
\\
&T_i\ar@/_1pc/[ul]
&T_i\ar@/_1pc/[dr]\ar@{~}[l]
&
\\
\ar@/_1pc/[ur]
&
&
&
}
\ecen
=
\bcen
\xymatrix{
&\cals_2\ar@2{-}[d]\ar[l]
&\ar[l] 
\\
\ar[r]
&\ar[r]
&
}
\ecen
\eeq


The Clebsch-Gordan series for $V\otimes V$ (i.e., decomposition of 
$V\otimes V$) is

\beq
\begin{array}{ccccccc}
\overbrace{V\otimes V}
^\calv
&=
&P_S\calv
&\oplus
&P_{SYM}\calv
&\oplus
&P_{TA}\calv
\\
\ydiagram{1}
\otimes\ydiagram{1}
&=
&\bullet
&\oplus
&\ydiagram{2}
&\oplus
&\ydiagram{1,1}
\\
n^2
&=
& 1
&+
&\frac{1}{2}n(n+1)
&+
&\frac{1}{2}(n+1)(n-2)
\end{array}
\eeq

The projection operator tree is
\begin{center}
\begin{minipage}{2cm}
\dirtree{%
.1 $P_{ANTI}$.
.2 $P_S$.
.2 $P_{TA}$.
.1 $P_{SYM}$.
}
\end{minipage}
\end{center}
