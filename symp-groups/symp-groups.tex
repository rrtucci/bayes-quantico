\chapter{Symplectic Groups}
\label{ch-symp-groups}


This chapter is based on Cvitanovic's Birdtracks book Ref. \cite{birdtracks-book}.

Throughout this chapter,
we will assume that $n$ is even. The {\bf symplectic group} $Sp(n)$ is defined as

\beq
Sp(n)=\{G\in GL(n,\CC):
G^Tf G= f
\}
\eeq
where $f$ is the anti-symmetric matrix

\beq
f = 
\left(
\begin{array}{cc}
0& I_{n/2}
\\
-I_{n/2}&0
\end{array}
\right)
\eeq
Note that

\beq
f^T=-f,\quad
f^Tf=I_{n},
\quad f^2 = 
-I_{n}
\eeq
$f^2$ is a primitive invariant matrix
because $f$ is, so $f^2$
must be proportional to
the identity. 



\begin{claim}
If $G\in Sp(n)$, then $det(G)=1$.
\end{claim}
\proof

Note that
\beq
\underbrace{det(G^T f G)}
_{det^2(G)det(f)}= det(f)
\eeq

\beq
det(f) = det(I_{n/2})det(-I_{n/2})= (-1)^{n/2}
\eeq
Hence,

\beq
det^2(G)=1\implies det(G)=\pm 1
\eeq
$Sp(n)$
is connected, $I_n\in Sp(n)$  and $det(I_n)=1$.
Hence $det(G)=1$.

\qed

Define the {\bf pseudo metric tensor} $f_{ab}$ to be a antisymetric matrix
that satisfies:


\beq
f_{ab}=-f_{ba} = [f]_{ab},\quad
f^{ab}=-f^{ba}= [f]_{ab},\quad
f\indices{_a^b}=f\indices{_b^a} =\delta_a^b
\eeq

\beq
f_{ba}x^a=x_b,\quad 
(f^T)^{cb}x_b= x^c
\quad (\text{so }
f_{ba}(f^T)^{cb}=\delta^c_a)
\eeq
where $a, b, c\in \{1,2, \ldots, n\}$
and $x_a$ is any tensor.

$Sp(n)$ leaves
invariant the following skew symmetric quadratic form:

\beq
h(x)=
f_{a b}x^a x^b
\eeq
where $
a,b\in\{1, \ldots, n\}$.
Thus 

\beq
h(Gx)= h(x)
\eeq

\beq
 f_{ab}G\indices{^a_{a'}}
G\indices{^b_{b'}}x^{a'} x^{b'} =
f_{{a'} {b'}}x^{a'} x^{b'}
\implies 
f_{ab}G\indices{^a_{a'}}
G\indices{^b_{b'}} = f_{{a'} {b'}}
\implies
G^Tf G= f
\eeq


In this chapter (and in this book),
we will point the arrows in a birdtrack so that the birdtrack is a DAG. Cycles that make the birdtrack not acyclic will
have a segment in red. Without that
red segment, the birdtrack becomes  acyclic. The reason we follow this arrow convention is that it
promotes acyclic birdtracks which are more
akin to bnets. We will eschew undirected birdtracks for the same reason: bnets are directed.

Let

\beq
\myboxed{f\indices{_a^b}
 =\delta_a^b},\quad
\xymatrix{
&\ar[l]f
&\ar[l]
}=
\xymatrix{
&\ar[l]
}
\eeq

\beq
\myboxed{f\indices{^a_b}
 =\delta^a_b},\quad
\xymatrix{
\ar[r]
&\ar[r]f
&
}=
\xymatrix{
\ar[r]
&
}
\eeq



\beq
\myboxed{
f\indices{_a_c} (f^T)\indices{^c ^b}=\delta_a^b}
\quad
\xymatrix{
&\ul{f}
\ar[l]\ar[r]
&\ol{f}^T
&\ar[l]
}=\;
\xymatrix{&\ar[l]}
\label{eq-ol-ul-eg-sym}
\eeq
Note that we used

\beq
\ul{f} = [f_{ab}],\quad
\ol{f} = [f^{ab}], \quad f^T=-f
\eeq
We could write Eq.(\ref{eq-ol-ul-eg-sym})
without the overline  and underline on
$f$. Those f-decorations are redundant as omitting them
would not introduce any
ambiguity. However, we will
use them  because
they make spotting errors
in the arrow directions easier.

The generators of 
symplectic
 groups will be represented by:


\beq
(T_i)\indices{_a^b}=
\bcen
\xymatrix{
&\ar@{~}@[green][d]&&
\\
&\ar[l] T_i
&\ar[l]
}
\ecen
\eeq
We will also use

\beq
(T_i)\indices{^a_b}=
\bcen
\xymatrix{
&\ar@{~}@[green][d]&
\\
&\ar@{<-}[l] \ol{f}^TT_i\ul{f}
&\ar@{<-}[l]
}
\ecen
(T_i)\indices{_a_b}=
\bcen
\xymatrix{
&\ar@{~}@[green][d]&
\\
& T_i\ul{f}\ar[l]\ar[r]
&
}
\ecen
(T_i)\indices{^a^b}=
\bcen
\xymatrix{
&\ar@{~}@[green][d]&
\\
&\ar@{<-}[l] \ol{f}^TT_i
&\ar[l]
}
\ecen
\eeq

For $G\in Sp(n)$, $f^TG^TfG=1$ with
$G=e^{iT_i\eps_i}$ where $\eps_i\in\RR$.
Hence, the generators $T_i$ 
must satisfy

\beq
\underbrace{f^T T_i^T f}
_{(f^T T_i f)^T} = -T_i
\implies T_i^T = fT_if
\implies (T_i f)^T = T_i  f
\eeq

\beq
\myboxed{
(T_i f)_{ab}
=
(T_if)_{ba}
}
\bcen
\xymatrix{
&\ar@{~}[d]&
\\
a
&T_if\ar[l]\ar[r]
&b
}
\ecen
=
\bcen
\xymatrix{
a
&T_i f\ar[l]\ar[r]
&b
\\
&\ar@{~}[u]
&
}
\ecen
\eeq

$f_a^b=\delta_a^b$
is obviously an invariant matrix. 
$f_{ab}$ must be invariant too, so

\beq
\begin{array}{l}
\myboxed{\underbrace{
(T_i)\indices{_a^c}f\indices{_c _b}
+
(T_i)\indices{_b^c}
f\indices{_a_c}=0}_
{
(T_i f)_{ab}
=
(T_if)_{ba}
}}
\\
\underbrace
{\bcen
\xymatrix{
&\ar@{~}[d]&&
\\
a&\ar[l] T_i
&\ar[l]
\ul{f}\ar[r]
&b
}
\ecen}_{(T_i f)_{ab}}
+
\underbrace{\bcen
\xymatrix{
a&\ar[l] \ul{f}\ar[r]
&
T_i\ar[r]
&b
\\
&&\ar@{~}[u]
}
\ecen}_{-(T_if)_{ba}}
=0
\end{array}
\label{eq-tf-is-sym}
\eeq
Hence, the invariance condition Eq.(\ref{eq-tf-is-sym}) reduces to
to the statement 
that $(T_i f)_{ab}$
is symmetric.









The anti-symmetrizer
$\cala_2$ 
is an invariant tensor (see Section \ref{sec-invariance-sp-ap}).
Other projectors of the $V\otimes V$ are not 
invariant tensors.
Therefore, we must have

\beq
\bcen
\xymatrix@R=1pc@C=2pc{
&&&\ar@/_1pc/[dl]
\\
&T_i \ul{f}\ar@{~}[r]
\ar@/_1pc/[ul]
\ar@/^1pc/[dl]
&\ol{f}^TT_i
&
\\
&
&&\ar@/^1pc/[ul]
}
\ecen
=
\bcen
\xymatrix{
&\cala_2\ar@2{-}[d]
\ar[l]
&\ar[l]
\\
&\ar[l]&\ar[l]
}
\ecen
\eeq

For $SO(n)$ and $O(n)$,
the dimension $N$ of the adjoint rep 
(= number of generators) is

\beq
N = \frac{n(n-1)}{2} = \xymatrix{&&\ar@{~}[ll]
\ar@[red]@/_1pc/[ll]}
\eeq
If you take
an $n\times n$ matrix and remove
its diagonal, this $N$ is the
number of entries in the upper (or
lower) triangular sector
of the matrix. Recall that
for $U(n)$, $N=n^2$,
and for $SU(n)$, $N=n^2-1$.
So for $U(n)$ (or $SU(n)$),
there is a generator for each
entry
(or each entry minus one)
of an $n\times n$ matrix.


\begin{claim}
\beq
\begin{array}{l}
\myboxed{
\Gamma_{fun}\delta^b_a=
\sum_i
(T_iT_i)\indices{_a^b}
 = \frac{n-1}{2}\delta_a^b}
\\
\sum_i
\bcen
\xymatrix{
a
&T_i\ar[l]
&T_i\ar@{~}@/_2pc/[l]|i
\ar[l]
&b  \ar[l]
}\ecen
=
\left(\frac{n-1}{2}\right)
\xymatrix{
a&\ar[l]|\bullet b
}
\end{array}
\label{eq-wavy-arc-son-ortho}
\eeq
\end{claim}
\proof

\beqa
(T_iT_i)\indices{_a^b}
&=&
\bcen
\xymatrix{
a
&T_i\ul{f}\ar[l]
&\ol{f}^TT_i
\ar@{~}@/_2pc/[l]|i
\ar@{<-}[l]
&\ar[l]b
}\ecen
\\
&=&
\bcen
\xymatrix@R=1pc@C=1pc{
a&&&\ar@/_1pc/[ld]b
\\
&T_i\ul{f}\ar@/_1pc/[lu]
&\ol{f}^TT_i\ar@{~}[l]
\ar@{<-}@/_1pc/[rd]&
\\
\ar@{<-}@/_1pc/[ru]&&&
\ar@/^1pc/[lll]
}
\ecen
\\
&&\nonumber
\\
&=&
\frac{1}{2}\left[
\bcen
\xymatrix@C=3pc{
&\ar[l]|\bullet
\\
&
\ar[l]|\bullet
\ar@{-}@/^1pc/[l]}
\ecen
-
\bcen
\xymatrix@C=3pc{
\ar@{<-}[dr]
&
\\
\ar@{<-}[ur]
&
\ar@{-}@/^1pc/[l]}
\ecen
\right]
\\
&&\nonumber
\\
&=&\left(\frac{n-1}{2}\right)
\xymatrix{
a&\ar[l]|\bullet b
}
\eeqa
\qed





\section{$V_{def}\otimes V_{def}$ Decomposition}

Define
\beq
M\indices{_a_b^c^d}
=
\bcen
\xymatrix@R=1pc@C=1pc{
a
&
&
&\ar@/_1pc/[dl]
d
\\
&\ul{f}\ar@[green]@/_1pc/[ul]
&\ol{f}^T
\ar@{<-}@/_1pc/[dr]
&
\\
\ar@{<-}@/_1pc/[ur]
b
&
&
&c
}
\ecen
\eeq



Note that $M$ is antisymmetric:
\beqa
\cala_2 M
&=&
\bcen
\xymatrix@R=1pc@C=1pc{
&\cala_2\ar@2{-}[dd]\ar[l]
&
&
&\ar@/_1pc/[dl]
\\
&
&\ul{f}\ar@/_1pc/[ul]
&\ol{f}^T\ar@{<-}@/_1pc/[dr]
&
\\
&\ar@{<-}@/_1pc/[ur] \ar[l]
&
&
&
}
\ecen
\\
&=&
\frac{1}{2}
\left[
\bcen
\xymatrix@R=1pc@C=1pc{
&\ar[l]
&
&
&\ar@/_1pc/[dl]
\\
&
&\ul{f}\ar@/_1pc/[ul]
&\ol{f}^T\ar@{<-}@/_1pc/[dr]
&
\\
&\ar@{<-}@/_1pc/[ur]\ar[l]
&
&
&
}
\ecen
-
\bcen
\xymatrix@R=1pc@C=1pc{
&\ar@{<->}[dd]\ar[l]
&
&
&\ar@/_1pc/[dl]
\\
&
&\ul{f}\ar@/_1pc/[ul]
&\ol{f}^T\ar@{<-}@/_1pc/[dr]
&
\\
&\ar@{<-}@/_1pc/[ur]\ar[l]
&
&
&
}
\ecen
\right]
\\
&=& M
\eeqa
Since $M$ is ant-symmetric, 
only the anti-symmetric
 space decomposes.


Note also that
\beq
M^2=nM
\eeq
Hence, $(M-n)M=0$
so $M$ has two eigenvalues $\lam=0,n$.

Next we will use the following equation from Chapter \ref{ch-reducibility}
\footnote{Note that this equation
projects to zero all eigenvalues except one.}
to obtain a projection (PO)
operator for each eigenvalue

\beq 
P_i = \sum_{j\neq i}
\frac{M-\lam_j}{\lam_i-\lam_j}
\eeq

Below, we use the following
traces to evaluate the traces
of our projection operators
\beq
\tr(M)=
\bcen
\xymatrix@R=1pc@C=1pc{
a
&
&
&\ar@/_1pc/[dl]
d
\ar@[red]@{-}@/_1pc/[lll]
\\
&\ul{f}\ar@/_1pc/[ul]
&\ol{f}^T\ar@{<-}@/_1pc/[dr]
&
\\
\ar@{<-}@/_1pc/[ur]
b
&
&
&c
\ar@[red]@{-}@/^1pc/[lll]
}
\ecen
=\tr(\ul{f}\ol{f}^T)= n
\eeq

\beqa
\tr(\cala_2)
&=&
\frac{1}{2}
\left[
\bcen
\xymatrix@R=1.5pc@C=1.5pc{
&
&\ar[ll]
\ar@[red]@{-}@/_.8pc/[ll]
\\
&
&\ar[ll]
\ar@[red]@{-}@/^.8pc/[ll]
}
\ecen
-
\bcen
\xymatrix@R=1pc@C=1.5pc{
&\ar@{<->}[d]\ar[l]
&\ar[l]
\ar@[red]@{-}@/_.8pc/[ll]
\\
&\ar[l]
&\ar[l]
\ar@[red]@{-}@/^.8pc/[ll]
}
\ecen
\right]
\\
\nonumber
\\
&=&
\frac{1}{2}(n^2-n)=\frac{1}{2}n(n-1)
\eeqa

\beqa
\tr(\cals_2)
&=&
\frac{1}{2}
\left[
\bcen
\xymatrix@R=1pc@C=1.5pc{
&
&\ar[ll]
\ar@[red]@{-}@/_.8pc/[ll]
\\
&
&\ar[ll]
\ar@[red]@{-}@/^.8pc/[ll]
}
\ecen
+
\bcen
\xymatrix@R=1.5pc@C=1.5pc{
&\ar@{<->}[d]\ar[l]
&\ar[l]
\ar@[red]@{-}@/_.8pc/[ll]
\\
&\ar[l]
&\ar[l]
\ar@[red]@{-}@/^.8pc/[ll]
}
\ecen
\right]
\\
\nonumber
\\
&=&
\frac{1}{2}(n^2+n)=\frac{1}{2}n(n+1)
\eeqa


\begin{enumerate}
\item Singlet (Anti-symmetric $\cala_2 P_S=0$) PO
\beq
P_S = \frac{1}{n}
\ul{f}\indices{_a_b}
(\ol{f}^T)\indices{^c^d}
=
\frac{1}{n}
\bcen
\xymatrix@R=1pc@C=1pc{
a
&
&
&\ar@/_1pc/[dl]
d
\\
&\ul{f}\ar@/_1pc/[ul]
&\ol{f}^T
\ar@{<-}@/_1pc/[dr]
&
\\
\ar@{<-}@/_1pc/[ur]
b
&
&
&c
}
\ecen
\eeq

\beq
\tr(P_S)= \frac{1}{n}\tr(M)=1
\eeq


\item Traceless Anti-symmetric PO\footnote{Traceless here refers to 
$P\indices{_a^a_c^d}V\indices{_d^c}=(PV)\indices{_a^a}=0$ for any vector $V\indices{_d^c}$. It does not refer 
to
$P\indices{_a^b_b^a}=0$}

\beqa
P_{TA}&=&
\frac{1}{2}
(\delta\indices{_a^d}\delta\indices{_b^c}
- \delta\indices{_b^d}\delta\indices{_a^c})
- \frac{1}{n}
\ul{f}\indices{_a_b}
(\ol{f}^T)\indices{^c^d}
\\
&=&
\bcen
\xymatrix{
a
&\cala_2\ar@2{-}[d]\ar[l]
&\ar[l] d
\\
b
&\ar[l]
&\ar[l]
c
}
\ecen
-\frac{1}{n}
\bcen
\xymatrix@R=1pc@C=1pc{
a
&
&
&\ar@/_1pc/[dl]
d
\\
&\ul{f}\ar@/_1pc/[ul]
&\ol{f}^T\ar@{<-}@/_1pc/[dr]
&
\\
\ar@{<-}@/_1pc/[ur]
b
&
&
&c
}
\ecen
\eeqa

\beq
\tr(P_{TA})= \frac{1}{2}(n^2-n-2)=
\frac{1}{2}(n+1)(n-2)
\eeq

\item Symmetric PO

\beqa
P_{SYM}&=&
\frac{1}{2}
(\delta\indices{_a^d}\delta\indices{_b^c}
+ \delta\indices{_b^d}\delta\indices{_a^c})
\\
&=&
\bcen
\xymatrix{
a
&\cals_2\ar@2{-}[d]\ar[l]
&\ar[l] d
\\
b
&\ar[l]
&\ar[l]
c
}
\ecen
\eeqa

\beq
\tr(P_{SYM}) = 
\frac{1}{2}(n^2+n)=
\frac{1}{2}n(n+1)
\eeq
\end{enumerate}

Clearly, 

\beq
P_{SYM}^2 =P_{SYM}, \quad P_S^2=P_S
\eeq
Note that 

\beq
P_{TA}P_S =(\cala_2 -P_S)P_S=P_S-P_S=0
\eeq
Hence

\beq
P_{TA}^2
=(\cala_2 -P_{S})^2
= P_{TA}
\eeq

$P_{SYM}$ is the only of the 3 POs
($P_{SYM}$, $P_S$, $P_{TA} $)
that
is an invariant tensor so

\beq
\bcen
\xymatrix@R=1pc@C=1pc{
&
&
&\ar@/_1pc/[dl]
\\
&T_i\ul{f}\ar@/_1pc/[ul]
&\ol{f}^TT_i
\ar@{<-}@/_1pc/[dr]
\ar@{~}[l]
&
\\
\ar@{<-}@/_1pc/[ur]
&
&
&
}
\ecen
=
\bcen
\xymatrix{
&\cals_2\ar@2{-}[d]\ar[l]
&\ar[l] 
\\
&\ar[l]
&\ar[l]
}
\ecen
\eeq

\begin{claim}
The Clebsch-Gordan series for $V\otimes V$ (i.e., decomposition of 
$V\otimes V$) is

\beq
\begin{array}{ccccccc}
\overbrace{V\otimes V}
^\calv
&=
&P_S\calv
&\oplus
&P_{SYM}\calv
&\oplus
&P_{TA}\calv
\\
\ydiagram{1}
\otimes\ydiagram{1}
&=
&\bullet
&\oplus
&\ydiagram{2}
&\oplus
&\ydiagram{1,1}
\\
n^2
&=
& 1
&+
&\frac{1}{2}n(n+1)
&+
&\frac{1}{2}(n+1)(n-2)
\end{array}
\eeq

The projection operator tree is
\begin{center}
\begin{minipage}{2cm}
\dirtree{%
.1 $P_{ANTI}$.
.2 $P_S$.
.2 $P_{TA}$.
.1 $P_{SYM}$.
}
\end{minipage}
\end{center}
\end{claim}
\proof
\qed