\chapter{Recoupling Identities}
\label{ch-recoupling}
This chapter is based on Cvitanovic's Birdtracks book Ref. \cite{birdtracks-book}.

In this chapter, we 
will refer to the following
2 birdtracks
as  s and t channels.\footnote{My mnemonic to remember which is which: {\bf s-channel}: particle \ul{synergy}, energy from particles coming together, {\bf t-channel}: particle \ul{trade}.}

\beq
\begin{array}{ccc}
\xymatrix@R=1pc@C=1.5pc{
&
&
&\ar[dl]
\\
&
\ar[lu]
\ar[ld]
&\ar@{~}[l]
&
\\
&
&
&\ar[ul]
}
&
\xymatrix{
&\ar[l]
\ar@{~}[d]&\ar[l]
\\
&\ar[l]&\ar[l]
}\\
\text{s-channel}
&\text{t-channel}
\end{array}
\eeq
This terminology comes from High Energy Physics, where these birdtracks are used to define the so called Mandelstam variables.
The Mandelstam variables measure the energy 
of particles in various birdtracks.




\section{Parallel Channels to 
Sum of t-channels}



Clebsch-Gordan (CG) coefficients were
introduced in Chapter \ref{ch-clebsch-gordan}.
Define the CG coefficients node

\beq 
C\indices{
_{\lam a}
^{\nu b}
^{\mu c}
}
=
\bcen
\xymatrix{
&&\mu c\ar@[green][ld]
\\
\lam a
&C\indices{
_{\lam}
^{\nu}
^{\mu}
}\ar@[red][l]
\\
&&\nu b\ar@[blue][lu]
}
\ecen
=
\bcen
\xymatrix{
&&\mu c\ar@[green][l]
\\
\lam a
&C\indices{
_{\lam}
^{\nu}
^{\mu}
}\ar@[red][l]
\ar@2{-}[u]
\ar@2{-}[d]
\\
&&\nu b\ar@[blue][l]
}
\ecen
\eeq

Note that

\beq
\bcen
\xymatrix{
&&\mu \ar@[green][l]
\\
\lam 
&
C_\lam\ar@[red][l]
\ar@2{-}[u]
\ar@2{-}[d]
\\
&&\nu\ar@[blue][l]
}
\ecen
\neq
\bcen
\xymatrix{
&&\mu \ar@[green][ldd]
\\
\lam 
&
C_\lam\ar@[red][l]
\ar@2{-}[u]
\ar@2{-}[d]
\\
&&\nu\ar@[blue][luu]
}
\ecen
\eeq

Note that we are defining 
the CG coefficient $C_\lam$ 
so that the $\lam$
rep particle is created in an s-channel 
by converging $\mu$ and $\nu$ rep particles.
When we define the
generators $T^i_\lam$, the $i$
(gluon, adj-rep particle) 
is in a t-channel
emanating from 
 incoming
and outgoing def-rep particles. 
Another big difference between 
$C_\lam$  and
$T^i_\lam$ is that $T^i_\lam$ 
is assumed to be Hermitian,
whereas $C_\lam$
is not Hermitian in general. $C_\lam$ is not
even a square matrix in general.


In this chapter, we won't use implicit summation over Greek indices.


In this section, 
sometimes
instead of labelling arrows by a lower case Greek letter denoting its  rep, we will disclose an arrow's rep by a color,
according to the following 
rep-to-color code.
\beq
\lam: red,\quad
\mu: green,\quad
\nu: blue
\eeq

According to Chapter \ref{ch-clebsch-gordan},
the CG coefficient $C_\lam$ satisfies

\beq
C_\lam C^\dagger_\lam = P_\lam
\eeq

\beq
\tr(P_\lam)=d_\lam
\eeq
where $P_\lam$
is the projection operator onto
the vector space of the rep $\lam$ and $d_\lam$ is the dimension of that vector space.



Note that if we divide
$C_\lam$ by $\sqrt{d_\lam}$,
then

\beq
\tr\left(\frac{C_\lam}
{\sqrt{d_\lam}} 
\frac{C^\dagger_\lam}
{\sqrt{d_\lam}} 
\right)=1
\eeq

Define

\beq
{\color{red}P_\lam}
=
\bcen
\xymatrix@C=3pc{
&\ar@[green][l]
&
&\ar@[green][l]
\\
&C^\dagger
_\lam
\ar@2{-}[u]
\ar@2{-}[d]
&
C
_\lam
\ar@2{-}[u]
\ar@2{-}[d]
\ar@[red][l]
\\
&\ar@[blue][l]
&
&\ar@[blue][l]
}
\ecen
\eeq

\beq
{\color{green}P_\mu}
=\frac{d_\mu}{d_\lam}
\bcen
\xymatrix@R=1pc{
&&&&&
\\
&\ar@[green]
`l[lu]
`[urrrr]
`[rrr]
`[rr]
[rr]
&
&
&
\\
&C^\dagger_\lam
\ar@2{-}[u]
\ar@2{-}[d]
&
&C_\lam
\ar@2{-}[u]
\ar@2{-}[d]
\ar@[red]
`l[ld]
`[ddr]
[ddr]
&
&
\\
&\ar@[blue][l]
&
&
&\ar@[blue][l]
\\
\ar@[red]
`r[rru]
`[uur]
[uur]
&
&
&
&
}
\ecen
\eeq

\beq
{\color{blue}P_\nu}
=\frac{d_\nu}{d_\lam}
\bcen
\xymatrix@R=1pc{
\\
\ar@[red]
`r[drr]
`[ddrr]
[ddr]
&
&
&
&
\\
&\ar@[green][l]
&
&
&\ar@[green][l]
\\
&C^\dagger_\lam
\ar@2{-}[u]
\ar@2{-}[d]
&
&C_\lam
\ar@2{-}[u]
\ar@2{-}[d]
\ar@[red]
`l[lu]
`[uur]
[uur]
&
&
\\
&\ar@[blue]
`l[ld]
`[drrrr]
`[rrr]
`[rr]
[rr]
&
&
&
\\
&&&&&
}
\ecen
\eeq



One can check that these
operators are projection operators normalized to the dimennsion
of their rep; i.e., 
for $\Omega\in\{\lam, \mu, \nu\}$,

\beq
P_\Omega^2=P_\Omega
\eeq
and 

\beq
\tr(P_\Omega)= d_\Omega
\eeq


The normalization of the projectors $P_\Omega$ can be remembered if 
one takes the denominator $d_\lam$ and splits it into two factors of $\sqrt{d_\lam}$
and puts one $\sqrt{d_\lam}$
under $C_\lam$
and the other under $C^\dagger_\lam$. Then
one \qt{trades} $\frac{C_\lam}{\sqrt{d_\lam}}$
by
$\frac{C_\mu}{\sqrt{d_\mu}}$
or
$\frac{C_\nu}{\sqrt{d_\nu}}$.

Next we define a scaled version of the CG coefficients $C_\lam$
as follows


\beq
\bcen
\xymatrix{
&&\mu \ar@[green][l]
\\
\lam 
&C_\lam\ar@[red][l]
\ar@2{-}[u]
\ar@2{-}[d]
\\
&&\nu\ar@[blue][l]
}
\ecen
=
\frac{1}
{\sqrt{
\kten{\lam}{\nu}{\mu}
}}
\bcen
\xymatrix{
&&\mu \ar@[green][l]
\\
\lam 
&
\ger{C}_\lam\ar@[red][l]
\ar@2{-}[u]
\ar@2{-}[d]
\\
&&\nu\ar@[blue][l]
}
\ecen
\eeq
The {\bf scaled CG coefficients} $\ger{C}_\lam$ satisfy


\beq
\xymatrix{
&\ger{C}_\lam
\ar[l]|\lam
&\ger{C}^\dagger_\s
\ar@/_1.5pc/[l]|\mu
\ar@/^1.5pc/[l]|\nu
&\ar[l]|\s
}
=
\kten{\lam}{\nu}{\mu}
\delta(\lam, \s)
\xymatrix{
&\bullet\ar[l]|\lam
&\ar[l]|\s
}
\eeq
Therefore


\beq
\trij{\mu}{\lam}{\nu}=
\kten{\lam}{\nu}{\mu} d_\lam 
\label{eq-trij-is}
\eeq

The projection
operators $P_\Omega$
for $\Omega\in\{\lam, \mu, \nu\}$
can be expressed in a more
symmetrical form using
nodes for
the scaled CG coefficients as follows

\beq
P_\lam=
\frac{1}{\kten{\lam}{\nu}{\mu}}
\bcen
\xymatrix{
&
&
&\ar[dl]|\mu
\\
&\ger{C}_\lam^\dagger
\ar[lu]|\mu
\ar[ld]|\nu
&\ger{C}_\lam\ar[l]|\lam
&
\\
&
&
&\ar[ul]|\nu
}
\ecen
\eeq


\beq
P_\mu=
\frac{1}{\kten{\mu}{\lam}{\nu}}
\bcen
\xymatrix{
&
&
&\ar[dl]|\nu
\\
&\ger{C}_\mu^\dagger
\ar[lu]|\nu
\ar[ld]|\lam
&\ger{C}_\mu\ar[l]|\mu
&
\\
&
&
&\ar[ul]|\lam
}
\ecen
\eeq

\beq
P_\nu=
\frac{1}{\kten{\nu}{\mu}{\lam}}
\bcen
\xymatrix{
&
&
&\ar[dl]|\lam
\\
&\ger{C}_\nu^\dagger
\ar[lu]|\lam
\ar[ld]|\mu
&\ger{C}_\nu\ar[l]|\nu
&
\\
&
&
&\ar[ul]|\mu
}
\ecen
\eeq

The CG series for 
$V_\mu \otimes V_\nu= \sum_\lam V_\lam$ can
be expressed in terms of
birdtracks as follows
\beq
\bcen
\xymatrix{
&\bullet\ar[l]
&\ar[l]|\mu
\\
&\bullet\ar[l]
&\ar[l]|\nu
}\ecen
=\sum_\lam P_\lam
=
\sum_\lam
\frac{d_\lam}
{
\trij{\mu}{\lam}{\nu}
}
\bcen
\xymatrix{
&
&
&\ar[dl]|\mu
\\
&\ger{C}_\lam^\dagger
\ar[lu]|\mu
\ar[ld]|\nu
&\ger{C}_\lam\ar[l]|\lam
&
\\
&
&
&\ar[ul]|\nu
}
\ecen
\eeq
This CG series
expresses two {\bf parallel
channels} as a sum of s-channels.

The CG series for
$N>2$ parallel channels $V_{\mu_1}\otimes  V_{\mu_2}\otimes \ldots \otimes V_{\mu_N}
= \sum_\lam V_\lam$
is obtained
by combining pairs of vector spaces
recursively. The series depends on what vector space pairs are chosen in what order. For example, we can use\footnote{For succinctness, we are dropping the rep labels $\mu, \lam$ from
$\kappa\indices{_\nu^{\mu\lam}}$, but
the $\kappa_\nu$ still depends on them.}

\beq
\bcen
\xymatrix@R=1pc{
&\ar[l]
\\
\\&\ar[l]
\\
\\&\ar[l]
\\
\\&\ar[l]
}
\ecen
=
\sum_{\lam, \mu, \nu}
\frac{1}{\kappa_\lam}
\frac{1}{\kappa_\mu}
\frac{1}{\kappa_\nu}
\bcen
\xymatrix@R=1pc{
&\ar[l]
&&&&&&\ar[l]
\\
&\ger{C}_\lam^\dagger
\ar@2{-}[u]
\ar@2{-}[d]
&\ar[l]
&
&&&\ger{C}_\lam
\ar@2{-}[u]
\ar@2{-}[d]
\ar[l]
\\
&\ar[l]
&\ger{C}_\mu^\dagger
\ar@2{-}[u]
\ar@2{-}[d]
&\ar[l]
&&\ger{C}_\mu\ar[l]
\ar@2{-}[u]
\ar@2{-}[d]
&&\ar[l]
\\
&&\ar[ll]
&
\ger{C}_\nu^\dagger
\ar@2{-}[u]
\ar@2{-}[d]
&\ger{C}_\nu\ar[l]
\ar@2{-}[u]
\ar@2{-}[d]
&&&\ar[ll]\ar[ll]
\\
&&&\ar[lll]
&&&&\ar[lll]
}
\ecen
\label{eq-four-to-one-loop}
\eeq
Another possibility is

\beq
\bcen
\xymatrix@R=1pc{
&\ar[l]
\\
\\&\ar[l]
\\
\\&\ar[l]
\\
\\&\ar[l]
}
\ecen
=
\sum_{\lam, \mu, \nu}
\frac{1}{\kappa_\lam}
\frac{1}{\kappa_\mu}
\frac{1}{\kappa_\nu}
\bcen
\xymatrix@R=1pc
{
&\ar[l]
&
&
&
&\ar[l]
&
&
\\
&\ger{C}_\lam^\dagger
\ar@2{-}[u]
\ar@2{-}[d]
&\ar[l]
&
&\ar[l]
\ger{C}_\lam
\ar@2{-}[u]
\ar@2{-}[d]
&
&
&
\\
&\ar[l]
&
&
&
&\ar[l]
&
&
\\
&
&\ger{C}_\nu^\dagger
\ar@2{-}[uu]
\ar@2{-}[dd]
&\ar[l]\ger{C}_\nu
\ar@2{-}[uu]
\ar@2{-}[dd]
&
&
&
\\
&\ar[l]
&
&
&
&\ar[l]
&
&
\\
&
\ger{C}_\mu^\dagger
\ar@2{-}[u]
\ar@2{-}[d]
&\ar[l]
&
&\ger{C}_\mu
\ar@2{-}[u]
\ar@2{-}[d]
\ar[l]
&
&
\\
&\ar[l]
&
&
&
&\ar[l]
&
&
}\ecen
\eeq


\section{t-channel to Sum of s-channels}

We can express a t-channel
as a sum over s-channels as follows



\begin{align}
\bcen
\xymatrix{
&\ger{C}^\dagger_\mu
\ar[l]|\s
\ar[d]|\omega
&\ar[l]|\mu
\\
&\ger{C}_\rho
\ar[l]|\rho
&\ar[l]|\nu
}
\ecen
&=
\sum_\lam
\left[
\begin{array}{l}
\frac{d_\lam}
{\trij{\s}{\lam}{\rho}
}
\frac{d_\lam}
{\trij{\mu}{\lam}{\nu}
}
\\
*
\bcen
\xymatrix{
&
&
&\ger{C}^\dagger_\mu\ar[dl]|\s
\ar[dd]|\omega
&
&
&\ar[dl]|\mu
\\
&\ger{C}^\dagger_\lam\ar[lu]|\s
\ar[ld]|\rho
&\ger{C}_\lam\ar[l]|\lam
&
&\ger{C}^\dagger_\lam
\ar[ul]|\mu
\ar[dl]|\nu
&\ger{C}_\lam\ar[l]|\lam
&
\\
&
&
&\ger{C}_\rho\ar[ul]|\rho
&
&
&\ar[ul]|\nu
}
\ecen
\end{array}
\right]
\\
&=
\sum_\lam
\left[
\Phi_\lam\left(
\bcen
\xymatrix{
&\ger{C}^\dagger_\mu
\ar[l]|\s
\ar[d]|\omega
&\ar[l]|\mu
\\
&\ger{C}_\rho
\ar[l]|\rho
&\ar[l]|\nu
}
\ecen
\right)
\bcen
\xymatrix{
&&&\ar[ld]|\mu
\\
&\ger{C}_\lam^\dagger
\ar[lu]|\s
\ar[ld]|\rho
&\ger{C}_\lam
\ar[l]|\lam
&
\\
&&&\ar[lu]|\nu
}
\ecen
\right]
\label{eq-gen-t-to-sum-of-s}
\end{align}
where

\begin{align}
\Phi_\lam\left(
\bcen
\xymatrix{
&\ger{C}^\dagger_\mu
\ar[l]|\s
\ar[d]|\omega
&\ar[l]|\mu
\\
&\ger{C}_\rho
\ar[l]|\rho
&\ar[l]|\nu
}
\ecen
\right)
&=
\frac{d_\lam}
{\trij{\s}{\lam}{\rho}
}
\frac{d_\lam}
{\trij{\mu}{\lam}{\nu}
}
\frac{
\sixj{\omega}{\rho}{\nu}{\lam}{\mu}{\s}
}
{d_\lam}
\\
&=
d_\lam
\frac{
\sixj{\omega}{\rho}{\nu}{\lam}{\mu}{\s}
}
{\trij{\s}{\lam}{\rho}
\trij{\mu}{\lam}{\nu}
}
\label{eq-phi-3j-6j}
\end{align}

\section{Wigner $3n-j$ Coefficients/DAGs}

A DAG with no incoming or outgoing arrows is called an {\bf isolated DAG}. Physicists sometimes call it a {\bf vacuum bubble} also.
On the right hand side of 
Eq.(\ref{eq-phi-3j-6j}), the isolated DAG 
with two $\ger{C} $ is called
a {\bf $3j$ coefficient/DAG},
and the one with 4 $\ger{C}$ is called
a {\bf $6j$ coefficient/DAG}.
So far we seen $3j$ and $6j$
coefficients/DAGs. Atomic
physicists
define
 {\bf Wigner $3n-j$ coefficients/DAGs},
  for $n=1,2, 3, \dots$.
They are called that because they
describe  how to \qt{add} $3n$
angular momenta $j$.
  There is
  only one topological distinct $3j$
DAG but two $6j$ DAGs, 
five $9j$ DAGs, and so  on.


In Chapter \ref{ch-casimir},
we discussed Casimir suns.
Next we show that they
can always be expressed in
terms of $3j$ and $6j$
coefficients
and CG coefficients.
We proceed as we did in Eq.(\ref{eq-four-to-one-loop})
but here we use the most general 
t-channel to sum of s-channels conversion Eq.(\ref{eq-gen-t-to-sum-of-s}).

\begin{align}
\bcen
\xymatrix@R=.5pc{
&\ar[l]\ar@{<-}[dd]T^{i_1}&
\\&&
\\&\ar[l]\ar@{<-}[dd]T^{i_2}&
\\&&
\\&\ar[l]\ar@{<-}[dd]T^{i_3}&
\\&&
\\&\ar[l]T^{i_4}
\ar`r[ru]
`[uuuuuu]
[uuuuuu]
&
}
\ecen
&=
\sum_{\lam, \mu}
{\Phi_\lam}
{\Phi_\mu}
\bcen
\xymatrix@R=1pc{
&\ar[l]
&
&
\\
&\ger{C}_\lam^\dagger
\ar@2{-}[u]
\ar@2{-}[d]
&\ar[l]
\ger{C}_\lam
&
\\
&\ar[l]
&
&
\\
&
&\ar@{<-}[d]T^{i_3}\ar[ll]
\ar[uu]
&
\\
&&
\ar`r[ru]`[uuu][uuu]T^{i_4}
\ar[ll]
&
}
\ecen
\\
&=
\sum_{\lam, \mu}
{\Phi_\lam}
{\Phi_\mu}
\bcen
\xymatrix@R=1pc{
&\ar[l]
&
&
\\
&\ger{C}_\lam^\dagger
\ar@2{-}[u]
\ar@2{-}[d]
&\ar[l]&
\\
&\ar[l]
&\ger{C}_\mu^\dagger
\ar@2{-}[u]
\ar@2{-}[d]
&\ar@{<-}[dd]\ger{C}_\mu
\ar[l]
\\
&&\ar[ll]
&
&
\\
&&&\ar[lll]
\ar`r[ru]`[uu][uu]T^{i_4}
&
}
\ecen
\\
&= 
\sum_{\lam, \mu, \nu}
{\Phi_\lam}
{\Phi_\mu}
{\Phi_\nu}
\bcen
\xymatrix@R=1pc{
&\ar[l]
&&&&
\\
&\ger{C}_\lam^\dagger
\ar@2{-}[u]
\ar@2{-}[d]
&\ar[l]
&
&
\\
&\ar[l]
&\ger{C}_\mu^\dagger
\ar@2{-}[u]
\ar@2{-}[d]
&\ar[l]
&&
\\
&&\ar[ll]
&
\ger{C}_\nu^\dagger
\ar@2{-}[u]
\ar@2{-}[d]
&\ger{C}_\nu\ar[l]
\ar@2{-}[u]
\ar@2{-}[d]
&
\\
&&&\ar[lll]
&
\ar`r[ru]
`[uu]
[uu]
&
}
\ecen
\end{align}
