\chapter{Recoupling Equations: COMING SOON}
\label{ch-recoupling}

\section{Parallel channels to 
t-channels}
\xymatrix{
{\circ} \ar 
`r[d] ^a
`[rr] ^b
`[rr] ^c
`[rrr] ^d
`_dl[drrr]^e
[drrr]^f
& {\circ} & {\circ} & {\circ} \\
{\circ} & {\circ} & {\circ} & {\circ} }


\xymatrix{
&A
\ar
`l[ld]
`[drr]
`[rr]
`[r]
[r]
&B&
\\x&x&x&x}


\beq
\lam =red,\quad
\mu=green,\quad
\nu=blue
\eeq

no implicit sum over Greek indices

\beq P_\lam
C\indices{
_{\lam a}
^{\nu b}
^{\mu c}
}
=
\bcen
\xymatrix{
&&\mu c\ar@[green][ld]
\\
\lam a
&P_\lam C\indices{
_{\lam}
^{\nu}
^{\mu}
}\ar@[red][l]
\\
&&\nu b\ar@[blue][lu]
}
\ecen
=
\bcen
\xymatrix{
&&\mu c\ar@[green][l]
\\
\lam a
&P_\lam C\indices{
_{\lam}
^{\nu}
^{\mu}
}\ar@[red][l]
\ar@2{-}[u]
\ar@2{-}[d]
\\
&&\nu b\ar@[blue][l]
}
\ecen
\eeq

\beq
\calc\indices{
_\lam
^\nu
^\mu
} = P_\lam
C\indices{
_\lam
^\nu
^\mu
}
\eeq

\beq
\calc_\lam \calc^\dagger_\lam=P_\lam^2 = P_\lam
\eeq

\beq
\tr(P_\lam)=d_\lam
\eeq
where $d_\lam$ is the dimension of rep $\lam$.
Actually, $\calc_\lam=P_\lam C_\lam=C_\lam$,
but we make the $P_\lam$ explicit
for pedagogical purposes.

Note that if we divide
$\calc_\lam$ by $\sqrt{d_\lam}$,
then

\beq
\tr\left(\frac{\calc_\lam}
{\sqrt{d_\lam}} 
\frac{\calc^\dagger_\lam}
{\sqrt{d_\lam}} 
\right)=1
\eeq



\beq
{\color{red}\calp_\lam}
=
\bcen
\xymatrix@C=3pc{
&\ar@[green][l]
&
&\ar@[green][l]
\\
&\calc^\dagger
_\lam
\ar@2{-}[u]
\ar@2{-}[d]
&
\calc
_\lam
\ar@2{-}[u]
\ar@2{-}[d]
\ar@[red][l]
\\
&\ar@[blue][l]
&
&\ar@[blue][l]
}
\ecen
\eeq

\beq
\calp_\lam^2=
\calp_\lam
\eeq


\beq
{\color{blue}\calp_\nu}
=\frac{d_\nu}{d_\lam}
\bcen
\xymatrix@R=1pc{
\\
\ar@[red]
`r[drr]
`[ddrr]
[ddr]
&
&
&
&
\\
&\ar@[green][l]
&
&
&\ar@[green][l]
\\
&\calc^\dagger_\lam
\ar@2{-}[u]
\ar@2{-}[d]
&
&\calc_\lam
\ar@2{-}[u]
\ar@2{-}[d]
\ar@[red]
`l[lu]
`[uur]
[uur]
&
&
\\
&\ar@[blue]
`l[ld]
`[drrrr]
`[rrr]
`[rr]
[rr]
&
&
&
\\
&&&&&
}
\ecen
\eeq

\beq
\calp_\nu^2=\calp_\nu
\eeq



\beq
{\color{green}\calp_\mu}
=\frac{d_\mu}{d_\lam}
\bcen
\xymatrix@R=1pc{
&&&&&
\\
&\ar@[green]
`l[lu]
`[urrrr]
`[rrr]
`[rr]
[rr]
&
&
&
\\
&\calc^\dagger_\lam
\ar@2{-}[u]
\ar@2{-}[d]
&
&\calc_\lam
\ar@2{-}[u]
\ar@2{-}[d]
\ar@[red]
`l[ld]
`[ddr]
[ddr]
&
&
\\
&\ar@[blue][l]
&
&
&\ar@[blue][l]
\\
\ar@[red]
`r[rru]
`[uur]
[uur]
&
&
&
&
}
\ecen
\eeq

\beq
\calp_\mu^2=\calp_\mu
\eeq
The normalization of the projectors $\calp_\lam, \calp_\nu, \calp_\mu$ can be remembered if 
one takes the denominator $d_\lam$ and splits it into two factors of $\sqrt{d_\lam}$
and puts one $\sqrt{d_\lam}$
under $\calc_\lam$
and the other under $\calc^\dagger_\lam$. Then
one \qt{trades} $\frac{\calc_\lam}{\sqrt{d_\lam}}$
by
$\frac{\calc_\nu}{\sqrt{d_\nu}}$
or
$\frac{\calc_\mu}{\sqrt{d_\mu}}$.


\newcommand{\trij}[3]{
\xymatrix{
T_#2^\dagger
\ar@/^1pc/[r]|#1
\ar@/_1pc/[r]|#3
&T_#2\ar[l]|#2
}
}

$
\trij{\mu}{\lam}{\nu}
$

\newcommand{\kten}[3]{
K\indices{
_#1^#2^#3}
}

\newcommand{\sixj}[6]{
\xymatrix@R=2pc@C=2pc{
&T_#5^\dagger\ar[d]|#1
\ar@/_1pc/[ddl]|#6
\ar@/^1pc/@{<-}[ddr]|#5
\\
&T_#2
\ar[dl]|#2
\ar@{<-}[dr]|#3
\\
T_#4
&
&T_#4^\dagger
\ar@{<-}[ll]|#4
}
}

\sixj{\lam}{\mu}{\nu}{\omega}{\rho}{\s}

arrow directions 
for specific case being
considered.
they can be changed

\beq
\bcen
\xymatrix{
&&\mu \ar@[green][l]
\\
\lam 
&\calc_\lam\ar@[red][l]
\ar@2{-}[u]
\ar@2{-}[d]
\\
&&\nu\ar@[blue][l]
}
\ecen
=
\frac{1}
{\sqrt{
\kten{\lam}{\nu}{\mu}
}}
\bcen
\xymatrix{
&&\mu \ar@[green][l]
\\
\lam 
&
T_\lam\ar@[red][l]
\ar@2{-}[u]
\ar@2{-}[d]
\\
&&\nu\ar@[blue][l]
}
\ecen
\eeq

\beq
\bcen
\xymatrix{
&&\mu \ar@[green][l]
\\
\lam 
&
T_\lam\ar@[red][l]
\ar@2{-}[u]
\ar@2{-}[d]
\\
&&\nu\ar@[blue][l]
}
\ecen
\neq
\bcen
\xymatrix{
&&\mu \ar@[green][ldd]
\\
\lam 
&
T_\lam\ar@[red][l]
\ar@2{-}[u]
\ar@2{-}[d]
\\
&&\nu\ar@[blue][luu]
}
\ecen
\eeq


\beq
\xymatrix{
&T_\lam
\ar[l]|\lam
&T_\s^\dagger
\ar@/_1.5pc/[l]|\mu
\ar@/^1.5pc/[l]|\nu
&\ar[l]|\s
}
=
\kten{\lam}{\nu}{\mu}
\xymatrix{
&\bullet\ar[l]|\lam
&\ar[l]|\s
}
\eeq


\beq
\trij{\mu}{\lam}{\nu}=
\kten{\lam}{\nu}{\mu} d_\lam 
\eeq


\beq
\calp_\lam=
\frac{1}{\kten{\lam}{\nu}{\mu}}
\bcen
\xymatrix{
&
&
&\ar[dl]|\mu
\\
&T_\lam^\dagger
\ar[lu]
\ar[ld]
&T_\lam\ar[l]|\lam
&
\\
&
&
&\ar[ul]|\nu
}
\ecen
\eeq

\beq
\calp_\mu=
\frac{1}{\kten{\mu}{\lam}{\nu}}
\bcen
\xymatrix{
&
&
&\ar[dl]|\nu
\\
&T_\mu^\dagger
\ar[lu]
\ar[ld]
&T_\mu\ar[l]|\mu
&
\\
&
&
&\ar[ul]|\lam
}
\ecen
\eeq

\beq
\calp_\nu=
\frac{1}{\kten{\nu}{\mu}{\lam}}
\bcen
\xymatrix{
&
&
&\ar[dl]|\lam
\\
&T_\nu^\dagger
\ar[lu]
\ar[ld]
&T_\nu\ar[l]|\nu
&
\\
&
&
&\ar[ul]|\mu
}
\ecen
\eeq

\beq
\bcen
\xymatrix{
&\bullet\ar[l]
&\ar[l]|\mu
\\
&\bullet\ar[l]
&\ar[l]|\nu
}\ecen
=\sum_\lam \calp_\lam
=
\sum_\lam
\frac{d_\lam}
{
\trij{\mu}{\lam}{\nu}
}
\bcen
\xymatrix{
&
&
&\ar[dl]|\mu
\\
&T_\lam^\dagger
\ar[lu]
\ar[ld]
&T_\lam\ar[l]|\lam
&
\\
&
&
&\ar[ul]|\nu
}
\ecen
\eeq

\beq
\bcen
\xymatrix@R=1pc{
&\ar[l]
\\
\\&\ar[l]
\\
\\&\ar[l]
\\
\\&\ar[l]
}
\ecen
=
\sum_{\lam, \mu, \nu}
\frac{1}{K_\lam}
\frac{1}{K_\mu}
\frac{1}{K_\nu}
\bcen
\xymatrix@R=1pc{
&\ar[l]
&&&&&&\ar[l]
\\
&T_\lam^\dagger
\ar@2{-}[u]
\ar@2{-}[d]
&\ar[l]
&
&&&T_\lam
\ar@2{-}[u]
\ar@2{-}[d]
\ar[l]
\\
&\ar[l]
&T_\mu^\dagger
\ar@2{-}[u]
\ar@2{-}[d]
&\ar[l]
&&T_\mu\ar[l]
\ar@2{-}[u]
\ar@2{-}[d]
&&\ar[l]
\\
&&\ar[ll]
&
T_\nu^\dagger
\ar@2{-}[u]
\ar@2{-}[d]
&T_\nu\ar[l]
\ar@2{-}[u]
\ar@2{-}[d]
&&&\ar[ll]\ar[ll]
\\
&&&\ar[lll]
&&&&\ar[lll]
}
\ecen
\eeq

\beq
\bcen
\xymatrix@R=1pc{
&\ar[l]
\\
\\&\ar[l]
\\
\\&\ar[l]
\\
\\&\ar[l]
}
\ecen
=
\sum_{\lam, \mu, \nu}
\frac{1}{K_\lam}
\frac{1}{K_\mu}
\frac{1}{K_\nu}
\bcen
\xymatrix@R=1pc
{
&\ar[l]
&
&
&
&\ar[l]
&
&
\\
&T_\lam^\dagger
\ar@2{-}[u]
\ar@2{-}[d]
&\ar[l]
&
&\ar[l]
T_\lam
\ar@2{-}[u]
\ar@2{-}[d]
&
&
&
\\
&\ar[l]
&
&
&
&\ar[l]
&
&
\\
&
&T_\nu^\dagger
\ar@2{-}[uu]
\ar@2{-}[dd]
&\ar[l]T_\nu
\ar@2{-}[uu]
\ar@2{-}[dd]
&
&
&
\\
&\ar[l]
&
&
&
&\ar[l]
&
&
\\
&
T_\mu^\dagger
\ar@2{-}[u]
\ar@2{-}[d]
&\ar[l]
&
&T_\mu
\ar@2{-}[u]
\ar@2{-}[d]
\ar[l]
&
&
\\
&\ar[l]
&
&
&
&\ar[l]
&
&
}\ecen
\eeq

s-channel, particles shag (have sex),

t-channel, particles have tea

\section{t-channel to s-channels}
\begin{align}
\bcen
\xymatrix{
&T^\dagger_\mu
\ar[l]|\s
\ar[d]|\omega
&\ar[l]|\mu
\\
&T_\rho
\ar[l]|\rho
&\ar[l]|\nu
}
\ecen
&=
\sum_\lam
\left[
\begin{array}{l}
\frac{d_\lam}
{\trij{\s}{\lam}{\rho}
}
\frac{d_\lam}
{\trij{\mu}{\lam}{\nu}
}
\\
*
\bcen
\xymatrix{
&
&
&T^\dagger_\mu\ar[dl]|\s
\ar[dd]|\omega
&
&
&\ar[dl]|\mu
\\
&T^\dagger_\lam\ar[lu]|\s
\ar[ld]|\rho
&T_\lam\ar[l]|\lam
&
&T^\dagger_\lam
\ar[ul]|\mu
\ar[dl]|\nu
&T_\lam\ar[l]|\lam
&
\\
&
&
&T_\rho\ar[ul]|\rho
&
&
&\ar[ul]|\nu
}
\ecen
\end{array}
\right]
\\
&=
\sum_\lam
\left[
\begin{array}{l}
\frac{d_\lam}
{\trij{\s}{\lam}{\rho}
}
\frac{d_\lam}
{\trij{\mu}{\lam}{\nu}
}
\frac{
\sixj{\omega}{\rho}{\nu}{\lam}{\mu}{\s}
}
{d_\lam}
\\
*
\bcen
\xymatrix{
&&&\ar[ld]|\mu
\\
&T_\lam^\dagger
\ar[lu]|\s
\ar[ld]|\rho
&T^\lam
\ar[l]|\lam
&
\\
&&&\ar[lu]|\nu
}
\ecen
\end{array}
\right]
\end{align}

\section{Wigner $3n-j$ coefficients}

we will refer to x
as a $3-j$ symbol,
and to x as a  $6-j$
symbol. Atomic
physicsists
also  define
  $3n-j$ symbols,
  for $n=1,2, 3, \dots$.
They are called that because they
describe  how to \qt{add} $3n$
angular momenta $j$.
  There is
  only one $3-j$
but two $6-j$'s .
five $9-j$s, and so  on.
We only show 
one $3-j$
and one $6-j$.