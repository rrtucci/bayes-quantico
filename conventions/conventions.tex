\chapter{Notational Conventions and Preliminaries}
\label{ch-conventions}

\section{Vector Notation}

$(x_1, x_2, \ldots, x_n) = x^{:n}$

$y^b = \sum_b g^{ba}x^{:n}$

$(y^1, y^2, \ldots, y^n)= gx^{:n}$

Reverse of vector $rev(x^{:n})=
(x_n, x_{n-1},
\ldots, x_1)$

Implicit Summation Convention

\beq
G_a^bx_b = \sum_{b=1}^n
G_a^bx_b
\eeq

\section{Tensors}

\section{Group}

A {\bf group}
$\calg$
is a set of elements
with a multiplication map $\calg\times \calg
\rarrow \calg$
such that


\begin{enumerate}
\item 
the multiplication is {\bf associative
}; i.e., 

\beq
(ab)c = a(bc)
\eeq
for $a,b,c\in\calg$.

\item
there exists an {\bf identity element}
$e\in \calg$
such that 

\beq
ea=ae=a
\eeq
for all $a\in \calg$

\item
for any $g\in\calg$,
there exists an {\bf inverse} $a^{-1}\in \calg$ such that

\beq
aa^{-1}=a^{-1}a=e
\eeq
\end{enumerate}

The number of elements in any set $S$ is denoted by $|S|$. 
$|\calg|$
is called the {\bf order}
of the group.

If multiplication is
{\bf commutative}
(i.e., $ab=ba$ for all $a,b\in\calg$,
the group is said to be {\bf abelian}.

A {\bf subgroup} $\calh$ 
of $\calg$
is a subset of $\calg$
($\calh \subset \calg$)
which is also a group.
It's easy to show that any $\calh\subset \calg$ is a group if it
contains the identity
and is {\bf closed 
under multiplication} (i.e., $ab\in \calh$ for all $a,b\in \calh$) 

\section{Group  Representation}

A {\bf group representation}
of a group $\calg$
is a map $\phi: \calg\rarrow \CC^{n\times n}$\footnote{More generally, the $\CC^{n\times n}$ can be replaced by $\RR^{n\times n}$ or by $\FF^{n\times n}$ for any field $\FF$} such that

\beq
\phi(a)\phi(b)=
\phi(ab)
\eeq
Such a map is called a {\bf homomorphism}.
When a group is 
defined using matrices, those
matrices are called the {\bf defining representation}.
The map $\phi$ 
partitions $\calg$
into disjoints subsets (equivalence classes),
such that all elements of $\calg$ in a disjoint set 
are represented by the same matrix.


For example,
the group
of {\bf General Linear Transformations}
is defined by

\beq
GL(n, \CC)=
\{M\in \CC^{n\times n}: \det{M}\neq 0\}
\eeq

\section{Invariance}

\section{Spectral Decomposition and Eigenvalue Projection Operators}
\label{ch-spectral-decom}

$M\in \CC^{d\times d }$

\beq
M\ket{v}=\lam \ket{v}
\eeq
If $M$ is Hermitian ($H^\dagger=H$), its eigenvalues are real. ( $\lam =
\av{\lam|M\lam}\in\RR$)


\beq
cp(\lam)\eqdef \det(M-\lam)=0
\eeq

If $M$ is a Hermitain  matrix, then there exists
a unitary matric ($CC^\dagger = C^\dagger C =1$)
such that

\beq
CMC^\dagger=
\left[
\begin{array}{cccc}
D_{\lam_1}
&0
&0
&0
\\
0
&D_{\lam_2}
&0
&0
\\
0
&0
&\ddots
&0
\\
0
&0
&0
&D_{\lam_r}
\end{array}
\right]
\eeq
where

\beq
D_{\lam_i} =
\text{diag}\underbrace{(\lam_i,\lam_i, \dots,\lam_i)}_{d_i\text{ times}}
\eeq

\beq
d=\sum_{i=1}^r d_i
\eeq


\beq
CMC^\dagger =
\left[
\begin{array}{cc}
\lam_1 &0
\\
0&\lam_2
\end{array}
\right]
\eeq

\beq
C P_1 C^\dagger=
\left[
\begin{array}{cc}
1&0
\\
0&0
\end{array}
\right]
=
\frac{CMC^\dagger -\lam_2}{\lam_1-\lam_2}
\eeq

\beq
CP_2 C^\dagger =
\left[
\begin{array}{cc}
0&0
\\
0&1
\end{array}
\right]
=
\frac{CMC^\dagger-\lam_1}{\lam_2-\lam_1}
\eeq

If $I^{d_i\times d_i}$
is the $d_i$
dimensional unit matrix,
\beqa
P_i &=&
C^\dagger
diag(0,\ldots,0, I^{d_i\times d_i},0, \dots, 0)C
\\
&=&
\prod_{j\neq i}
\frac{M -\lam_j}{\lam_i -\lam_j}
\eeqa

Note that $P_i$ are Hermitian
($P_i^\dagger = P_i$)
because $M$
is Hermitian and
its eigenvalues are real.)

Note that
$P_i$ and $M$
commute

\beq
[P_i, M]=
P_iM-MP_i=0
\eeq

orthogonal
\beq
P_i P_j =\delta(i,j)P_j
\eeq

complete
\beq
\sum_i P_i =1
\eeq

\beq
M= \sum_{i=1}^r
P_iM P_i
\eeq

\beq
d_i = \tr P_i
\eeq

\beqa
CMP_1C^\dagger &=&
\left[
\begin{array}{cc}
\lam_1&0
\\
0&\lam_2
\end{array}
\right] 
\left[
\begin{array}{cc}
1&0
\\
0&0
\end{array}
\right] 
\\
&=&
\lam_1
\left[
\begin{array}{cc}
1&0
\\
0&0
\end{array}
\right] 
\eeqa

\beq
MP_i = \lam_i P_i \;
\text{(no $i$ sum)}
\eeq

\beq
f(M) P_i = f(\lam_i)P_i \;
\text{(no $i$ sum)}
\eeq

$M^{(1)}, M^{(2)}$

\beq
[M^{(1)}, M^{(2)}]  =0
\eeq
Use $M^{(1)}$ to decompose $V$
into $\bigoplus_i V_i$.
Use  $M^{(2)}$ to decompose $V_i$ into
$\bigoplus_j V_{i,j}$. 
If $M^{(1)}$ and $M^{(2)}$ don't
commute, let $P^{(1)}_i$ be the eigenvalue 
projection operators of $M^{(1)}$. The replace $M^{(2)}$ by $P^{(1)}_i M^{(2)}P_i^{(1)}$

\beq
[M^{(1)}, P^{(1)}_iM^{(2)}P^{(1)}_i]  =0
\eeq



