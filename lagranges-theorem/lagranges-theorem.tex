\chapter{Lagrange's Theorem and Stabilizer-Orbit Theorem: COMING SOON}
\label{ch-lagranges-theorem}


\section{Lagrange's Theorem}

Let $H$ be a subgroup of a finite
group $G$ ($H\le G$)



$g H = \{gh| h\in H\}$ Left coset with coset representative $g$

$G:H=$ set of coset representatives

$G/H=\{gH| g\in G:H\}$


Form the Lagrange table with
rows labelled by $g\in G:H$ and columns
labelled by $h\in H$. Table entries equal to $gh$

\begin{claim}Every element of $G$ appears exactly once in the Lagrange Table. Therefore

\beq
|G:H| \; |H| = |G|
\eeq
\end{claim}
\proof

Given $x\in G$, calculate \beq
A_x=\{g=xh^{-1}|
h\in H\}\eeq
One of the elements
$g\in A_x$ must be in $G:H$. Hence $x=gh\in gH$.  So every $x\in G$ appears at least one coset.

Suppose $x$ appears in 2 cosets $g_1H$ and $g_2H$.
\beq
 g_1h_1=g_2h_2\eeq
so
\beq
g_2^{-1} g_1= h_2 h_1^{-1}\in H
\eeq
Hence, 
\beq
g_2^{-1} g_1 H = H\implies g_2H = g_1H
\eeq
\qed


* or connect this directly to quotient spaces in Lie groups.



\hrule
Example 1




$
G=S_3 = \{e,\ (12),\ (13),\ (23),\ (123),\ (132)\}
$ The symmetric group on 3 letters,
$|G|=6$

$H = \{e,(12)\}$, $|H|=2$

Cosets

\beq\left\{
\begin{array}{ll}
eH &= \{e,(12)\}
\\
(13)H &= \{(13),(123)\}
\\
(23)H &= \{(23),(132)\}
\end{array}
\right.
\eeq

$G:H =
\{e,\ (13),\ (23)\}$ coset representatives. $|G:H|=3$



$G$ gives all operations.

$H=$ symmetries for a given reference state

$G:H=$ all distinct reference states.

$H$ is a normal subgroup of $G$ ($H\) iff

In this example, $G/H$ is not a group. 
In the next example, it is.  $G/H$
is a group iff $H$ is {\bf normal group}
\hrule

 1. Is (H={e,(12)}) normal in (S_3)?

No.
(H) is not a normal subgroup of (S_3).

 Quick check

Conjugate ((12)) by another element of (S_3), say ((13)):
[
(13)(12)(13)^{-1} = (23) \notin H.
]

So
[
gHg^{-1} \neq H
\quad\text{for some } g\in S_3,
]
hence (H \ntrianglelefteq S_3).

---

 2. Simple example where (H) *is* normal

 Example: cyclic group (\mathbb Z_6)

Let
[
G=\mathbb Z_6={0,1,2,3,4,5}
]
(addition mod 6).

Take
[
H={0,3}.
]

* (H) is a subgroup (order 2)
* (G) is abelian ⇒ every subgroup is normal

---

 3. Lagrange table for the normal case

Choose coset representatives:
[
{0,1,2}.
]

Build the table (g+h).

| (g \backslash h) | 0 | 3 |
| ---------------- | - | - |
| 0                | 0 | 3 |
| 1                | 1 | 4 |
| 2                | 2 | 5 |

Every element of (\mathbb Z_6) appears exactly once.

---

 4. Why normality matters (but not for Lagrange)

* Lagrange’s theorem only needs (H \le G)
* Normality is needed to define a group structure on (G/H)

In this example:
[
\mathbb Z_6 / {0,3} \cong \mathbb Z_3.
]

In the (S_3) example, (G/H) is just a set of cosets — not a group.

---

 5. Another classic normal example (non-abelian)

If you want a non-abelian group with a normal subgroup:

 Example: (S_3) again

Let
[
H = A_3 = {e,(123),(132)}.
]

* (|H|=3)
* (H \trianglelefteq S_3) (index 2 ⇒ automatically normal)

Cosets:
[
A_3,\quad (12)A_3.
]

Then
[
S_3 / A_3 \cong \mathbb Z_2.
]

---

 6. One-sentence summary

* The earlier (H={e,(12)}) was not normal
* A simple normal example is ({0,3} \le \mathbb Z_6)
* Normality affects quotients, not Lagrange’s counting

If you want, I can next:

* put the normal vs non-normal cases side-by-side as tables,
* show how coset multiplication fails when (H) isn’t normal,
* or connect this back to stabilizer normalizers in QEC.
