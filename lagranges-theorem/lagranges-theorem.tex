\chapter{Lagrange's Theorem and Orbit-Stabilizer Theorem: COMING SOON}
\label{ch-lagranges-theorem}


\section{Lagrange's Theorem}

Let $H$ be a subgroup of a finite
group $G$ ($H\le G$)



$g H = \{gh| h\in H\}$ Left coset with coset representative $g$

$G:H=$ set of coset representatives

$G/H=\{gH| g\in G:H\}$


Form the Lagrange table with
rows labelled by $g\in G:H$ and columns
labelled by $h\in H$. Table entries equal to $gh$

\begin{claim}Every element of $G$ appears exactly once in the Lagrange Table. Therefore\footnote{Eq.(\ref{eq-finite-lagrange}) is for finite groups. For infinite Lie groups,
the order of each group
is replace by the dimension of the Lie algebra vector space: $dim(\ger{g})=dim(\ln G:H) \; dim (\ger{h})$}

\beq
|G|=|G:H| \; |H|
\label{eq-finite-lagrange}
\eeq
\end{claim}
\proof

Given $x\in G$, calculate \beq
A_x=\{g=xh^{-1}|
h\in H\}\eeq
One of the elements
$g\in A_x$ must be in $G:H$. Hence $x=gh\in gH$.  So every $x\in G$ appears at least one coset.

Suppose $x$ appears in 2 cosets $g_1H$ and $g_2H$.
\beq
 g_1h_1=g_2h_2\eeq
so
\beq
g_2^{-1} g_1= h_2 h_1^{-1}\in H
\eeq
Hence, 
\beq
g_2^{-1} g_1 H = H\implies g_2H = g_1H
\eeq
\qed

A group $G$ is {\bf abelian} if $g_1 g_2=g_2 g_1$ for all $g_1,g_2\in  G$. 

$H$ is  {\bf normal}
subgroup of $G$ (i.e., $H\trianglelefteq G$)
iff $gH=Hg$ (i.e., $gGg^{-1}=H$)
for all $g\in G$.
All subgroups  $H$ of an abelian group $G$  are
normal subgroups.

$G$ gives all operations.

$H=$ symmetries for a given reference state

$G:H=$ all distinct reference states,
often called the {\bf index set}

$G/H=$ set of left cosets, often called a
{\bf quotient space or coset space}. Also called {\bf quotient group or coset group} if $H$ is a normal
subgroup



Examples:
\begin{enumerate}
\hrule
\item




$
G=S_3 = \{e,\ (12),\ (13),\ (23),\ (123),\ (132)\}
$ The symmetric group on 3 letters,


$H = \{e,(12)\}$

Cosets

\beq\left\{
\begin{array}{ll}
eH &= \{e,(12)\}
\\
(13)H &= \{(13),(123)\}
\\
(23)H &= \{(23),(132)\}
\end{array}
\right.
\eeq

$G:H =
\{e,\ (13),\ (23)\}$ coset representatives. 

\beq
|G|= 6, \quad |G:H|=3, \quad |H|=2
\eeq




$H$ is a normal subgroup of $G$ ($H\) iff

In this example, $H$ is not a normal subgroup of $G$
because 
\beq
(13)H = \{(13),\underbrace{(13)(12)}
_{(123)}\}
\eeq

and

\beq
H(13)
=\{(13),\underbrace{(12)(13)}
_{(213)}\}
\eeq
are not equal.

\hrule
\item


$G= Z_6=\{0,1,2,3,4,5\}$
 under addition mod 6.

$H=\{0,3\}$

$G$ is abelian so every subgroup of $G$ is normal

$G:H=\{0,1,2\}$, coset representatives

Cosets

\beq
\left\{
\begin{array}{ll}
0+ H=\{0,3\}
\\
1+ H=\{1,4\}
\\
2+ H=\{2,5\}
\end{array}
\right.
\eeq

\hrule
\item ($H$ is normal subgroup even though $G$ is not abelian)

$S_3^e = \{e,(123),(132)\}$, even number of cycles

$S_3^o=\{(12),(13),(23)\}$, odd number of cycles

$G=S_3=S_3^o \cup S_3^e$



$
H = S_3^e.
$



$S_3^e \trianglelefteq S_3$ because $\s S_3^e =S_3^e \s =S_3^o$
for all $\s \in S_3$

$G:H=\{e, (12)\}$ coset representatives


Cosets

\beq
\left\{
\begin{array}{ll}
eH &= S_3^e
\\
(12)H&= S_3^o
\end{array}
\right.
\eeq

\beq
S_3 / S_3^e \cong  Z_2
\eeq

\end{enumerate}

\section{Orbit-Stabilizer Theorem}


Let a group $G$ act on a set $X$ by an action $\odot$. This means that $\odot: G\times X\rarrow X$ and $(g,x)\mapsto g\odot x$. Normally this {\bf action} $\odot$ is {\bf group conjugation}. $g\odot x= g x g^{-1}$. In the special case  of Lagrange's theorem, $X=G$ and $g\odot x=gx$.

$Stab_G(x)
=\{ g \in G | g x g^{-1} = x \}
$= {\bf Stabilizer of $x$}, the symmetries of $x$, group operations that keep it fixed or \qt{stabilized}.

$O(x)=\{gx g^{-1}| g\in G \}=$ {\bf Orbit of $x$}, all possible projections of $x$ via $g\in G$



\begin{claim}(Orbit-Stabilizer theorem for single point $x\in X$)
\beq
|G| = |O(x)| \;|Stab_G(x)|
\eeq
\end{claim}
\proof
\qed

$Z(G)=\{z\in G| gzg^{-1}=z \text{ for all }g\in G\}=$ {\bf center of $G$}

Suppose $H$ is a subgroup of $G$ (i.e., $H\leq G$) and action is group conjugation.
There are two ways of defining $Stab(x)$ for a subgroup $H$, instead of a 
single point $x$

\beq
\left\{
\begin{array}{ll}
\cup_{h\in H}Stab_G(x)=
\cup_{h\in H}
\{g\in G| ghg^{-1}=h\}
=C_G(H)& \text{\bf Centralizer}
\\
Stab_G(H)=\{g\in G| gHg^{-1}=H\})=N_G(H)& \text{\bf Normalizer}
\end{array}
\right.
\eeq

 $N_G(H)$ is the biggest subset of $G$ that leaves the subgroup $H$ fixed under conjugation.
 $N_G(H)$ is always  a subgroup of $G$.
 If $H$ is a normal subgroup of $G$, 
then $N_G(H)=G$.

$C_G(H)$ is a subgroup of $N_G(H)$

 \beq
\begin{pmatrix}Z(G)
\\
H
\end{pmatrix}\leq C_G(H)\leq N_G(H)\leq G
\eeq

$O(H)=\cup_{h\in H}O(x)=$ projections (orbits) of 
structure(subgroup) $H$ instead of single point $H$


\begin{claim}(Orbit-Stabilizer theorem for subgroup $H\leq G$)
\beq
|G| = |O(H)| \;|\underbrace{Stab_G(H)}_{N_G(H)}|
\eeq
\end{claim}
\proof
\qed





Examples:
\begin{enumerate}
\hrule
\item 
$G= S_3$


$x=(12)$

$Stab_G(x)=\{e, (12)\}$

$O(x)= \{ (12), (13)(12)(13), (23)(12)(23)\} =
\{
(12), (23), (13)
\}$

\beq
|G|=6, |O(x)|=3, |Stab(x)|=2\implies
|G| = |O(x)| \;|Stab_G(x)|
\eeq
\hrule
\end{enumerate}

\begin{itemize}
\item error correction intuition

Pauli group =$\{1, \s_x, \s_y, \s_z\}\times\{\pm 1, \pm i\}$. Has 16 elements.

Clifford group
=group generated by $H=\frac{1}{\sqrt{2}}=\begin{pmatrix}1&1\\1&-1\end{pmatrix}$
and $S=\begin{pmatrix}1&0\\0&i
\end{pmatrix}$. Has 24 elements

$G=$ Pauli or Clifford group

$x$= code state

$O(x)$= projections of code state $x$

$Stab(x)=$ subgroup of $G$ that preserves (doesn't change, stabilizes, fixes) code state $x$

$H$ = subgroup of $G$

$Stab(H)=$ subgroup of $G$ that fixes $H$

\item thermodynamic intuition

$G$= group

$x$= single microstate


$O(x)$= set of equivalent microstates=single macrostate

$\ln |O(x)|$= entropy

$Stab(x)=$ subgroup of $G$ that preserves (doesn't change, stabilizes, fixes) microstate $x$, 
residual symmetry group



\item Lie groups intuition

$G$= Lie group

$H$= stabilizer, principal fibre

$G:H=$ base space


$\pi(g)=gH=$ projection, fibre

$\pi: G\rarrow G/H=\{gH: g\in G\}$ fibre bundle

Example

$G=SU(2)$

$H=U(1)$ fibre

$G:H=S^2$= two sphere, base space




\end{itemize}
