\chapter{Young Tableau}
\label{ch-young-tableau}

This chapter is based on Cvitanovic's Birdtracks book Ref. \cite{birdtracks-book}.

We recommend that the read Chapter \ref{ch-sym}
on symmetrizers and antisymetrizers
before reading this one.

A {\bf Young Diagram} (YD) $\caly=[\lam_1, \lam_2, \ldots, \lam_D]$ 
consists of $\lam_1$ left-aligned empty boxes (LAEB)
over $\lam_2$ LAEB, over 
$\lam_3$ LAEB, up to
$\lam_{NR}$ LAEB,
where $\lam_1\geq \lam_2\geq \ldots\geq \lam_{NR}\geq 1$.
$NR=$ number of rows.
For example,

\beq
\ydiagram{4,2,1,1} =
[4,2,1,1]
\eeq
We will call $[4, 2,1,1]$ the
{\bf row lengths} (RL) method 
of labeling YD.

A alternative method of labelling YD is called the {\bf Dynkin (D) labels} 
or {\bf row changes (RC)}.
These labels list the change in number of columns as 
we go down the YD.
For example,

\beq
\ydiagram{4,2,1,1}
=
[2,1,0,1, 0\ldots]_{D}
\eeq



A {\bf Young Tableau} (YT)
$\caly_\alp$ is a YD in which integers from 1 to $n$ where $n\leq n_b$ and $n_b$ is the number boxes, are inserted according to some rules. The rules for insertion are that integers must increase when reading a row  left to right and
when reading a column from top to bottom. Obviously, for $n<n_b$, some integers are repeated.

A {\bf Standard Young Tableau} (SYT)
$\caly_\alp$  is a YT such that $n=n_b$ and 
no integer is repeated. 
Fig.\ref{fig-syt-1234}
shows all SYT for $n_b=1,2,3,4$ 
\begin{figure}[h!]
\begin{itemize}

\item $n_b=1$

\begin{ytableau}
1
\end{ytableau}

\item $n_b=2$

\begin{ytableau}
1 & 2
\end{ytableau}
\quad \begin{ytableau}
1 \\
2
\end{ytableau}

\item  $n_b=3$

\begin{ytableau}
1 & 2 &3
\end{ytableau}
\quad\begin{ytableau}
1 & 2 \\
3
\end{ytableau}
\quad\begin{ytableau}
1 & 3 \\
2
\end{ytableau}
\quad\begin{ytableau}
1 \\ 2 \\
3
\end{ytableau}

\item $n_b=4$

\quad\begin{ytableau}
1 & 2 & 3 & 4
\end{ytableau}
\quad\begin{ytableau}
1 & 2 & 3 \\ 4
\end{ytableau}
\quad\begin{ytableau}
1 & 2 & 4 \\ 3
\end{ytableau}
\quad\begin{ytableau}
1 & 3& 4\\2
\end{ytableau}
\quad\begin{ytableau}
1 & 2 \\3 & 4
\end{ytableau}

\quad\begin{ytableau}
1 & 3\\ 2 & 4
\end{ytableau}
\quad\begin{ytableau}
1 & 2 \\ 3 \\ 4
\end{ytableau}
\quad\begin{ytableau}
1 & 3 \\ 2 \\ 4
\end{ytableau}
\quad\begin{ytableau}
1 & 4 \\ 2\\ 3
\end{ytableau}
\quad\begin{ytableau}
1 \\ 2 \\ 3 \\ 4
\end{ytableau}
\end{itemize}
\caption{All SYT for $n_b=1, 2,3,4$.}
\label{fig-syt-1234}
\end{figure}

We will use
$|\caly|$, or $|\caly_\alp|$
or $|\alp|$ to denote the number
of boxes in a YD or YT.\footnote
{For many authors and for us too,  $|S|$
stands for the number of elements
in a finite set $S$.
This should not
lead to confusion
as a YD or YT are not sets.}

\section{Symmetric group $S_{n_b}$}

Let

$S_{n_b}=$ the symmetric group in $n_b$ letters (or $n_b$ boxes)

$irreps(S_{n_b})=$
the set of all
irreps of $S_{n_b}$.


The {\bf transpose of a YT} is defined as if it were a matrix. For example
\beq
transpose\left(
\bcen
\begin{ytableau}
1&3& 5
\\
2&6
\\
4
\end{ytableau}
\ecen
\right)
=
\bcen
\begin{ytableau}
1&2&4
\\
3&6
\\
5
\end{ytableau}\ecen
\eeq

{\bf n-dim General   Linear group} $GL(n)=\{ M\in\CC^{n\times n}:
det(M)\neq 0\}$

{\bf n-dim Special Linear group} $SL(n)=\{ M\in GL(n):
det(M)=1\}$

{\bf n-dim Unitary group}, $U(n)=\{ M\in GL(n):
M M^\dagger =M^\dagger M =1\}$

{\bf n-dim Special Unitary group}
$SU(n)=\{ M\in U(n):
det(M)=1\}$

$YD(n_b)=$ set of YD with $n_b$ boxes.
$YD=\cup_{n_n=1}^\infty YD(n_b)$.

$SYD(n_b)=$ set of SYD with $n_b$ boxes. $YT=\cup_{n_b=1}^\infty YT(n_b)$.

$SYT(n_b, NR)=$ set of STY with $n_b$
boxes and $NR$ rows.

$YT(\caly)=$ set of YT with a YD $\caly$.

$SYT(\caly)=$ set of SYT with a YD $\caly$.


$dim(\caly| S_{n_b})=$ dimension of irrep $\caly$ of $S_{n_b}$

$dim(\caly_\alp| U(n))=$ dimension of irrep $\caly_\alp$ of $U(n)$ or $SU(n)$.




\begin{claim}\
\begin{enumerate}
\item
The YD with $n_b$ boxes label all irreps of the symmetric group 
$S_{n_b}$.

\beq
irreps(S_{n_b})=YD(n_b)
\eeq

\item
The SYT with $n_b$ boxes and no more than $n$ rows
($NR\leq n$),
label the irreps of $GL(n)$ and of $U(n)$

\beq
irreps(U(n)) =\bigcup_{n_b\leq n, \;NR\leq n}STY(n_b, NR) 
\eeq

\item
The SYT with $n_b$ boxes and
no more than $n-1$
rows ($NR\leq n -1$), label the irreps of $SL(n)$ and $SU(n)$.

\beq
irreps(SU(n)) =
\bigcup_{n_b\leq n,\; NR\leq n-1}STY(n_b, NR) 
\eeq
\end{enumerate}
\end{claim}
\proof
\qed










\subsection{$dim({\cal Y}|S_{n_b})$}

\begin{claim}

\beq
dim(\caly|S_{n_b})=|SYT(\caly)|
\eeq
\end{claim}
\proof
\qed

For example, there are 3
irreps of $S_4$ 
associated withe the YD

\beq
\caly=
\ydiagram{2,1,1}
\eeq
And each of those 3 irreps has dimension 3. Why? Because there are 3 possible SYT for this YD:

\beq
\begin{ytableau}
1&2
\\
3
\\
4
\end{ytableau}
,\quad
\begin{ytableau}
1&3
\\
2
\\
4
\end{ytableau}
,\quad
\begin{ytableau}
1&4
\\
2
\\
3
\end{ytableau}
\implies dim(\caly| S_4)=3
\eeq
Thus, we can denote the
basis vectors of one of
these 3 degenerate irreps by

\beq
\ket{\bcen\begin{ytableau}
1&2
\\
3
\\
4
\end{ytableau}\ecen}
,\quad
\ket{\bcen\begin{ytableau}
1&3
\\
2
\\
4
\end{ytableau}\ecen}
,\quad
\ket{\bcen\begin{ytableau}
1&4
\\
2
\\
3
\end{ytableau}\ecen}
\eeq

To compute $hook(\caly)$:

\begin{enumerate}
\item Fill each box of the YD with the number of boxes below and to the right of the box, including the
box itself. 
\item Multiply
the numbers in
all the boxes.
\end{enumerate}
For example,

\beq
\caly=
\bcen
\ydiagram{4,3,2}
\ecen
\implies
hook(\caly)=
\bcen
\begin{ytableau}
6&5&3&1
\\
4&3&1
\\
2&1
\end{ytableau}
\ecen
=6! 3
\eeq


\begin{claim} ({\bf hook rule} for computing $dim(\caly|S_{n_b})$)
\beq
dim(\caly| S_{n_b})=
\frac{n_b!}{hook(\caly)}
\eeq
\end{claim}
\proof
\qed

For example

\beq
\caly=\ydiagram{2,1,1}
\implies
hook(\caly)=
\begin{ytableau}
4&1
\\
2
\\
1
\end{ytableau}=
8
\eeq
so
\beq
dim(\caly| S_4)= \frac{4!}{4(2)}=3
\eeq

\subsection{Regular Representation}
The {\bf regular representation} of the
symmetric group $S_{n_b}$ is defined as follows.
For each permutation $\s \in S_{n_b}$, define
an independent vector $\ket{\s}$
in a vector space $\calv=\{\ket{\s}: \s \in S_{n_b}\}=\{\ket{\s_i} : i=1, 2, \ldots , n_b!\}$. Let 


\beq
\ket{x}= \sum_i x_i
\ket{\s_i}
\eeq
For any $\tau\in S_{n_b}$, suppose

\beq
\bra{\s_j}\tau\ket{\s_i}=
\av{\s_j \tau|
\s_i}
\eeq


\beq
\bra{\s_j}\tau\ket{x}=
\av{\s_j\tau|
x}=\sum_i x_i \av{\s_j\tau|\s_i}
\eeq


\begin{claim}
The regular
rep is $n_b!$ dimensional
and reducible.
In the decomposition of the regular rep of $S_{n_b}$,
each $\lam\in irreps(S_{n_b})$ 
appears $dim(\lam|S_{n_b})$ times. 
\end{claim}
\proof
\qed

From the last claim, it follows that

\begin{align}
n_b! = |S_{n_b}| &=\sum_{\lam\in irreps(S_{n_b})}[dim(\lam|S_{n_b})]^2
\\
&=[n_b!]^2\sum_{\caly \in YD(n_b)}
\frac{1}{[hook(\caly)]^2}\quad
\text{(Because  $|irreps(S_{n_b})|=|YD(n_b)|$)}
\end{align}
Hence,

\beq
1 = n_b!\sum_{\caly \in YD(n_b)}
\frac{1}{[hook(\caly)]^2}
\label{eq-1-hook-sq}
\eeq


The Clebsch-Gordan series 
for the regular rep of $S_{n_b}$ is

\beq
1 = \sum_{\caly\in YD(n_b)}
P_\caly
\eeq
where each $P_\caly$
can be further decomposed into 

\beq
P_\caly = \sum_{\caly_\alp \in SYT(\caly)} \underbrace{\ket{\caly_\alp}\bra{\caly_\alp}}_{P_{\caly_\alp}}
\eeq
The projection operators 
\beq
\{P_{\caly_\alp}
: \caly_\alp\in STY(\caly), \caly\in YD(n_b)\}=
\{P_{\caly_\alp}: \caly_\alp \in SYT(n_b)\}
\eeq
are complete and orthogonal.

\subsection{Tensor product decompositions}

\beq
\ydiagram{1}\otimes \ydiagram{1}=
\ydiagram{2} \oplus \ydiagram{1,1}
\eeq


\beq
\begin{array}{l}
\ydiagram{3}
\otimes\ydiagram{2,1}=
\\
\\
\ydiagram{5,1}
\oplus
\ydiagram{4,2}
\oplus
\ydiagram{4,1,1}
\oplus
\ydiagram{3,2,1}
\end{array}
\eeq


\section{Unitary group $U(n)$}

Let 

$STY(n_b, NR<n')=$ set of STY with
$n_b$ boxes  and number of rows $NR<n'$

Recall that\footnote{Note that
$STY(n_b) $ only
contains STY with $n_b\leq n$ boxes, so the $n_b\leq n$
constraint might seem redundant in Eqs.(\ref{eq-irreps-of-un-sun}).
It isn't redundant because
by $\cup_{n_b\leq n}$
we mean $\cup_{n_b=1}^n$.}

\begin{subequations}
\label{eq-irreps-of-un-sun}
\beq
irreps(U(n)) =
\bigcup_{n_b\leq n,\; NR\leq n}STY(n_b, NR) 
=
\bigcup_{n_b=1}^n STY(n_b, NR<n)
\eeq

\beq
irreps(SU(n)) =
\bigcup_{n_b\leq n,\; NR\leq n-1}STY(n_b, NR) 
=
\bigcup_{n_b=1}^n STY(n_b, NR<n-1)
\eeq
\end{subequations}

A SYT with $n_b$ boxes represents a 
tensor with $n_b$ indices ($n_b$-particles state). Each index ranges from $1$ to $n$.

$n_b=1$: A 1-index, 1-box tensor is a 1-particle
with $n$ states. This corresponds to the
fundamental representation.

$n_b=2$: A 2-index, 2-box tensor is a 2-particle
with $n^2$ states. These $n^2$ states 
break into two sets, symmetric and anti-symmetric. 

\beqa
\bcen\begin{ytableau}1
\end{ytableau}
\ecen\otimes 
\bcen\begin{ytableau}1
\end{ytableau}
\ecen &=&
\begin{ytableau}1\\2
\end{ytableau} \quad\oplus\quad \begin{ytableau}1&2
\end{ytableau}
\\
\xymatrix@R=1pc@C=1pc{
&&\ar[ll]
\\
&&\ar[ll]
}
&=&
\bcen\xymatrix@R=1pc@C=1pc{
&\ar[l]\ar@2{-}[d]\cala_2&\ar[l]
\\
&\ar[l]&\ar[l]
}\ecen
+
\bcen\xymatrix@R=1pc@C=1pc{
&\ar[l]\ar@2{-}[d]\cals_2&\ar[l]
\\
&\ar[l]&\ar[l]
}\ecen
\eeqa

The SYT of  an irrep describes
the symmetry of the indices
of a tensor in that irrep.
A single column SYT indicates a
totally
anti-symmetric tensor, a
single row SYT indicates a totally symmetric tensor,
and a SYT with more
than one row or column indicates a mixed symmetry tensor. This
is why we can't have more than $n$ rows,
because there are only $n$ integers
to fill all boxes so more
than $n$ rows would require a  repetition
of an integer in a column, and
such a column, after antisymetrizing, would
lead to zero.


\subsection{Young Projection operators}

For each  SYT $\caly_\alp\in irreps(U(n))$, define
the {\bf Young projection operator}

\beq
P_{\caly\alp}=
\caln
\left(\prod_i S_i\right)
\left(\prod_j A_j
\right)
\eeq
for some normalization
constant $\caln$ yet to
be  determined.
These projection
operators are not
unique.

\begin{claim}

\beq
\caln
=
\frac{
\left(\prod_i |S_{i}|!\right)
\left(\prod_j |A_j|!
\right)
}{hook(\caly)}
\label{eq-n-caly}
\eeq
where $|S_i|$
and $|A_j|$ are
the number of arrows entering 
the symmetrizer or
anti-symmetrizer.
Note that the normalization
constant $\caln$
depends only on the
YD $\caly$. Furthermore, the operators $P_{\caly_\alp}$ are idempotent 
(i.e., their square equals themselves) mutually orthogonal and complete:
\beq
P_{\caly_\alp}
P_{\caly_\beta}=
P_{\caly_\alp}
\delta(\alp, 
\beta)
\eeq
\beq
1 =\sum_{\caly_\alp\in SYT(n_b, \;NR<n')}
P_{\caly_\alp}
\label{eq-complete-un-proj}
\eeq
where
\beq
n'=
\left\{
\begin{array}{ll}
n &\text{for $U(n)$}
\\
n-1 & \text{for $SU(n)$}
\end{array}
\right.
\eeq
\end{claim}
\proof

\beq
P_{\caly_\alp}=
\caln
\frac{1}
{\prod_i |S_i|!
\prod_j |A_j|!}
\left(
\bcen
\underbrace
{\xymatrix@C=1pc@R=1pc{
&&\ar[ll]
\\
&&\ar[ll]
\\
\vdots
\\
&&\ar[ll]
}}_\indi
\ecen
+
\cdots
\right)
\eeq
From Eq.(\ref{eq-complete-un-proj})

\beqa
\indi
&=&
\sum_{\caly_\alp\in SYT(n_b, NR<n')}
\caln
\frac{1}
{\prod_i |S_i|!
\prod_j |A_j|!}
\quad\indi
\\
&=&
\sum_{\caly\in YD(n_b)}
\frac{n_b!}{hook(\caly)}
\caln
\frac{1}
{\prod_i |S_i|!
\prod_j |A_j|!}
\quad \indi
\\
&=&
\sum_{\caly\in YD(n_b)}
\frac{n_b!}{
[hook(\caly)]^2}
\frac{1}
{\prod_i |S_i|!
\prod_j |A_j|!}
\quad \indi
\quad
\text{(if assume Eq.(\ref{eq-n-caly}))}
\\
&=& \indi \quad\text{
(by Eq.(\ref{eq-1-hook-sq}))}
\eeqa
\qed

\subsection{$dim({\cal Y}_\alpha|U(n))$}

Let $dim(\caly_\alp|U(n))$
be the dimension of an irrep 
of $U(n)$
with STY given by $\caly_\alp\in STY(n_b, NR<n)$.
\begin{claim}
\beq
dim(\caly_\alp|U(n))= |YT(\caly)|
\label{eq-dim-yalp}
\eeq
Note that the right hand
side is independent
of $\alp$, so
this dimension
is  the same for
all irreps $\alp$
with the same YD $\caly$.
\end{claim}
\proof
\qed

Hence, $\{\ket{\alp}:  \alp\in YT(\caly)\}$
are a basis for the 
irrep $\caly_\alp$ of $U(n)$.
Note that the irreps of $U(n)$ are given by SYT  $\caly_\alp$,
whereas the basis vectors of an irrep are given by YT (not just STY).
For example,
for

\beq
\caly_\alp=
\begin{ytableau}1&2
\end{ytableau}
\eeq
the basis vectors are



\beq
\ket{\begin{ytableau}1&1
\end{ytableau}}
,\quad
\ket{\begin{ytableau}1&2
\end{ytableau}}
,\quad
\ket{\begin{ytableau}2&2
\end{ytableau}}
\eeq
so
\beq
dim(\caly_\alp|U(2))=
3
\eeq




 In Eq.(\ref{eq-dim-yalp})
we gave a way of finding $dim(\caly_\alp|U(n))$
A second way is by taking the trace of
the corresponding projection operator
\beq
dim(\caly_\alp|U(n))=
\tr(P_{\caly_\alp})\eeq
For example, if

\beq
\caly_\alp=
\begin{ytableau}
1&2
\end{ytableau}
\eeq
then

\beqa
dim(\caly_\alp| U(n))
&=&
\bcen
\xymatrix@R=1pc@C=1pc{
&\ar[l]\ar@2{-}[d]\cals_2
&\ar[l]\ar@{-}@[red]@/_1pc/[ll]
\\ 
&\ar[l]&\ar[l]\ar@{-}@[red]@/_1pc/[ll]
}\ecen
\\
&=&
\frac{1}{2}
\left(
\bcen
\xymatrix@R=1.5pc@C=1pc{
&
&\ar[ll]\ar@{-}@[red]@/_1pc/[ll]
\\ 
&
&\ar[ll]\ar@{-}@[red]@/_1pc/[ll]
}\ecen
+
\bcen
\xymatrix@R=1.5pc@C=1pc{
&\ar[l]\ar@{<->}[d]
&\ar[l]\ar@{-}@[red]@/_1pc/[ll]
\\ 
&\ar[l]&\ar[l]\ar@{-}@[red]@/_1pc/[ll]
}\ecen\right)
\\
&=& \frac{1}{2}(n^2 + n)
\\
&=&3 \text{ for $n=2$}
\eeqa

A third way of computing $dim(\caly_\alp| U(n))$
is by computing the hook and coat functions
and using the formula

\beq
dim(\caly_\alp| U(n))=
\frac{coat(\caly)}{hook(\caly)}
\eeq
Note that right
hand side is independent of $\alp$; it 
depends only on the YD.
We've already discussed how to compute
$hook(\caly)$.
$coat (\caly)$ is
calculated as follows.\footnote{
I invented the name
$coat(\caly)$. I don't know if  it has a name.}

\begin{enumerate}
\item Fill $\caly$
with 
\begin{itemize}
\item $n$ at the diagonal blocks
\item $n$ 
increasing by 1 per block when reading from left to right
\item
$n$ 
decreasing by 1 
per blockk
when reading from
top to bottom
\end{itemize}
\item multiply all
the boxes
\end{enumerate}

Examples

\beq
dim(\begin{ytableau}
1&2
\end{ytableau}, U(2)
)  =
\frac{\begin{ytableau}
$\scriptsize n$& $\scriptsize n+1$
\end{ytableau}
}{
\begin{ytableau}
2&1
\end{ytableau}
}
= \frac{n(n+1)}{2}
\eeq


\beq
dim(\begin{ytableau}
1\\2
\end{ytableau}, U(2)
)  =
\frac{\begin{ytableau}
$\scriptsize n$\\ $\scriptsize n-1$
\end{ytableau}
}{
\begin{ytableau}
2\\1
\end{ytableau}
}
= \frac{n(n-1)}{2}
\eeq

\beq
dim(\begin{ytableau}
1&2&3&4
\\
5&6
\\
7
\end{ytableau}, U(7)
)  =
\frac{\begin{ytableau}
$\scriptsize n$
&$\scriptsize n+1$
&$\scriptsize n+2$
&$\scriptsize n+3$
\\
$\scriptsize n-1$
&$\scriptsize n$
\\
$\scriptsize n-2$
\end{ytableau}
}{
\begin{ytableau}
6&4&2&1
\\
3&1
\\
1
\end{ytableau}
}
= \frac{n^2(n^2-1)(n^2-4)(n+3)}
{144}
\eeq




\subsection{Young Projection operators for $n_b=1,2,3,4$}
\label{sec-yp-holes}

Symmetrizers $\cals_p$ and
antisymmetriers $\cala_p$
are discussed in Chapter \ref{ch-sym}.

In this section,
we will use symmetrizers and
antisymmetrizers with \qt{holes}
A hole, denoted
by an empty square, will
signify a particle
that
the symmetrizer or
antisymmetrizer does not touch.
For example

\beq
\bcen\xymatrix@R=1pc@C=1pc{
&\ar[l]\ar@2{-}[d]\cals_2
&\ar[l]1
\\
&\ar[l]\ar@2{-}[d]\square
&\ar[l]2
\\
&\ar[l]
&\ar[l]3
}\ecen
\eeq
denotes a symmetrizer of the 
particles 1 and 3 but not 2.

Note that
\beq
\myboxed{
(c,a)=(b,c)(b,a)(b,c)
}
\quad
\bcen\xymatrix@R=1pc@C=1pc{
&\ar@{<->}[dd]
&\ar[ll]a
\\
&
&\ar[ll]b
\\
&
&\ar[ll]c
}\ecen
=
\bcen\xymatrix@R=1pc@C=1pc{
&
&\ar@{<->}[d]
&
&\ar[llll]a
\\
&\ar@{<->}[d]
&
&\ar@{<->}[d]
&\ar[llll]b
\\
&
&
&
&\ar[llll]c
}
\ecen
\eeq
Similarly

\beq
\bcen\xymatrix@R=1pc@C=1pc{
&\ar[l]\ar@2{-}[d]\cals_2
&\ar[l]
\\
&\ar[l]\ar@2{-}[d]\square
&\ar[l]
\\
&\ar[l]
&\ar[l]
}\ecen
=
\bcen\xymatrix@R=1pc@C=1pc{
&\ar[l]
&\ar[l]\ar@2{-}[d]\cals_2
&\ar[l]
&\ar[l]
\\
&\ar[l]\ar@{<->}[d]
&\ar[l]
&\ar[l]\ar@{<->}[d]
&\ar[l]
\\
&\ar[l]
&\ar[l]
&\ar[l]
&\ar[l]
}
\ecen
\eeq
Hence, one can
avoid using symmetrizers
and antisymmetrizers with holes,
if one is willing to use swaps
instead of holes.

Below, we use holes,
but
keep in mind that  those holes can we replaced
by swaps.

Below, we give
the Clebsch-Gordan decomposition of 

\beq
\begin{array}{l}
\ydiagram{1}^{\otimes n_b}
\\
(\xymatrix{&\ar[l]})^{\otimes n_b}
\end{array}
\eeq
for $n_b=1,2,3,4$.



\newcommand{\YTone}[0]{
$
\bcen
\text{
\begin{ytableau}
1
\end{ytableau}}
\\
\xymatrix{
&\ar[l]
}
\ecen
$
}

\newcommand{\YTs}[0]{
$
\bcen
\text{
\begin{ytableau}
1 & 2
\end{ytableau}
}
\\
\xymatrix@R=1pc@C=1pc{
&\ar[l]\ar@2{-}[d]\cals_2&\ar[l]
\\
&\ar[l]&\ar[l]
}\ecen
$
}
\newcommand{\YTss}[0]{
$
\bcen
\text{
\begin{ytableau}
1 & 2 &3
\end{ytableau}
}
\\
\xymatrix@R=1pc@C=1pc{
&\ar[l]\ar@2{-}[dd]\cals_3&\ar[l]
\\
&\ar[l]&\ar[l]
\\
&\ar[l]&\ar[l]
}\ecen
$
}

\newcommand{\YTsss}[0]{
$
\bcen
\text{
\begin{ytableau}
1 & 2 &3 &4
\end{ytableau}
}
\\
\xymatrix@R=1pc@C=1pc{
&\ar[l]\ar@2{-}[ddd]\cals_4&\ar[l]
\\
&\ar[l]&\ar[l]
\\
&\ar[l]&\ar[l]
\\
&\ar[l]&\ar[l]
}\ecen
$
}


\newcommand{\YTa}[0]{
$\bcen
\begin{ytableau}
1 \\2
\end{ytableau}
\\
\xymatrix@R=1pc@C=1pc{
&\ar[l]\ar@2{-}[d]\cala_2&\ar[l]
\\
&\ar[l]&\ar[l]
}
\ecen $
}
\newcommand{\YTaa}[0]{
$\bcen
\begin{ytableau}
1 \\2\\3
\end{ytableau}
\\
\xymatrix@R=1pc@C=1pc{
&\ar[l]\ar@2{-}[dd]\cala_3&\ar[l]
\\
&\ar[l]&\ar[l]
\\
&\ar[l]&\ar[l]
}
\ecen $
}

\newcommand{\YTaaa}[0]{
$\bcen
\begin{ytableau}
1 \\2\\3\\4
\end{ytableau}
\\
\xymatrix@R=1pc@C=1pc{
&\ar[l]\ar@2{-}[ddd]\cala_4&\ar[l]
\\
&\ar[l]&\ar[l]
\\
&\ar[l]&\ar[l]
\\
&\ar[l]&\ar[l]
}
\ecen $
}

\newcommand{\YTsa}[0]{
$
\bcen
\text{
\begin{ytableau}
1 & 2\\3
\end{ytableau}
}
\\
\xymatrix@R=1pc@C=1pc{
&\ar[l]\ar@2{-}[d]\cals_2
&\ar[l]\ar@2{-}[d]\cala_2
&\ar[l]
\\ 4/3
&\ar[l]&\ar[l]\ar@2{-}[d] \square&\ar[l]
\\
&\ar[l]&\ar[l]&\ar[l]
}\ecen
$
}
\newcommand{\YTas}[0]{
$
\bcen
\text{
\begin{ytableau}
1 & 3\\2
\end{ytableau}
}
\\
\xymatrix@R=1pc@C=1pc{
&\ar[l]\ar@2{-}[d]\cals_2
&\ar[l]\ar@2{-}[d]\cala_2
&\ar[l]
\\ 4/3
&\ar[l]\ar@2{-}[d] \square
&\ar[l]
&\ar[l]
\\
&\ar[l]&\ar[l]&\ar[l]
}\ecen
$
}

\newcommand{\YTasFour}[0]{
$
\bcen
\text{
\begin{ytableau}
1 & 3&4\\2
\end{ytableau}
}
\\
\xymatrix@R=1pc@C=1pc{
&\ar[l]\ar@2{-}[d]\cals_3
&\ar[l]\ar@2{-}[d]\cala_2
&\ar[l]
\\ 3/2
&\ar[l]\ar@2{-}[dd] \square
&\ar[l]
&\ar[l]
\\
&\ar[l]&\ar[l]&\ar[l]
\\
&\ar[l]&\ar[l]&\ar[l]
}\ecen
$
}

\newcommand{\YTsaFour}[0]{
$
\bcen
\text{
\begin{ytableau}
1 & 2&4\\3
\end{ytableau}
}
\\
\xymatrix@R=1pc@C=1pc{
&\ar[l]\ar@2{-}[dd]\cals_3
&\ar[l]\ar@2{-}[d]\cala_2
&\ar[l]
\\ 3/2
&\ar[l]
&\ar[l]\square \ar@2{-}[d]
&\ar[l]
\\
&\ar[l]\ar@2{-}[d] \square
&\ar[l]
&\ar[l]
\\
&\ar[l]&\ar[l]&\ar[l]
}\ecen
$
}


\newcommand{\YTsFour}[0]{
$
\bcen
\text{
\begin{ytableau}
1 & 2&3\\4
\end{ytableau}
}
\\
\xymatrix@R=1pc@C=1pc{
&\ar[l]\ar@2{-}[dd]\cals_3
&\ar[l]\ar@2{-}[d]\cala_2
&\ar[l]
\\ 3/2
&\ar[l]
&\ar[l]\square \ar@2{-}[d]
&\ar[l]
\\ 
&\ar[l]&\ar[l]\square \ar@2{-}[d]
&\ar[l]
\\
&\ar[l]&\ar[l]&\ar[l]
}\ecen
$
}

\newcommand{\YTsquare}[0]{
$
\bcen
\text{
\begin{ytableau}
1 & 2\\
3&4
\end{ytableau}
}
\\
\xymatrix@R=1pc@C=1pc{
&\ar[l]\ar@2{-}[d]\cals_2
&\ar[l]
&\ar[l]\ar@2{-}[d]\cala_2
&\ar[l]
&\ar[l]
\\ 4/3
&\ar[l]
&\ar[l]
&\ar[l]\square\ar@2{-}[d]
&\ar[l]\ar@2{-}[d]\cala_2
&\ar[l]
\\
&\ar[l]
&\ar[l]\ar@2{-}[d]\cals_2
&\ar[l]
&\ar[l]\ar@2{-}[d]\square
&\ar[l]
\\
&\ar[l]
&\ar[l]
&\ar[l]
&\ar[l]
&\ar[l]
}\ecen
$
}

\newcommand{\YTsquareP}[0]{
$
\bcen
\text{
\begin{ytableau}
1 & 3\\
2&4
\end{ytableau}
}
\\
\xymatrix@R=1pc@C=1pc{
&\ar[l]
&\ar[l]\ar@2{-}[d]\cals_2
&\ar[l]
&\ar[l]\ar@2{-}[d]\cala_2
&\ar[l]
\\ 4/3
&\ar[l]\ar@2{-}[d]\cals_2
&\ar[l]\square\ar@2{-}[d]
&\ar[l]
&\ar[l]
&\ar[l]
\\
&\ar[l]\ar@2{-}[d]\square
&\ar[l]
&\ar[l]\ar@2{-}[d]\cala_2
&\ar[l]
&\ar[l]
\\
&\ar[l]
&\ar[l]
&\ar[l]
&\ar[l]
&\ar[l]
}\ecen
$
}

\newcommand{\YTttwo}[0]{
$
\bcen
\text{
\begin{ytableau}
1 & 2
\\
3
\\
4
\end{ytableau}
}
\\
\xymatrix@R=1pc@C=1pc{
&\ar[l]\ar@2{-}[d]\cals_2
&\ar[l]\ar@2{-}[d]\cala_3
&\ar[l]
\\ 3/2
&\ar[l]
&\ar[l]\square \ar@2{-}[dd]
&\ar[l]
\\
&\ar[l]
&\ar[l]
&\ar[l]
\\
&\ar[l]&\ar[l]&\ar[l]
}\ecen
$
}

\newcommand{\YTtthree}[0]{
$
\bcen
\text{
\begin{ytableau}
1 & 3
\\
2
\\
4
\end{ytableau}
}
\\
\xymatrix@R=1pc@C=1pc{
&\ar[l]\ar@2{-}[d]\cals_2
&\ar[l]\ar@2{-}[dd]\cala_3
&\ar[l]
\\ 3/2
&\ar[l]\square \ar@2{-}[d]
&\ar[l]
&\ar[l]
\\
&\ar[l]
&\ar[l]\square \ar@2{-}[d]
&\ar[l]
\\
&\ar[l]&\ar[l]&\ar[l]
}\ecen
$
}

\newcommand{\YTtfour}[0]{
$
\bcen
\text{
\begin{ytableau}
1 & 4
\\
2
\\
3
\end{ytableau}
}
\\
\xymatrix@R=1pc@C=1pc{
&\ar[l]\ar@2{-}[d]\cals_2
&\ar[l]\ar@2{-}[dd]\cala_3
&\ar[l]
\\ 3/2
&\ar[l]\square \ar@2{-}[d]
&\ar[l]
&\ar[l]
\\
&\ar[l]\square \ar@2{-}[d]
&\ar[l]
&\ar[l]
\\
&\ar[l]&\ar[l]&\ar[l]
}\ecen
$
}



\begin{itemize}

\item $n_b=1$

\YTone

\item $n_b=2$

\YTs
\YTa

\item  $n_b=3$

\YTss

\YTsa
\quad\YTas

\YTaa

\item $n_b=4$

\YTss

\YTsFour
\quad\YTsaFour
\quad\YTasFour

\YTsquare
\quad\YTsquareP

\YTttwo
\quad\YTtthree
\quad\YTtfour

\YTaaa
\end{itemize}


\subsection{Young Projection Operator with swaps}
Eq.(\ref{eq-pyalp-bt})
gives a particular
STY $\caly_\alp$,
and its projector
$P_{\caly_\alp}$.
the projector
is expressed
using
swaps instead of holes.


\beq
\begin{array}{cc}
\begin{array}{l}
\caly_\alp=
\\
\begin{tabular}{llllll}
 & $A_a$ & $A_b$ & $A_c$ & $A_d$ & $A_e$ \\ \cline{2-6} 
\multicolumn{1}{l|}{$S_x$} & \multicolumn{1}{l|}{1} & \multicolumn{1}{l|}{2} & \multicolumn{1}{l|}{3} & \multicolumn{1}{l|}{4} & \multicolumn{1}{l|}{5} \\ \cline{2-6} 
\multicolumn{1}{l|}{$S_y$} & \multicolumn{1}{l|}{6} & \multicolumn{1}{l|}{7} & \multicolumn{1}{l|}{8} & \multicolumn{1}{l|}{9} &  \\ \cline{2-5}
\multicolumn{1}{l|}{$S_z$} & \multicolumn{1}{l|}{10} & \multicolumn{1}{l|}{11} &  &  &  \\ \cline{2-3}
\end{tabular}
\\
\\
\begin{tabular}{|>{$}c<{$}|}
\hline
a1 \xymatrix{&\ar@{<->}[l]} 1 \\ \hline
a2 \xymatrix{&\ar@{<->}[l]} 6 \\ \hline
a3 \xymatrix{&\ar@{<->}[l]} 10 \\ \hline
b1 \xymatrix{&\ar@{<->}[l]} 2 \\ \hline
b2 \xymatrix{&\ar@{<->}[l]} 7 \\ \hline
b3 \xymatrix{&\ar@{<->}[l]} 11 \\ \hline
c1 \xymatrix{&\ar@{<->}[l]} 3 \\ \hline
c2 \xymatrix{&\ar@{<->}[l]} 8 \\ \hline
d1 \xymatrix{&\ar@{<->}[l]} 4 \\ \hline
d2 \xymatrix{&\ar@{<->}[l]} 9 \\ \hline
e1 \xymatrix{&\ar@{<->}[l]} 5 \\ \hline
\end{tabular}

\end{array}
&
P_{\caly_\alp}=
\bcen
\xymatrix@R=1pc@C=.5pc{
\scriptstyle
1 (x1,a1)
&\ar@{-}[l]
S_x \ar@2{-}[dddd]
&&&&&&&&&&\ar@{-}[llllllllll]
A_a\ar@2{-}[dd]
&\ar@{-}[l]
\\ \scriptstyle 2 (x2,a2)
&\ar@{-}[l]
&\ar@{<->}[dddd]
&&&&&&&&&\ar@{-}[llllllllll]
&\ar@{-}[l]
\\\scriptstyle 3 (x3,a3)
&\ar@{-}[l]
&&\ar@{<->}[ddddddd]
&&&&&&&&\ar@{-}[llllllllll]
&\ar@{-}[l]
\\\scriptstyle 4 (x4, b1)
&\ar@{-}[l]
&&&\ar@{<->}[uu]
&&&&&&&\ar@{-}[llllllllll]
A_b\ar@2{-}[dd]
&\ar@{-}[l]
\\\scriptstyle 5 (x5, b2)
&\ar@{-}[l]
&&&&\ar@{<->}[dd]
&&&&&&\ar@{-}[llllllllll]
&\ar@{-}[l]
\\\scriptstyle 6  (y1, b3)
&\ar@{-}[l]
S_y \ar@2{-}[ddd]
&&&&&\ar@{<->}[ddddd]
&&&&&\ar@{-}[llllllllll]
&\ar@{-}[l]
\\\scriptstyle 7 (y2, c1)
&\ar@{-}[l]
&&&&&&\ar@{<->}[uuuu]
&&&&\ar@{-}[llllllllll]
A_c\ar@2{-}[d]&\ar@{-}[l]
\\\scriptstyle 8 (y3,c2)
&\ar@{-}[l]
&&&&&&&&&&\ar@{-}[llllllllll]
&\ar@{-}[l]
\\\scriptstyle 9 (y4,d1)
&\ar@{-}[l]
&&&&&&
&\ar@{<->}[uuuuu]
&&
&\ar@{-}[llllllllll]A_d\ar@2{-}[d]
&\ar@{-}[l]
\\\scriptstyle 10 (z1, d2)
&\ar@{-}[l]
S_z \ar@2{-}[d]
&&&&&&&&\ar@{<->}[u]
&&\ar@{-}[llllllllll] 
&\ar@{-}[l]
\\\scriptstyle 11 (z2, e1)
&\ar@{-}[l]
&&&&&&&&&\ar@{<->}[uuuuuu]&\ar@{-}[llllllllll]
A_e
&\ar@{-}[l]
}
\ecen
\end{array}
\label{eq-pyalp-bt}
\eeq

\subsection{Tensor product decompositions}

\beqa
\begin{ytableau}
1
\end{ytableau}
\otimes
\begin{ytableau}
2
\end{ytableau}
\otimes
\begin{ytableau}
3
\end{ytableau}
&=&
\left(
\begin{ytableau}
1 & 2
\end{ytableau}
\oplus
\begin{ytableau}
1 \\ 2
\end{ytableau}
\right)
\otimes 
\begin{ytableau}
3
\end{ytableau}
\\
&=&
\begin{ytableau}
1 & 2 & 3
\end{ytableau}
\oplus
\begin{ytableau}
1 & 2 \\ 3
\end{ytableau}
\oplus
\begin{ytableau}
1 & 3 \\ 2
\end{ytableau}
\oplus
\begin{ytableau}
1 \\ 2 \\3
\end{ytableau}
\eeqa

\beq
\begin{array}{l}
\bcen\xymatrix@R=1pc@C=1pc{
&&\ar[ll]
\\
&&\ar[ll]
\\
&&\ar[ll]
}\ecen
=
\\ \\
\bcen
\xymatrix@R=1pc@C=1pc{
&\ar[l]\ar@2{-}[dd]\cals_3&\ar[l]
\\
&\ar[l]&\ar[l]
\\
&\ar[l]&\ar[l]
}\ecen
+
\bcen
\xymatrix@R=1pc@C=1pc{
&\ar[l]\ar@2{-}[d]\cals_2
&\ar[l]\ar@2{-}[d]\cala_2
&\ar[l]
\\ 4/3
&\ar[l]&\ar[l]\ar@2{-}[d] \square&\ar[l]
\\
&\ar[l]&\ar[l]&\ar[l]
}\ecen
+
\bcen
\xymatrix@R=1pc@C=1pc{
&\ar[l]\ar@2{-}[d]\cals_2
&\ar[l]\ar@2{-}[d]\cala_2
&\ar[l]
\\ 4/3
&\ar[l]\ar@2{-}[d] \square
&\ar[l]
&\ar[l]
\\
&\ar[l]&\ar[l]&\ar[l]
}\ecen
+
\bcen
\xymatrix@R=1pc@C=1pc{
&\ar[l]\ar@2{-}[dd]\cala_3&\ar[l]
\\
&\ar[l]&\ar[l]
\\
&\ar[l]&\ar[l]
}
\ecen
\end{array}
\eeq


\beq
n^3 =
\frac{n(n+1)(n+2)}{6}
+
\frac{n(n^2-1)}{3}
+
\frac{n(n^2-1)}{3}
=
\frac{(n-2)(n-1)n }
{6}
\eeq

\beq
\begin{ytableau}
\;&\;&\;
\end{ytableau}
\otimes
\begin{ytableau}
\;&\;
\\
\;
\end{ytableau}
=
\left\{
\begin{array}{l}
\begin{ytableau}
\;&\;&\;&\;&\;
\\
\;
\end{ytableau}
\oplus
\begin{ytableau}
\;&\;&\;&\;
\\
\;&\;
\end{ytableau}
\\ \\
\oplus
\begin{ytableau}[*(yellow!40)]
\;&\;&\;&\;
\\
\;
\\
\;
\end{ytableau}
\oplus
\begin{ytableau}[*(yellow!40)]
\;&\;&\;
\\
\;&\;
\\
\;
\end{ytableau}
\end{array}
\right\}
\eeq
For $SU(n)$, the 
yellow YDs are zero for $n=2$,
and non-zero for $n\geq 2$.

\subsection{$SU(n)$}

The elements $G$
of $SU(n)$ satisfy

\beq
\underbrace{\eps_{12\ldots n}}_{1}= 
\underbrace{
G\indices{_{1}^{a_1'}}
G\indices{_{2}^{a_2'}}
\cdots
G\indices{_{n}^{a_n'}}\;
\eps_{a_1' a_2' \ldots  a_n'}
}_{\det G}
\eeq

\beq
\eps_{a_1 a_2 \ldots a_n}=
G\indices{_{a_1}^{a_1'}}
G\indices{_{a_2}^{a_2'}}
\cdots
G\indices{_{a_n}^{a_n'}}\;
\eps_{a_1' a_2' \ldots  a_n'}
\eeq
so the Levi-Civita
tensor 
is a primitive
invariant of $SU(n)$
(but not of $U(n)$)

This leads to 2 consequences.
\begin{enumerate}
\item YD for $SU(n)$ has a maximum of $n-1$ rows.

For an example
of this, see Fig.\ref{fig-u4--to-su4}.
The yellow columns in that figure are 
singlets obtained
by fully
contracting
Levi-Civita tensors.
Hence, those yellow columns
can removed.

\item Conjugate YD 

Given a YD $\caly$,
its {\bf conjugate YD} $conj(\caly)$ is 
obtained as follows:
\begin{itemize}
\item add yellow colored boxes 
to the original YD
so that the resulting YD 
is rectangular and has $n$ rows 
for each column.
\item keep
only
the yellow boxes,
and rotate those
clockwise by 180 degrees.
\end{itemize}
See Fig.\ref{fig-conj-construct}
for an example
of constructing a 
conjugate YD.

This is possible because 
in the intermediate rectangular YD, the 
columns with $n$ white and yellow boxes
represent a fully contracted Levi-Civita
tensor.

\begin{claim}
The reps
corresponding to
YDs $\caly$
and $conj(\caly)$
have the same dimension.
\end{claim}
\proof
\qed

\end{enumerate}



\begin{figure}[h!]
\begin{ytableau}
*(yellow!40)\;&*(yellow!40)\;&\;&\;&\;&\;&\;
\\
*(yellow!40)\;&*(yellow!40)\;&\;&\;&\;
\\
*(yellow!40)\;&*(yellow!40)\;&\;&\;
\\
*(yellow!40)\;&*(yellow!40)\;
\end{ytableau}
$\rarrow$
\ydiagram{5,3,2}
\caption{Iluustration
of removal of
columns 4 boxes long when dealing with $SU(4)$. In this case,
the YD in Dynkin notation goes from
$[2,1,2,2,0,\ldots]_{D}$
to
$[2,1,2,0,\ldots]_{D}$}
\label{fig-u4--to-su4}
\end{figure}

\begin{figure}[h!]
\begin{ytableau}
\;&\;&\;
\\
\;&\;
\\
\;
\\
\;
\end{ytableau}
$\quad\rarrow\quad$
\begin{ytableau}
\;&\;&\;
\\
\;&\;&*(yellow!40)\;
\\
\;&*(yellow!40)\;&*(yellow!40)\;
\\
\;&*(yellow!40)\;&*(yellow!40)\;
\\
*(yellow!40)\;&*(yellow!40)\;&*(yellow!40)\;
\\
*(yellow!40)\;&*(yellow!40)\;&*(yellow!40)\;
\end{ytableau}
$\quad\rarrow\quad$
\begin{ytableau}
*(yellow!40)\;&*(yellow!40)\;&*(yellow!40)\;
\\
*(yellow!40)\;&*(yellow!40)\;&*(yellow!40)\;
\\
*(yellow!40)\;&*(yellow!40)\;
\\
*(yellow!40)\;&*(yellow!40)\;
\\
*(yellow!40)\;
\end{ytableau}
\caption{Construction of a conjugate YD for $SU(6)$}
\label{fig-conj-construct}
\end{figure}

Besides the RL (row lengths) and RC/D (row change/Dynkin)
methods discussed previously,
a third method
commonly used 
to label YD 
for 
$SU(n)$
is as follows. Label them
by their dimension, and then add
a  subscript or overline
if there are more than one 
reps with a different YD but
the same dimension.
This method is used mostly
by physicists for $SU(3)$ (The Eightfold  Way).
Note that all SYT
with the same YD have the
same dimension, so
this really labels YD instead
of YT. For example, for $SU(3)$  we have

\beq
\begin{array}{ccc}
\ydiagram{1}=[1,0]_{D}= 3
&\quad\quad&
\ydiagram{2}=[0,1]_{D}= \ol{3}
\\
\\
\ydiagram{2}=[2,0]_{D}= 6
&\quad\quad&
\ydiagram{2,2}=[0,2]_{D}= \ol{6}
\\
\\
\ydiagram{2,1}=[1,1]_{D}= 8
&\quad\quad&
\ydiagram{3,1}=[2,1]_{D}= 15
\end{array}
\eeq

Using this notation, we have for $SU(n)$, 
\beqa
\ydiagram{1} 
\otimes
\left.\bcen\begin{ytableau}
\;
\\
\;
\\
\none [\vdots]
\\
\;
\end{ytableau}\ecen\right\} n-1
\text{ rows}
&=&
1
\oplus
\left.\bcen\begin{ytableau}
\;&\;
\\
\;
\\
\none [\vdots]
\\
\;
\end{ytableau}\ecen\right\} n-1
\text{ rows}
\\
n \otimes \ol{n}
&=&
1 \oplus (n^2-1)
\\
\text{fun rep}
\otimes 
\text{conjugate rep}&=&
\text{singlet rep}
\oplus \text{ adjoint rep}
\eeqa
{\bf Adjoint representation}

\beq
P_{adj}=
\frac{2(n-1)}{n}
\bcen
\xymatrix@R=1pc@C=1pc{
&\ar[l]
\ar@2{-}[d]\cals_{2}
&\ar@2{-}[d]\cala_{n-1}\ar[l]
&\ar[l]
\\
&\ar[l]&\ar[l]\ar@2{-}[ddd]\square&\ar[l]
\\
&&\ar[ll]&\ar[l]
\\
&&\ar[ll]&\ar[l]
\\
&&\ar[ll]&\ar[l]
}
\ecen
\eeq