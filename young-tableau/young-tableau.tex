\chapter{Young Tableau}
\label{ch-young-tableau}

A {\bf Young Diagram} (YD) $\caly=[\lam_1, \lam_2, \ldots, \lam_D]$ 
consists of $\lam_1$ left-aligned empty boxes
over $\lam_2$, over 
$\lam_3$ left-aligned empty boxes, up to
$\lam_D$ boxes,
where $\lam_1\geq \lam_2\geq \ldots\geq \lam_D\geq 1$.
For example,

\beq
[4, 2,1,1]=
\ydiagram{4,2,1,1}
\eeq
We will call $[4, 2,1,1]$ the
{\bf row lengths} (RL) method 
of labeling YD.

A alternative method o labelling YD is called the {\bf Dynkin labels}
or {\bf row changes (RC)}.
These labels list the change in number of columns as 
we go down the YD.
For example,

\beq
\ydiagram{4,2,1,1}
=
[2,1,0,1, 0\ldots]_{RC}
\eeq



A {\bf Young Tableau} (YT)
$\caly_\alp$ is a YD in which integers from 1 to $n$ where $n\leq n_b$ and $n_b$ is the number boxes, are inserted according to some rules. The rules for insertion are that integers must increase when reading a row  left to right and
when reading a column from top to bottom. Obviously, for $n<n_b$, some integers are repeated.

A {\bf Standard Young Tableau} (SYT)
$\caly_\alp$  is a YT such that $n=n_b$ and 
no integer is repeated. 
\begin{figure}
\begin{itemize}

\item $n_b=1$

\begin{ytableau}
1
\end{ytableau}

\item $n_b=2$

\begin{ytableau}
1 & 2
\end{ytableau}
\quad \begin{ytableau}
1 \\
2
\end{ytableau}

\item  $n_b=3$

\begin{ytableau}
1 & 2 &3
\end{ytableau}
\quad\begin{ytableau}
1 & 2 \\
3
\end{ytableau}
\quad\begin{ytableau}
1 & 3 \\
2
\end{ytableau}
\quad\begin{ytableau}
1 \\ 2 \\
3
\end{ytableau}

\item $n_b=4$

\quad\begin{ytableau}
1 & 2 & 3 & 4
\end{ytableau}
\quad\begin{ytableau}
1 & 2 & 3 \\ 4
\end{ytableau}
\quad\begin{ytableau}
1 & 2 & 4 \\ 3
\end{ytableau}
\quad\begin{ytableau}
1 & 3& 4\\2
\end{ytableau}
\quad\begin{ytableau}
1 & 2 \\3 & 4
\end{ytableau}

\quad\begin{ytableau}
1 & 3\\ 2 & 4
\end{ytableau}
\quad\begin{ytableau}
1 & 2 \\ 3 \\ 4
\end{ytableau}
\quad\begin{ytableau}
1 & 3 \\ 2 \\ 4
\end{ytableau}
\quad\begin{ytableau}
1 & 4 \\ 2\\ 3
\end{ytableau}
\quad\begin{ytableau}
1 \\ 2 \\ 3 \\ 4
\end{ytableau}
\end{itemize}
\caption{SYT for $n_b=1, 2,3,4$.}
\label{fig-syt-1234.}
\end{figure}




\section{Symmetric group $S_{n_b}$}

Let

$S_{n_b}=$ the symmetric group in $n_b$ letters (or $n_b$ boxes)

$irreps(S_{n_b})=$
the set of all
irreps of $S_{n_b}$.


The {\bf transpose of a YT} is defined as if it were a matrix. For example
\beq
transpose\left(
\bcen
\begin{ytableau}
1&3& 5
\\
2&6
\\
4
\end{ytableau}
\ecen
\right)
=
\bcen
\begin{ytableau}
1&2&4
\\
3&6
\\
5
\end{ytableau}\ecen
\eeq

{\bf n-dim General   Linear group} $GL(n)=\{ M\in\CC^{n\times n}:
det(M)\neq 0\}$

{\bf n-dim Special Linear group} $SL(n)=\{ M\in GL(n):
det(M)=1\}$

{\bf n-dim Unitary group}, $U(n)=\{ M\in GL(n):
M M^\dagger =M^\dagger M =1\}$

{\bf n-dim Special Unitary group}
$SU(n)=\{ M\in U(n):
det(M)=1\}$


\begin{claim}\
\begin{enumerate}
\item
The YD with $n_b$ boxes label all irreps of the symmetric group 
$S_{n_b}$.

\item
The SYT with $n_b$ boxes and no more than $n_b$ rows,
label the irreps of $GL(n_b)$ and of $U(n_b)$
\item
The SYT with $n_b$ boxes and
no more than $n_b-1$
rows, label the irreps of $SL(n_b)$ and $SU(n_b)$.
\end{enumerate}
\end{claim}
\proof
\qed



$YD_{n_b}=$ set of YD with $n_b$ boxes

The irreps of $S_{n_b}$  are labelled by the $\caly\in YD_{n_b}$. 
Hence, there is a 1-1 onto map between $irreps(S_{n_b})$ and $YD_{n_b}$


\beq
irreps(S_{n_b})=YD_{n_b}
\eeq

$YT(\caly)=$ set of YT for a $\caly\in YD_{n_b}$.

$SYT(\caly)=$ set of SYT for a $\caly\in YD_{n_b}$.


$dim(\caly)=$ dimension of irrep $\caly$.

\beq
dim(\caly)=|SYT(\caly)|
\eeq
For example, the 
irrep 

\beq
\caly=
\ydiagram{2,1,1}
\eeq
has dimension 3 because there are 3 possible SYT for this YD:

\beq
\begin{ytableau}
1&2
\\
3
\\
4
\end{ytableau}
,\quad
\begin{ytableau}
1&3
\\
2
\\
4
\end{ytableau}
,\quad
\begin{ytableau}
1&4
\\
2
\\
3
\end{ytableau}
\implies dim(\caly)=3
\eeq
Thus, we can denote the
basis vectors of this irrep by

\beq
\ket{\bcen\begin{ytableau}
1&2
\\
3
\\
4
\end{ytableau}\ecen}
,\quad
\ket{\bcen\begin{ytableau}
1&3
\\
2
\\
4
\end{ytableau}\ecen}
,\quad
\ket{\bcen\begin{ytableau}
1&4
\\
2
\\
3
\end{ytableau}\ecen}
\eeq

To compute $hook(\caly)$:

\begin{enumerate}
\item Fill each box of the YD with the number of boxes below and to the right of the box, including the
box itself. 
\item Multiply
the numbers in
all the boxes.
\end{enumerate}
For example,

\beq
\caly=
\bcen
\ydiagram{4,3,2}
\ecen
\implies
hook(\caly)=
\bcen
\begin{ytableau}
6&5&3&1
\\
4&3&1
\\
2&1
\end{ytableau}
\ecen
=6! 3
\eeq


\begin{claim} ({\bf hook rule} for computing $dim(\caly)$)
\beq
dim(\caly)=
\frac{n_b!}{hook(\caly)}
\eeq
\end{claim}
\proof
\qed

For example

\beq
\caly=\ydiagram{2,1,1}
\implies
hook(\caly)=
\begin{ytableau}
4&1
\\
2
\\
1
\end{ytableau}=
8
\eeq
so
\beq
dim(\caly)= \frac{4!}{4(2)}=3
\eeq

The {\bf regular representation} of the
symmetric group $S_{n_b}$ is defined as follows.
For each permutation $\s \in S_{n_b}$, define
an independent vector $\ket{\s}$
in a vector space $\calv=\{\ket{\s}: \s \in S_{n_b}\}=\{\ket{\s_i} : i=1, 2, \ldots , n_b!\}$. Let 


\beq
\ket{x}= \sum_i x_i
\ket{\s_i}
\eeq
For any $\tau\in S_{n_b}$, suppose

\beq
\bra{\s_j}\tau\ket{\s_i}=
\av{\tau^{-1}\s_j|
\s_i}
\eeq


\beq
\bra{\s_j}\tau\ket{x}=
\av{\tau^{-1}\s_j|
x}
\eeq
The regular
rep is $n_b!$ dimensional
and reducible.

\begin{claim}
The regular rep of $S_{n_b}$
decomposes into each $\lam\in rreps(S_{n_b})$,
appearing $dim(\lam)$ times. Thus

\begin{align}
n_b! = |S_{n_b}| &=\sum_{\lam\in irreps(S_{n_b})}[dim(\lam)]^2
\\
&=[n_b!]^2\sum_{\caly \in YD_{n_b}}
\frac{1}{[hook(\caly)]^2}\quad
\text{(Because  $irreps(S_{n_b})=YD_{n_b}$)}
\end{align}
Hence,

\beq
1 = n_b!\sum_{\caly \in YD_{n_b}}
\frac{1}{[hook(\caly)]^2}
\label{eq-1-hook-sq}
\eeq


\end{claim}
\proof
\qed

\beq
1 = \sum_{\caly \in YD_{n_b}} P_\caly
\eeq

\beq
P_\caly = \sum_{\caly_\alp \in SYT(\caly)} \ket{\caly_\alp}\bra{\caly_\alp}
\eeq
The projection operators  
are complete and orthogonal.


\beq
\ydiagram{1}\otimes \ydiagram{1}=
\ydiagram{2} \oplus \ydiagram{1,1}
\eeq


\beq
\begin{array}{l}
\ydiagram{3}
\otimes\ydiagram{2,1}=
\\
\\
\ydiagram{5,1}
\oplus
\ydiagram{4,2}
\oplus
\ydiagram{4,1,1}
\oplus
\ydiagram{3,2,1}
\end{array}
\eeq


\section{Unitary group $U(n)$}

$SYT(n_b, R\leq n)=$ SYTs with $n_b$ boxes and less then $n$ rows

\beq
irreps(U(n))=
\cup_{n_b\leq n}
SYT(n_b, R\leq n)
\eeq

A SYT with $n_b$ boxes represents a 
tensor with $n_b$ indices ($n_b$-particles state). Each index ranges from $1$ to $n$.

$n_b=1$: A 1-index, 1-box tensor is a 1-particle
with $n$ states. This corresponds to the
fundamental representation.

$n_b=2$: A 2-index, 2-box tensor is a 2-particle
with $n^2$ states. These $n^2$ states 
break into two sets, symmetric and anti-symmetric. 

\beqa
\bcen\begin{ytableau}1
\end{ytableau}
\ecen\otimes 
\bcen\begin{ytableau}1
\end{ytableau}
\ecen &=&
\begin{ytableau}1\\2
\end{ytableau} \quad\oplus\quad \begin{ytableau}1&2
\end{ytableau}
\\
\xymatrix@R=1pc@C=1pc{
&&\ar[ll]
\\
&&\ar[ll]
}
&=&
\bcen\xymatrix@R=1pc@C=1pc{
&\ar[l]\ar@2{-}[d]\cala_2&\ar[l]
\\
&\ar[l]&\ar[l]
}\ecen
+
\bcen\xymatrix@R=1pc@C=1pc{
&\ar[l]\ar@2{-}[d]\cals_2&\ar[l]
\\
&\ar[l]&\ar[l]
}\ecen
\eeqa

The SYT of  an irrep describes
the symmetry of the indices
of a tensor in that irrep.
A single column SYT indicates a
totally
anti-symmetric tensor, a
single row SYT indicates a totally symmetric tensor,
and a neither single column nor single
row SYT indicates mixed symmetry. This
is why we can't have more than $n$ rows,
because there are only $n_b$ integers
to fill all boxes so more
than $n$ rows would require a  repetition
in of an integer in a column, and
such a column, after antisymetrizing, equals zero.

$YD_{n_b}=$ set of YD with $n_b$ boxes

$YT(\caly)=$ set of YT for a $\caly\in YD_{n_b}$.

If $\caly_\alp$ is a SYT in $irreps(U(n))$
and the YD of $\caly_\alp$
is  $\caly$, then
\beq
dim(\caly_\alp)= |YT(\caly)|
\label{eq-dim-yalp}
\eeq
Hence, $\{\ket{\alp}:  \alp\in YT(\caly)\}$
are a basis for the 
irrep $\caly_\alp$.
Note that the irreps of $U(n)$ are given by SYT  $\caly_\alp$,
whereas the basis vectors of an irrep are given by YT (not just STY).
For example,
for

\beq
\caly_\alp=
\begin{ytableau}1&2
\end{ytableau}
\eeq
the basis vectors are



\beq
\ket{\begin{ytableau}1&1
\end{ytableau}}
,\quad
\ket{\begin{ytableau}1&2
\end{ytableau}}
,\quad
\ket{\begin{ytableau}2&2
\end{ytableau}}
\eeq
so
\beq
dim(\caly_\alp)=
3
\eeq




\section{Young Projection operators}

For each in the SYT $\caly_\alp\in irreps(U(n))$, we have

\beq
P_{\caly\alp}=
\caln_\caly
\left(\prod_i S_i\right)
\left(\prod_j A_j
\right)
\eeq
Note that the normalization
constant $\caln_\caly$
depends only on the
YD $\caly$.

These projection
operators are not
unique.

\begin{claim}
Let 
\beq
\caln_\caly
=
\frac{
\left(\prod_i |S_{i}|!\right)
\left(\prod_j |A_j|!
\right)
}{hook(\caly)}
\label{eq-n-caly}
\eeq
where $|S_i|$
and $|A_j|$ are
the number of arrows entering 
the symmetrizer or
anti-symmetrizer. Then the operators $P_{\caly_\alp}$ are idempotent 
(i.e., their square equals themselves) mutually orthogonal and complete:
\beq
P_{\caly_\alp}
P_{\caly_\beta}=
P_{\caly_\alp}
\delta(\alp, 
\beta)
\eeq
\beq
1 =\sum_{\caly_\alp\in SYT(n_b, R<n)}
P_{\caly_\alp}
\label{eq-complete-un-proj}
\eeq
\end{claim}
\proof

\beq
P_{\caly_\alp}=
\caln_\caly
\frac{1}
{\prod_i |S_i|!
\prod_jj |A_j|!}
\left(
\bcen
\underbrace
{\xymatrix@C=1pc@R=1pc{
&&\ar[ll]
\\
&&\ar[ll]
\\
\vdots
\\
&&\ar[ll]
}}_\indi
\ecen
+
\cdots
\right)
\eeq
From Eq.(\ref{eq-complete-un-proj})

\beqa
\indi
&=&
\sum_{\caly_\alp\in SYT(n_b, R<n)}
\caln_\caly
\frac{1}
{\prod_i |S_i|!
\prod_j |A_j|!}
\quad\indi
\\
&=&
\sum_{\caly}
\frac{n_b!}{hook(\caly)}
\caln_\caly
\frac{1}
{\prod_i |S_i|!
\prod_j |A_j|!}
\quad \indi
\\
&=&
\sum_\caly
\frac{n_b!}{
[hook(\caly)]^2}
\frac{1}
{\prod_i |S_i|!
\prod_j |A_j|!}
\quad \indi
\quad
\text{(if assume Eq.(\ref{eq-n-caly}))}
\\
&=& \indi \quad\text{
(by Eq.(\ref{eq-1-hook-sq}))}
\eeqa
\qed

Let $dim(\caly_\alp)$
be the dimension of an irrep 
of $U(n)$
with STY given by $\caly_\alp\in STY(n_\rvb, R<n)$. In Eq.(\ref{eq-dim-yalp})
we gave a way of finding $dim(\caly_\alp)$
A second way is by taking the trace of
the corresponding projection operator
\beq
dim(\caly_\alp)=
\tr(P_{\caly_\alp})\eeq
For example, if

\beq
\caly_\alp=
\begin{ytableau}
1&2
\end{ytableau}
\eeq
then

\beqa
d_{\caly_\alp}
&=&
\bcen
\xymatrix@R=1pc@C=1pc{
&\ar[l]\ar@2{-}[d]\cals_2
&\ar[l]\ar@{-}@[red]@/_1pc/[ll]
\\ 
&\ar[l]&\ar[l]\ar@{-}@[red]@/_1pc/[ll]
}\ecen
\\
&=&
\frac{1}{2}
\left(
\bcen
\xymatrix@R=1.5pc@C=1pc{
&
&\ar[ll]\ar@{-}@[red]@/_1pc/[ll]
\\ 
&
&\ar[ll]\ar@{-}@[red]@/_1pc/[ll]
}\ecen
+
\bcen
\xymatrix@R=1.5pc@C=1pc{
&\ar[l]\ar@{<->}[d]
&\ar[l]\ar@{-}@[red]@/_1pc/[ll]
\\ 
&\ar[l]&\ar[l]\ar@{-}@[red]@/_1pc/[ll]
}\ecen\right)
\\
&=& \frac{1}{2}(n^2 + n)
\\
&=&3 \text{ for $U(n=2)$}
\eeqa

A third way of computing $dim(\caly_\alp)$
is by computing the hook and coat functions
and using the formula

\beq
dim(\caly_\alp)=
\frac{coat(\caly)}{hook(\caly)}
\eeq
Note that right
hand side is independent of $\alp$; it 
depends only on the YD.
We've already discussed how to compute
$hook(\caly)$.
$coat (\caly)$ is
calculated as follows.\footnote{
I invented that name. I don't know if  it has a name.}

\begin{enumerate}
\item Fill $\caly$
with 
\begin{itemize}
\item $n$ at the diagonal blocks
\item $n$ increments
increasing by 1 when reading from left to right
\item
$n$ increments
decreasing by 1 
when reading from
top to bottom
\end{itemize}
\item multiply all
the boxes
\end{enumerate}

Examples

\beq
dim(\begin{ytableau}
1&2
\end{ytableau}
)  =
\frac{\begin{ytableau}
$\scriptsize n$& $\scriptsize n+1$
\end{ytableau}
}{
\begin{ytableau}
2&1
\end{ytableau}
}
= \frac{n(n+1)}{2}
\eeq


\beq
dim(\begin{ytableau}
1\\2
\end{ytableau}
)  =
\frac{\begin{ytableau}
$\scriptsize n$\\ $\scriptsize n-1$
\end{ytableau}
}{
\begin{ytableau}
2\\1
\end{ytableau}
}
= \frac{n(n-1)}{2}
\eeq

\beq
dim(\begin{ytableau}
1&2&3&4
\\
5&6
\\
7
\end{ytableau}
)  =
\frac{\begin{ytableau}
$\scriptsize n$
&$\scriptsize n+1$
&$\scriptsize n+2$
&$\scriptsize n+3$
\\
$\scriptsize n-1$
&$\scriptsize n$
\\
$\scriptsize n-2$
\end{ytableau}
}{
\begin{ytableau}
6&4&2&1
\\
3&1
\\
1
\end{ytableau}
}
= \frac{n^2(n^2-1)(n^2-4)(n+3)}
{144}
\eeq




\section{Young Projection operators for $n_b=1,2,3,4$}

\beq
\bcen\xymatrix@R=1pc@C=1pc{
&\ar[l]\ar@2{-}[d]\cals_2
&\ar[l]
\\
&\ar[l]\ar@2{-}[d]\square
&\ar[l]
\\
&\ar[l]
&\ar[l]
}\ecen
=
\bcen\xymatrix@R=1pc@C=1pc{
&\ar[l]
&\ar[l]\ar@2{-}[d]\cals_2
&\ar[l]
&\ar[l]
\\
&\ar[l]\ar@{<->}[d]
&\ar[l]
&\ar[l]\ar@{<->}[d]
&\ar[l]
\\
&\ar[l]
&\ar[l]
&\ar[l]
&\ar[l]
}
\ecen
\eeq




\newcommand{\YTone}[0]{
$
\bcen
\text{
\begin{ytableau}
1
\end{ytableau}}
\\
\xymatrix{
&\ar[l]
}
\ecen
$
}

\newcommand{\YTs}[0]{
$
\bcen
\text{
\begin{ytableau}
1 & 2
\end{ytableau}
}
\\
\xymatrix@R=1pc@C=1pc{
&\ar[l]\ar@2{-}[d]\cals_2&\ar[l]
\\
&\ar[l]&\ar[l]
}\ecen
$
}
\newcommand{\YTss}[0]{
$
\bcen
\text{
\begin{ytableau}
1 & 2 &3
\end{ytableau}
}
\\
\xymatrix@R=1pc@C=1pc{
&\ar[l]\ar@2{-}[dd]\cals_2&\ar[l]
\\
&\ar[l]&\ar[l]
\\
&\ar[l]&\ar[l]
}\ecen
$
}

\newcommand{\YTsss}[0]{
$
\bcen
\text{
\begin{ytableau}
1 & 2 &3 &4
\end{ytableau}
}
\\
\xymatrix@R=1pc@C=1pc{
&\ar[l]\ar@2{-}[ddd]\cals_2&\ar[l]
\\
&\ar[l]&\ar[l]
\\
&\ar[l]&\ar[l]
\\
&\ar[l]&\ar[l]
}\ecen
$
}


\newcommand{\YTa}[0]{
$\bcen
\begin{ytableau}
1 \\2
\end{ytableau}
\\
\xymatrix@R=1pc@C=1pc{
&\ar[l]\ar@2{-}[d]\cala_2&\ar[l]
\\
&\ar[l]&\ar[l]
}
\ecen $
}
\newcommand{\YTaa}[0]{
$\bcen
\begin{ytableau}
1 \\2\\3
\end{ytableau}
\\
\xymatrix@R=1pc@C=1pc{
&\ar[l]\ar@2{-}[dd]\cala_2&\ar[l]
\\
&\ar[l]&\ar[l]
\\
&\ar[l]&\ar[l]
}
\ecen $
}

\newcommand{\YTaaa}[0]{
$\bcen
\begin{ytableau}
1 \\2\\3\\4
\end{ytableau}
\\
\xymatrix@R=1pc@C=1pc{
&\ar[l]\ar@2{-}[ddd]\cala_2&\ar[l]
\\
&\ar[l]&\ar[l]
\\
&\ar[l]&\ar[l]
\\
&\ar[l]&\ar[l]
}
\ecen $
}

\newcommand{\YTsa}[0]{
$
\bcen
\text{
\begin{ytableau}
1 & 2\\3
\end{ytableau}
}
\\
\xymatrix@R=1pc@C=1pc{
&\ar[l]\ar@2{-}[d]\cals_2
&\ar[l]\ar@2{-}[d]\cala_2
&\ar[l]
\\ 4/3
&\ar[l]&\ar[l]\ar@2{-}[d] \square&\ar[l]
\\
&\ar[l]&\ar[l]&\ar[l]
}\ecen
$
}
\newcommand{\YTas}[0]{
$
\bcen
\text{
\begin{ytableau}
1 & 3\\2
\end{ytableau}
}
\\
\xymatrix@R=1pc@C=1pc{
&\ar[l]\ar@2{-}[d]\cals_2
&\ar[l]\ar@2{-}[d]\cala_2
&\ar[l]
\\ 4/3
&\ar[l]\ar@2{-}[d] \square
&\ar[l]
&\ar[l]
\\
&\ar[l]&\ar[l]&\ar[l]
}\ecen
$
}

\newcommand{\YTasFour}[0]{
$
\bcen
\text{
\begin{ytableau}
1 & 3&4\\2
\end{ytableau}
}
\\
\xymatrix@R=1pc@C=1pc{
&\ar[l]\ar@2{-}[d]\cals_2
&\ar[l]\ar@2{-}[d]\cala_2
&\ar[l]
\\ 3/2
&\ar[l]\ar@2{-}[dd] \square
&\ar[l]
&\ar[l]
\\
&\ar[l]&\ar[l]&\ar[l]
\\
&\ar[l]&\ar[l]&\ar[l]
}\ecen
$
}

\newcommand{\YTsaFour}[0]{
$
\bcen
\text{
\begin{ytableau}
1 & 2&4\\3
\end{ytableau}
}
\\
\xymatrix@R=1pc@C=1pc{
&\ar[l]\ar@2{-}[dd]\cals_2
&\ar[l]\ar@2{-}[d]\cala_2
&\ar[l]
\\ 3/2
&\ar[l]
&\ar[l]\square \ar@2{-}[d]
&\ar[l]
\\
&\ar[l]\ar@2{-}[d] \square
&\ar[l]
&\ar[l]
\\
&\ar[l]&\ar[l]&\ar[l]
}\ecen
$
}


\newcommand{\YTsFour}[0]{
$
\bcen
\text{
\begin{ytableau}
1 & 2&3\\4
\end{ytableau}
}
\\
\xymatrix@R=1pc@C=1pc{
&\ar[l]\ar@2{-}[dd]\cals_2
&\ar[l]\ar@2{-}[d]\cala_2
&\ar[l]
\\ 3/2
&\ar[l]
&\ar[l]\square \ar@2{-}[d]
&\ar[l]
\\ 
&\ar[l]&\ar[l]\square \ar@2{-}[d]
&\ar[l]
\\
&\ar[l]&\ar[l]&\ar[l]
}\ecen
$
}

\newcommand{\YTsquare}[0]{
$
\bcen
\text{
\begin{ytableau}
1 & 2\\
3&4
\end{ytableau}
}
\\
\xymatrix@R=1pc@C=1pc{
&\ar[l]\ar@2{-}[d]\cals_2
&\ar[l]
&\ar[l]\ar@2{-}[d]\cala_2
&\ar[l]
&\ar[l]
\\ 4/3
&\ar[l]
&\ar[l]
&\ar[l]\square\ar@2{-}[d]
&\ar[l]\ar@2{-}[d]\cala_2
&\ar[l]
\\
&\ar[l]
&\ar[l]\ar@2{-}[d]\cals_2
&\ar[l]
&\ar[l]\ar@2{-}[d]\square
&\ar[l]
\\
&\ar[l]
&\ar[l]
&\ar[l]
&\ar[l]
&\ar[l]
}\ecen
$
}

\newcommand{\YTsquareP}[0]{
$
\bcen
\text{
\begin{ytableau}
1 & 3\\
2&4
\end{ytableau}
}
\\
\xymatrix@R=1pc@C=1pc{
&\ar[l]
&\ar[l]\ar@2{-}[d]\cals_2
&\ar[l]
&\ar[l]\ar@2{-}[d]\cala_2
&\ar[l]
\\ 4/3
&\ar[l]\ar@2{-}[d]\cals_2
&\ar[l]\square\ar@2{-}[d]
&\ar[l]
&\ar[l]
&\ar[l]
\\
&\ar[l]\ar@2{-}[d]\square
&\ar[l]
&\ar[l]\ar@2{-}[d]\cala_2
&\ar[l]
&\ar[l]
\\
&\ar[l]
&\ar[l]
&\ar[l]
&\ar[l]
&\ar[l]
}\ecen
$
}

\newcommand{\YTttwo}[0]{
$
\bcen
\text{
\begin{ytableau}
1 & 2
\\
3
\\
4
\end{ytableau}
}
\\
\xymatrix@R=1pc@C=1pc{
&\ar[l]\ar@2{-}[d]\cals_2
&\ar[l]\ar@2{-}[d]\cala_2
&\ar[l]
\\ 3/2
&\ar[l]
&\ar[l]\square \ar@2{-}[dd]
&\ar[l]
\\
&\ar[l]
&\ar[l]
&\ar[l]
\\
&\ar[l]&\ar[l]&\ar[l]
}\ecen
$
}

\newcommand{\YTtthree}[0]{
$
\bcen
\text{
\begin{ytableau}
1 & 3
\\
2
\\
4
\end{ytableau}
}
\\
\xymatrix@R=1pc@C=1pc{
&\ar[l]\ar@2{-}[d]\cals_2
&\ar[l]\ar@2{-}[dd]\cala_2
&\ar[l]
\\ 3/2
&\ar[l]\square \ar@2{-}[d]
&\ar[l]
&\ar[l]
\\
&\ar[l]
&\ar[l]\square \ar@2{-}[d]
&\ar[l]
\\
&\ar[l]&\ar[l]&\ar[l]
}\ecen
$
}

\newcommand{\YTtfour}[0]{
$
\bcen
\text{
\begin{ytableau}
1 & 4
\\
2
\\
3
\end{ytableau}
}
\\
\xymatrix@R=1pc@C=1pc{
&\ar[l]\ar@2{-}[d]\cals_2
&\ar[l]\ar@2{-}[dd]\cala_2
&\ar[l]
\\ 3/2
&\ar[l]\square \ar@2{-}[d]
&\ar[l]
&\ar[l]
\\
&\ar[l]\square \ar@2{-}[d]
&\ar[l]
&\ar[l]
\\
&\ar[l]&\ar[l]&\ar[l]
}\ecen
$
}



\begin{itemize}

\item $n_b=1$

\YTone

\item $n_b=2$

\YTs
\YTa

\item  $n_b=3$

\YTss

\YTsa
\quad\YTas

\YTaa

\item $n_b=4$

\YTsss

\YTsFour
\quad\YTsaFour
\quad\YTasFour

\YTsquare
\quad\YTsquareP

\YTttwo
\quad\YTtthree
\quad\YTtfour

\YTaaa
\end{itemize}
SYT
and corresponding unnormalized projection operators for $n_b=1, 2,3,4$

\section{Young Projection Operator with swaps}
\beq
\begin{array}{cc}
\begin{array}{l}
\caly_\alp=
\\
\begin{tabular}{llllll}
 & $A_a$ & $A_b$ & $A_c$ & $A_d$ & $A_e$ \\ \cline{2-6} 
\multicolumn{1}{l|}{$S_x$} & \multicolumn{1}{l|}{1} & \multicolumn{1}{l|}{2} & \multicolumn{1}{l|}{3} & \multicolumn{1}{l|}{4} & \multicolumn{1}{l|}{5} \\ \cline{2-6} 
\multicolumn{1}{l|}{$S_y$} & \multicolumn{1}{l|}{6} & \multicolumn{1}{l|}{7} & \multicolumn{1}{l|}{8} & \multicolumn{1}{l|}{9} &  \\ \cline{2-5}
\multicolumn{1}{l|}{$S_z$} & \multicolumn{1}{l|}{10} & \multicolumn{1}{l|}{11} &  &  &  \\ \cline{2-3}
\end{tabular}
\\
\\
\begin{tabular}{|>{$}c<{$}|}
\hline
a1 \xymatrix{&\ar@{<->}[l]} 1 \\ \hline
a2 \xymatrix{&\ar@{<->}[l]} 6 \\ \hline
a3 \xymatrix{&\ar@{<->}[l]} 10 \\ \hline
b1 \xymatrix{&\ar@{<->}[l]} 2 \\ \hline
b2 \xymatrix{&\ar@{<->}[l]} 7 \\ \hline
b3 \xymatrix{&\ar@{<->}[l]} 11 \\ \hline
c1 \xymatrix{&\ar@{<->}[l]} 3 \\ \hline
c2 \xymatrix{&\ar@{<->}[l]} 8 \\ \hline
d1 \xymatrix{&\ar@{<->}[l]} 4 \\ \hline
d2 \xymatrix{&\ar@{<->}[l]} 9 \\ \hline
e1 \xymatrix{&\ar@{<->}[l]} 5 \\ \hline
\end{tabular}

\end{array}
&
P_{\caly_\alp}=
\bcen
\xymatrix@R=1pc@C=.5pc{
\scriptstyle
1 (x1,a1)
&\ar@{-}[l]
S_x \ar@2{-}[dddd]
&&&&&&&&&&\ar@{-}[llllllllll]
A_a\ar@2{-}[dd]
&\ar@{-}[l]
\\ \scriptstyle 2 (x2,a2)
&\ar@{-}[l]
&\ar@{<->}[dddd]
&&&&&&&&&\ar@{-}[llllllllll]
&\ar@{-}[l]
\\\scriptstyle 3 (x3,a3)
&\ar@{-}[l]
&&\ar@{<->}[ddddddd]
&&&&&&&&\ar@{-}[llllllllll]
&\ar@{-}[l]
\\\scriptstyle 4 (x4, b1)
&\ar@{-}[l]
&&&\ar@{<->}[uu]
&&&&&&&\ar@{-}[llllllllll]
A_b\ar@2{-}[dd]
&\ar@{-}[l]
\\\scriptstyle 5 (x5, b2)
&\ar@{-}[l]
&&&&\ar@{<->}[dd]
&&&&&&\ar@{-}[llllllllll]
&\ar@{-}[l]
\\\scriptstyle 6  (y1, b3)
&\ar@{-}[l]
S_y \ar@2{-}[ddd]
&&&&&\ar@{<->}[ddddd]
&&&&&\ar@{-}[llllllllll]
&\ar@{-}[l]
\\\scriptstyle 7 (y2, c1)
&\ar@{-}[l]
&&&&&&\ar@{<->}[uuuu]
&&&&\ar@{-}[llllllllll]
A_c\ar@2{-}[d]&\ar@{-}[l]
\\\scriptstyle 8 (y3,c2)
&\ar@{-}[l]
&&&&&&&&&&\ar@{-}[llllllllll]
&\ar@{-}[l]
\\\scriptstyle 9 (y4,d1)
&\ar@{-}[l]
&&&&&&
&\ar@{<->}[uuuuu]
&&
&\ar@{-}[llllllllll]A_d\ar@2{-}[d]
&\ar@{-}[l]
\\\scriptstyle 10 (z1, d2)
&\ar@{-}[l]
S_z \ar@2{-}[d]
&&&&&&&&\ar@{<->}[u]
&&\ar@{-}[llllllllll] 
&\ar@{-}[l]
\\\scriptstyle 11 (z2, e1)
&\ar@{-}[l]
&&&&&&&&&\ar@{<->}[uuuuuu]&\ar@{-}[llllllllll]
A_e
&\ar@{-}[l]
}
\ecen
\end{array}
\eeq

\section{Tensor product decompositions}

\beqa
\begin{ytableau}
1
\end{ytableau}
\otimes
\begin{ytableau}
2
\end{ytableau}
\otimes
\begin{ytableau}
3
\end{ytableau}
&=&
\left(
\begin{ytableau}
1 & 2
\end{ytableau}
\oplus
\begin{ytableau}
1 \\ 2
\end{ytableau}
\right)
\otimes 
\begin{ytableau}
3
\end{ytableau}
\\
&=&
\begin{ytableau}
1 & 2 & 3
\end{ytableau}
\oplus
\begin{ytableau}
1 & 2 \\ 3
\end{ytableau}
\oplus
\begin{ytableau}
1 & 3 \\ 2
\end{ytableau}
\oplus
\begin{ytableau}
1 \\ 2 \\3
\end{ytableau}
\eeqa

\beq
\begin{array}{l}
\bcen\xymatrix@R=1pc@C=1pc{
&&\ar[ll]
\\
&&\ar[ll]
\\
&&\ar[ll]
}\ecen
=
\\ \\
\bcen
\xymatrix@R=1pc@C=1pc{
&\ar[l]\ar@2{-}[dd]\cals_2&\ar[l]
\\
&\ar[l]&\ar[l]
\\
&\ar[l]&\ar[l]
}\ecen
+
\bcen
\xymatrix@R=1pc@C=1pc{
&\ar[l]\ar@2{-}[d]\cals_2
&\ar[l]\ar@2{-}[d]\cala_2
&\ar[l]
\\ 4/3
&\ar[l]&\ar[l]\ar@2{-}[d] \square&\ar[l]
\\
&\ar[l]&\ar[l]&\ar[l]
}\ecen
+
\bcen
\xymatrix@R=1pc@C=1pc{
&\ar[l]\ar@2{-}[d]\cals_2
&\ar[l]\ar@2{-}[d]\cala_2
&\ar[l]
\\ 4/3
&\ar[l]\ar@2{-}[d] \square
&\ar[l]
&\ar[l]
\\
&\ar[l]&\ar[l]&\ar[l]
}\ecen
+
\bcen
\xymatrix@R=1pc@C=1pc{
&\ar[l]\ar@2{-}[dd]\cala_2&\ar[l]
\\
&\ar[l]&\ar[l]
\\
&\ar[l]&\ar[l]
}
\ecen
\end{array}
\eeq


\beq
n^3 =
\frac{n(n+1)(n+2)}{6}
+
\frac{n(n^2-1)}{3}
+
\frac{n(n^2-1)}{3}
=
\frac{(n-2)(n-1)n }
{6}
\eeq

\beq
\begin{array}{l}
\begin{ytableau}
1&2&3
\end{ytableau}
\otimes
\begin{ytableau}
4&5
\\
6
\end{ytableau}
=
\\
\\
\left\{
\begin{array}{l}
\begin{ytableau}
1&2&3&4&5
\\
6
\end{ytableau}
\oplus
\begin{ytableau}
1&2&3&4
\\
5&6
\end{ytableau}
\oplus
\begin{ytableau}
1&2&3&5
\\
4&6
\end{ytableau}
\\ \\
\oplus
\begin{ytableau}[*(blue!20)]
1&2&3&5
\\
4
\\
6
\end{ytableau}
\oplus
\begin{ytableau}[*(blue!20)]
1&2&3&4
\\
5
\\
6
\end{ytableau}
\oplus
\begin{ytableau}[*(blue!20)]
1&2&3
\\
4&5
\\
6
\end{ytableau}
\end{array}
\right\}
\end{array}
\eeq
For $U(n)$, the blue YT are zero for $n_b=2$,
and non-zero otherwise.

\section{$SU(n)$}
A third way,
besides RL and RC,
 of labelling YD (reps of 
$SU(n)$)
is
by their dimension, and then adding
a  subscript or overline
if there are more than one 
reps with a different YD but
the same dimension.
This method is used mostly
by physicists for $SU(3)$ (The Eightfold  Way).
Note that all YT
with the same YD have the
same dimension, so
this really labels YD instead
of YT. For example, for $SU(3)$  we have

\beq
\begin{array}{ccc}
\ydiagram{1}=[1,0]_{RC}= 3
&\quad\quad&
\ydiagram{2}=[0,1]_{RC}= \ol{3}
\\
\\
\ydiagram{2}=[2,0]_{RC}= 6
&\quad\quad&
\ydiagram{2,2}=[0,2]_{RC}= \ol{6}
\\
\\
\ydiagram{2,1}=[1,1]_{RC}= 8
&\quad\quad&
\ydiagram{3,1}=[2,1]_{RC}= 15
\end{array}
\eeq



