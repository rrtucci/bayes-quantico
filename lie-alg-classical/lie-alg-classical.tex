\chapter{Lie Algebras of Classical Groups}
\label{ch-lie-alg-classical}

In this chapter, we
present an overview of
the Lie algebras
for the classical simple Lie groups.


Recall from Chapter \ref{ch-dynkin} that the Lie algebras of the simple Lie groups 
can be divided into two classes: the Classical and the Exceptional.
The Lie algebras of the classical simple Lie groups are
\begin{itemize}
\item $A_n = \ger{su}(n+1)$
\item $B_n = \ger{so}(2n+1)$
\item $C_n = \ger{sp}(2n)$
\item $D_n = \ger{so}(2n)$
\end{itemize}
and the Lie algebras of Exceptional 
simple Lie groups are:
\begin{itemize}
\item $E_6, E_7, E_8$
\item $F_4$
\item $G_2$

\end{itemize}

\section{$SU(n)$}




\beq
SU(n) = \{ U \in \CC^{n\times n} : U^\dagger U = I,\ \det U = 1 \}
\eeq

Note that
\beq
1=U^\dagger U = e^{X^\dagger} e^X 
\implies X^\dagger = -X
\eeq

\beq 
1=\det U = \det e^X= e^{\tr X}\implies \tr X = 0
\eeq
Thus:

\beq
\ger{su}(n)=\{X\in \CC^{n\times n}: X^\dagger = -X,\quad\tr X=0\}
\eeq

\begin{claim}(Real dimension) 

\beq
\dim_\RR \ger{su}(n) = n^2 - 1
\eeq
\end{claim}
\proof

Real parameter count:
\begin{itemize}
 \item +1 for each diagonal entry
\item +1 for the real part of each entry above the diagonal
\item
+1 for the imaginary part of each 
entry below the diagonal
\item $-1$ for the zero trace constraint.
\end{itemize}
This adds up to $n^2-1$.
\qed

Let $X_j$ be the generators of the Lie algebra, and set
\beq
X_j = -i T_j.
\eeq

By definition, the generators $X_j$ are closed under commutation
\beq
[X_i, X_j] = f_{ijk} X_k
\implies [T_i, T_j] = i f_{ijk} T_k
\eeq
where $f_{ijk}$ are the real structure constants of $\ger{su}(n)$

\beq
X^\dagger = -X,\quad \tr X=0 \implies
(T_i)^\dagger = T_i, \quad \tr(T_i)=0
\eeq

It's also possible to assume
\beq
\tr(T_i T_j) = \frac{1}{2} \delta_{ij}
\eeq



For $SU(2)$, $T^i=\frac{1}{2}\s_i$ (Pauli matrices)

For $SU(3)$, $T^i=\frac{1}{2}\lam_i$ (Gel-Mann matrices)


\section{$SO(n)$}

\beq
SO(n)= \{G\in \RR^{n\times n}: G^T G= 1, \quad\det G=1\}
\eeq


\beq
\ger{so}(n) = \{ X \in \RR^{n\times n} |  X^T = -X\}
\eeq
Note that $X^T=-X$ implies that the diagonal of $X$ is zero, so $\tr X=0$
automatically.
\begin{claim}
\beq
\dim_\RR \ger{so}(n) = \frac{n(n-1)}{2}
\eeq
\end{claim}
\proof


Real parameter count
\begin{itemize}
\item +1 for each entry above the diagonal
\end{itemize}
This adds up to 
$
\frac{(n^2-n)}{2}
= \frac{n(n-1)}{2}$.
\qed

Define the generator basis
$\{L_{ij}| i<j\}$, where

\beq
(L_{ij})_{ab} = \delta_{ia}\delta_{jb} - \delta_{ja}\delta_{ib}
=
\bcen
\xymatrix{
i\ar[d]&j
\\
a&b\ar[u]
}
\ecen
-
\bcen
\xymatrix{
i\ar[dr]&j\ar[dl]
\\
a&b
}
\ecen
\label{eq-Lij-def}
\eeq

Note that $L_{ij}^T= L_{ji} = - L_{ij}$.
Hence these generators satisfy $X^T=-X$.
Physicists often define generators $T_{ij}=iL_{ij}$ that are Hermitian.

\begin{claim}
\beq
\begin{array}{l}
\myboxed{[L_{ij}, L_{kl}]
= \delta_{jk}L_{il}-\delta_{ik}L_{jl}
-\delta_{jl}L_{ik}+\delta_{il}
L_{jk}}
\\
\bcen
\xymatrix@R=1pc@C=.5pc{
i\ar@[red][dr]
&&j\ar@[red][dl]
&&k\ar[dr]
&&l\ar[dl]
&
\\
&\ar[l]L
&&&
&\ar[llll]L
&\ar[l]
}
\ecen
-
\bcen
\xymatrix@R=1pc@C=.5pc{
k\ar[dr]
&&l\ar[dl]
&&i\ar@[red][dr]
&&j\ar@[red][dl]
&
\\
&\ar[l]L
&&&
&\ar[llll]L
&\ar[l]
}
\ecen
=\\
\left\{
\begin{array}{l}
\bcen
\xymatrix@R=1pc@C=.5pc{
i\ar@[red]@{-}@/_1pc/[rrrrrr]
&&j\ar@[red][dr]
&&k\ar[dl]
&&l
&
\\
&
&&L\ar[lll]
&
&
&\ar[lll]
}
\ecen +
\bcen
\xymatrix@R=1pc@C=.5pc{
i\ar@[red][drrr]
&&j\ar@[red]@{-}@/_.7pc/[rr]
&&k
&&l\ar[dlll]
&
\\
&
&&L\ar[lll]
&
&
&\ar[lll]
}
\ecen
\\ \\
-\bcen
\xymatrix@R=1pc@C=.5pc{
i\ar@[red]@{-}@/_1pc/[rrrr]
&&j\ar@[red][dr]
&&k
&&l\ar[dlll]
&
\\
&
&&L\ar[lll]
&
&
&\ar[lll]
}
\ecen -
\bcen
\xymatrix@R=1pc@C=.5pc{
i\ar@[red][drrr]
&&j\ar@[red]@{-}@/_1pc/[rrrr]
&&k\ar[dl]
&&l
&
\\
&
&
&L\ar[lll]
&
&
&\ar[lll]
}
\ecen
\end{array}
\right\}
\end{array}
\eeq
\end{claim}\
\proof

Just calculate the commutator $[L_{ij}, L_{kl}]$  using the definition Eq.(\ref{eq-Lij-def})
of $L_{ij}$
\qed


\section{$Sp(n)$}

Assume $n$ is even. Then
\beq
Sp(n) =\{U \in \CC^{n\times n} :
U^\dagger U = I, \quad U^T J U = J \}
\eeq
where

\beq
J=
\left(
\begin{array}{cc}
0 & I_{n/2} \\
-I_{n/2} & 0
\end{array}
\right)
\eeq

Note that $J$ satisfies
\beq
J^T = -J,\quad J^TJ=1
\eeq

Note that
\beq
1=U^\dagger U =e^{X^\dagger}e^X \implies  X^\dagger = -X.
\eeq

\beq
U^T J U = J \implies 1= J^T U^T J U=
e^{J^T X^T J}e^{X}
\implies J^T X^T J =- X
\eeq
Hence
\beq
\ger{sp}(n)
= \{ X\in \CC^{n\times n} : X^\dagger = -X,\quad  J^TX^T J =- X \}
\eeq


\begin{claim}
 $X^\dagger = -X$ and $J^T X^TJ=-X$
iff $X$ has the
{\bf A-B form} 
\beq
X =
\begin{pmatrix}
A & B \\
-B^* & A^*
\end{pmatrix}
\eeq
where 

\beq A^\dagger = -A,
\quad
B^T = B
\eeq
\end{claim}
\proof

Let
\beq
X=
\begin{pmatrix}
A&B
\\
C&D
\end{pmatrix}
\eeq
Then
\beqa
J^T X^T J &=&
\begin{pmatrix}
0 & -1
\\
1&0
\end{pmatrix}
\begin{pmatrix}
A^T & C^T
\\
B^T& D^T
\end{pmatrix}
\begin{pmatrix}
0 & 1
\\
-1&0
\end{pmatrix}
\\
&=&
\begin{pmatrix}
-B^T&-D^T
\\
A^T&C^T
\end{pmatrix}
\begin{pmatrix}
0 & 1
\\
-1&0
\end{pmatrix}
\\
&=&
\begin{pmatrix}
D^T&-B^T
\\
-C^T&A^T
\end{pmatrix}
\eeqa

\beq
J^T X^T J +X=
\begin{pmatrix}
D^T + A & -B^T + B
\\
-C^T + C & A^T + D
\end{pmatrix}
=
0
\eeq
Hence
\beq
\boxed{B^T=B, \quad C^T=C,\quad A^T=-D}
\label{eq-lie-alg-class-box1}
\eeq
Also
\beq
X^\dagger + X=
\begin{pmatrix}
A^\dagger + A& C^\dagger + B
\\
B^\dagger+C&D^\dagger + D
\end{pmatrix}
=0
\eeq
Hence
\beq
\boxed {A^\dagger = -A,\quad D^\dagger = -D,
\quad C^\dagger = -B}
\label{eq-lie-alg-class-box2}
\eeq
Combining Eqs.(\ref{eq-lie-alg-class-box1}) and 
(\ref{eq-lie-alg-class-box2}), we get

\beq
C= -B^*, \quad D=A^*
\eeq

\qed

Let

\beq
X = X_A + \eps_B
\eeq
where

\beq
X_A = \begin{pmatrix}
A & 0 \\
0 & A^*
\end{pmatrix},
\quad
\quad X_A^\dagger = -X_A 
\eeq
and

\beq
\eps_B = \begin{pmatrix}
0 & B \\
 -B^* & 0
\end{pmatrix},
 \quad \eps_B^\dagger =-\eps_B
\eeq

Note that
\beqa
J^T \eps_B^\dagger J = -\eps_B,
\quad J^T X_A^\dagger J = - X_A
\eeqa


\begin{claim}(Lie Algebra is closed)
\beq
[X_1,X_2] \in \ger{sp}(n),
\eeq
\end{claim}
\proof

If $X,Y\in \ger{sp}(n)$,
\beq
[X, Y]^\dagger = [Y^\dagger, X^\dagger]= [-Y, -X]= -[X, Y]
\eeq


\beqa
X_A  \eps_B &=&
\begin{pmatrix}
A&0\\
0&A^*
\end{pmatrix}
\begin{pmatrix}
0 & B \\
 -B^* & 0
\end{pmatrix}
\\
&=&
\begin{pmatrix}
0&AB\\
-(AB)^*&0
\end{pmatrix}
\\
&=& \eps_{AB}
\eeqa

\beqa
\eps_B  X_A &=&
\begin{pmatrix}
0 & B \\
 -B^* & 0
\end{pmatrix}
\begin{pmatrix}
A&0\\
0&A^*
\end{pmatrix}
\\
&=&
\begin{pmatrix}
0&B A^*\\
-B^* A&0
\end{pmatrix}
\\
&=& \eps_{BA^*}
\eeqa


\beq
X_{A_1}X_{A_2}= X_{A_1A_2}
\eeq

\beqa
\eps_{B_1}\eps_{B_2}
&=&
\begin{pmatrix}
0 & B_1 \\
- B^*_1 & 0
\end{pmatrix}
\begin{pmatrix}
0 & B_2\\
- B^*_2 & 0
\end{pmatrix}
\\
&=&
\begin{pmatrix}
-B_1B_2^* & 0\\
0& -B^*_1 B_2
\end{pmatrix}
\\
&=& X_{-B_1 B_2^*}
\eeqa


\beq
[X_A, \eps_B]=\eps_{AB - B A^*}
\eeq

\beq
[X_{A_1}, X_{A_2}]=
X_{[A_1,A_2]}
\eeq



\beq
[\eps_{B_1}, \eps_{B_2}]=X_{-B_1 B_2^* - B_2 B_1^*}
\eeq
\qed

\begin{claim} (Real dimension)


\beq
\dim_\RR \ger{sp}(n)= n(2n+1)
\eeq

\end{claim}
\proof

$A^\dagger =-A$ so $A$ contributes  $n^2$ real parameters.

$B^T=B$ so $B$ contributes 
\beq
\frac{(n^2-n)}{2} + n=\frac{(n^2+n)}{2} = \frac{n(n+1)}{2}
\eeq
 complex parameters.

This adds up to 
\beq
n^2 + n(n+1) = n(2n+1)
\eeq
real parameters.
\qed



