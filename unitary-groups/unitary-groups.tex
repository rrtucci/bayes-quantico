\chapter{Unitary Groups}
\label{ch-unitary-groups}

This chapter is based on Cvitanovic's Birdtracks book Ref. \cite{birdtracks-book}.

Let

{\bf n-dim General   Linear group} $GL(n;\CC)=\{ G\in\CC^{n\times n}:
det(G)\neq 0\}$

{\bf n-dim Special Linear group} $SL(n;\CC)=\{ G\in GL(n;\CC):
det(G)=1\}$

{\bf n-dim Unitary group}, $U(n)=\{ G\in GL(n;\CC):
G G^\dagger =G^\dagger G =1\}$

{\bf n-dim Special Unitary group}
$SU(n)=\{ G\in U(n):
det(G)=1\}$

Chapter \ref{ch-young-tableau}
on Young Tableaux
is closely connected to this chapter.

\section{$SU(n)$}

In $SU(n)\subset \CC^{n\times n}$
in the defining rep,
we have the quadratic form

\beq
m(p,q) =   
(p_b)^* \delta_b^a q_a
\eeq
Let

\beq
\indi\indices{_d^a_b^c}=
\delta_b^a\delta_d^c
=
\bcen
\xymatrix@R=1pc{d
&c\ar[l]|\bullet
\\
a\ar[r]|\bullet
&b}
\end{array}
\eeq
and

\beq
M\indices{^d_a^b_c}=
\delta_d^a\delta_b^c
=
\begin{array}{c}
\xymatrix{d
&c\ar@/_1pc/[d]|\bullet
\\
a\ar@/_1pc/[u]|\bullet
&b}
\end{array}
\eeq
Note that

\beq
\begin{array}{ll}
\myboxed{
M^2 = nM
}
&
\bcen
\xymatrix{d
&\ar@/_1pc/[d]|\bullet
\\
a\ar@/_1pc/[u]|\bullet
&}
\xymatrix{
&c\ar@/_1pc/[d]|\bullet
\\
\ar@/_1pc/[u]|\bullet
&b}
\ecen
=
n
\bcen
\xymatrix{d
&c\ar@/_1pc/[d]|\bullet
\\
a\ar@/_1pc/[u]|\bullet
&b}
\ecen
\end{array}\eeq




Hence, $(M-n)M=0$
so $M$ has two eigenvalues $\lam=0,n$.

Next we will use the following equation from Chapter \ref{ch-reducibility}
\footnote{Note that this equation
projects to zero all eigenvalues except one.}
to obtain a projection
operator (PO) for each eigenvalue

\beq 
P_i = \sum_{j\neq i}
\frac{M-\lam_j}{\lam_i-\lam_j}
\eeq

\begin{enumerate}
\item Singlet PO ($\lam_S = 0$)

\beq
\begin{array}{ll}
\myboxed{
P_S =\frac{M-0}{n-0}= \frac{1}{n} M 
}
&
\bcen
\xymatrix@C=1pc@R=1pc{
a
&&b\ar[dl]
\\
&P_S\ar[dr]\ar[lu]
\\
c\ar[ru]&&d
}
\ecen
=
\frac{1}{n}
\bcen
\xymatrix@C=3pc{
a
&b\ar@/_1pc/[d]|\bullet
\\
c\ar@/_1pc/[u]|\bullet
&d}
\ecen
\end{array}\eeq

The singlet projection operator $P_S$ projects the singlet part of a tensor $x$:
\beq
P_{S}x = \frac{1}{n}
x\indices{^b_b}
\delta_a^c
\eeq
$P_S$ has dimension 1:


\beqa
dim(P_S)=\tr P_S &=&
\frac{1}{n}
\bcen
\xymatrix@C=3pc{
&\ar@/_1pc/[d]|\bullet\ar@[red]@{-}@/_.5pc/[l]
\\
\ar@/_1pc/[u]|\bullet
&\ar@[red]@{-}@/^.5pc/[l]
}
\ecen
\\
&=&
1
\eeqa

\item Adjoint PO

\beq
\begin{array}{l}
\myboxed{
P_{adj} = \frac{M-n}{0-n}=
1 -\frac{1}{n}M
}
\\
\bcen
\xymatrix@C=1pc@R=1pc{
a
&&b\ar[dl]
\\
&P_{adj}\ar[dr]\ar[lu]
\\
c\ar[ru]&&d
}
\ecen
=
\bcen
\xymatrix@C=3pc{
a&b\ar[l]|\bullet
\\
c\ar[r]|\bullet&d
}
\ecen
-
\frac{1}{n}
\bcen
\xymatrix{
a
&b\ar@/_1pc/[d]|\bullet
\\
c\ar@/_1pc/[u]|\bullet
&d}
\ecen
=
\bcen
\xymatrix@R=1pc@C=1pc{
&&&\ar@/_1pc/[ld]
\\
&T_i\ar@/_1pc/[lu]&T_i\ar@{~}[l]\ar@/_1pc/[rd]&
\\
\ar@/_1pc/[ru]&&&
}
\ecen
\end{array}
\eeq
The adjoint projection operator $P_{adj}$ projects the traceless part of
a tensor $x$
\beq
P_{adj}x=
x\indices{^a_c}-
\left(
\frac{1}{n}x\indices{^b_b}\delta_a^c
\right)
\eeq

The $P_{adj}$ has dimension $n^2-1$
\beqa
dim(P_{adj})=\tr P_{adj} &=&
\bcen
\xymatrix@C=3pc{
&\ar[l]|\bullet\ar@[red]@{-}@/_.5pc/[l]
\\
\ar[r]|\bullet\ar@[red]@{-}@/_.5pc/[r]&
}
\ecen
-
\frac{1}{n}
\bcen
\xymatrix@C=3pc{
&\ar@/_.5pc/[d]|\bullet\ar@[red]@{-}@/_.5pc/[l]
\\
\ar@/_1pc/[u]|\bullet
&\ar@[red]@{-}@/^.5pc/[l]
}
\ecen
\\
&=& n^2 -1
\eeqa

\end{enumerate}

We will denote
the generators $T_i$ of $SU(n)$ by


\beq
(T_i)\indices{_a^b} = 
\bcen
\xymatrix{
&i\ar@{~}[d]
\\
a&T^i\ar[l]&b\ar[l]
}
\ecen
\eeq
For $G\in U(n)$, $G^\dagger G=1$ with
$G=e^{iT_i\eps_i}$ where $\eps_i\in\RR$.
Hence, the generators $T_i$ 
must be Hermitian

\beq
T_i^\dagger = T_i
\eeq
We will
assume that they 
also satisfy

\beq
\begin{array}{l}
\myboxed{\tr(T_i T_j)=\kappa \delta_i^j}
\\
\\
\xymatrix{
i&\ar@{~}[l] T_i
\ar@/_1pc/[r]
&T_j\ar@/_1pc/[l]
&\ar@{~}[l]j
}
=
\kappa
\xymatrix{
i&\ar@{~}[l]|\bullet j
}
\end{array}
\eeq
Usually, we set $\kappa=1$
and, if necessary, restore the $\kappa$'s at
the end by dimensional
analysis. (Replace each $T_i$
in a $\kappa$-less equation by $T_i/\sqrt{\kappa}$.)

The adjoint projection operator for $SU(n)$ is
\beq
\bcen
\xymatrix@R=1pc@C=1pc{
&&&\ar@/_1pc/[ld]
\\
&T_i\ar@/_1pc/[lu]&T_i\ar@{~}[l]\ar@/_1pc/[rd]&
\\
\ar@/_1pc/[ru]&&&
}
\ecen
\eqdef P_{adj}
=
\bcen
\xymatrix@C=3pc{
&\ar[l]|\bullet
\\
\ar[r]|\bullet&}
\ecen
-
\frac{1}{n}
\bcen
\xymatrix@C=3pc{
&\ar@/_1pc/[d]|\bullet
\\
\ar@/_1pc/[u]|\bullet
&}
\ecen
\eeq

The Lie Algebra commutators for
$SU(n)$ are
\beq
\begin{array}{l}
\myboxed{
T_i T_j - T_j T_i = i f_{ijk}T_k
}
\\
\bcen
\xymatrix{
&\ar[l]T_i
&\ar[l]T_j
&\ar[l]
\\
&\ar@{~}[u]i
&\ar@{~}[u]j
&
}
\ecen
-
\bcen
\xymatrix{
&\ar[l]T_j
&\ar[l]T_i
&\ar[l]
\\
&\ar@{~}[ur]i
&\ar@{~}[ul]j
&
}
\ecen
=
\bcen
\xymatrix{
&\ar[l]T_k
&\ar[l]
\\
&\ar@{~}@[green][u]i f
&
\\
\ar@{~}[ru]i&&\ar@{~}[lu]j
}
\ecen
\end{array}
\eeq
The structure constants
 $f_{ijk}$ for
$SU(n)$ 
is a totally antisymmetric tensor. In the CC convention, the first index of $f_{ijk}$ corresponds to
the green leg in the birdtracks.\footnote{Actually, it doesn't
matter which index is taken first.
This is explained in 
Chapter \ref{ch-birdtracks}}

Multiplying Lie Algebra commutator by by $T_k$
and taking the trace, we get
\beq
\begin{array}{l}
\myboxed{
if_{ijk}= 
\tr([T_i, T_j] T_k)
}
\\
\bcen
\xymatrix{
&\ar@{~}[d]&
\\
&f\ar@{~}[dl]\ar@{~}[dr]&
\\
&&
}
\ecen
=
\bcen
\xymatrix@R=1pc@C=1pc{
&&\ar@{~}[d]&&
\\
&&\ar[dl]T_k&&
\\
&T_i\ar[rr]&&\ar[ul]T_j&
\\
\ar@{~}[ur]&&&&\ar@{~}[ul]
}
\ecen
-
\bcen
\xymatrix@R=1pc@C=1pc{
&&\ar@{~}[d]&&
\\
&&\ar@{<-}[dl]T_k&&
\\
&\ar@{<-}[rr]T_i&&\ar@{<-}[ul]T_j&
\\
\ar@{~}[ur]&&&&\ar@{~}[ul]
}
\ecen
\\
\quad\quad\quad\quad\quad\quad\quad
=2\bcen
\xymatrix@R=1pc@C=1pc{
&\ar@{~}[d]&
\\
&\ar[dl]T_k&
\\
\ar[rr]T_{i'}\ar@{~}[d]
&&\ar[ul]T_{j'}\ar@{~}[d]
\\
\cala_2\ar@2{-}[rr]\ar@{~}[d]
&&\ar@{~}[d]
\\
&&
}
\ecen
\end{array}
\eeq


One can define
a totally symmetric tensor
$d_{ijk}$ analogously by

\beq
\begin{array}{l}
\myboxed{
d_{ijk}= 
\tr([T_i, T_j]_+ T_k)
}
\\
\bcen
\xymatrix{
&\ar@{~}[d]&
\\
&f\ar@{~}[dl]\ar@{~}[dr]&
\\
&&
}
\ecen
=
\bcen
\xymatrix@R=1pc@C=1pc{
&&\ar@{~}[d]&&
\\
&&\ar[dl]T_k&&
\\
&T_i\ar[rr]&&\ar[ul]T_j&
\\
\ar@{~}[ur]&&&&\ar@{~}[ul]
}
\ecen
+
\bcen
\xymatrix@R=1pc@C=1pc{
&&\ar@{~}[d]&&
\\
&&\ar@{<-}[dl]T_k&&
\\
&\ar@{<-}[rr]T_i&&\ar@{<-}[ul]T_j&
\\
\ar@{~}[ur]&&&&\ar@{~}[ul]
}
\ecen
\\
\quad\quad\quad\quad\quad\quad\quad
=2\bcen
\xymatrix@R=1pc@C=1pc{
&\ar@{~}[d]&
\\
&\ar[dl]T_k&
\\
\ar[rr]T_{i'}\ar@{~}[d]
&&\ar[ul]T_{j'}\ar@{~}[d]
\\
\cals_2\ar@2{-}[rr]\ar@{~}[d]
&&\ar@{~}[d]
\\
&&
}
\ecen
\end{array}
\eeq

\begin{claim}.

\begin{itemize}
\item $\tr([T_i, T_j] T_k)$
is totally anti-symmetric
\item $\tr([T_i, T_j]_+ T_k)$
is totally symmetric
\end{itemize}
in the indices $i,j,k$
\end{claim}
\proof

\beq
\tr([T_i, T_j] T_k)
=
-\tr([T_k, T_j] T_i)
\eeq

\beq
\tr([T_i, T_j] T_k)
=
+\tr([T_k, T_j]_+ T_i)
\eeq
\qed

\begin{claim}
\beq
\myboxed{\tr(T_i)=0}
\quad
\xymatrix{
&\ar@{~}[l] T_i
\loopright{5}{}
&
}=0
\eeq
\end{claim}
\proof

\beq
0=P_{adj} P_S=
\bcen
\xymatrix@R=1pc@C=1pc{
&&&\ar@/_1pc/[ld]
&&\ar`l[ldd]`[dd][dd]
\\
&T_i\ar@/_1pc/[lu]
&T_i\ar@{~}[l]\ar@/_1pc/[rd]
&&&
\\
\ar@/_1pc/[ru]
&&&\ar[uu]
&&
}
\ecen
\eeq

\qed

\begin{claim}
\beq
\begin{array}{l}
\myboxed{
\Gamma_{fun}\delta^b_a=
\sum_i
(T_iT_i)\indices{_a^b}
 = \frac{n^2-1}{n}\delta_a^b}
\\
\sum_i
\bcen
\xymatrix{
a&T_i\ar[l]&T_i\ar@{~}@/_2pc/[l]|i\ar[l]&
\ar[l]b
}\ecen
=
\left(\frac{n^2-1}{n}\right)
\xymatrix{
a&\ar[l]|\bullet b 
}
\end{array}
\label{eq-wavy-arc}
\eeq
\end{claim}
\proof

\beqa
(T_iT_i)\indices{_a^b}
&=&
\bcen
\xymatrix{
a&T_i\ar[l]&T_i\ar@{~}@/_2pc/[l]|i\ar[l]&
\ar[l]b
}\ecen
\\
&=&
\bcen
\xymatrix@R=1pc@C=1pc{
a&&&\ar@/_1pc/[ld]b
\\
&T_i\ar@/_1pc/[lu]&T_i\ar@{~}[l]\ar@/_1pc/[rd]&
\\
\ar@/_1pc/[ru]&&&
\ar@/^1pc/[lll]
}
\ecen
\\
&&\nonumber
\\
&=&
\bcen
\xymatrix@C=3pc{
&\ar[l]|\bullet
\\
\ar[r]|\bullet&
\ar@/^1pc/[l]}
\ecen
-
\frac{1}{n}
\bcen
\xymatrix@C=3pc{
&\ar@/_1pc/[d]|\bullet
\\
\ar@/_1pc/[u]|\bullet
&
\ar@/^1pc/[l]}
\ecen
\\
&&\nonumber
\\
&=&\left(n-\frac{1}{n}\right)
\xymatrix{
a&\ar[l]|\bullet b 
}
\eeqa
\qed






\begin{claim}
\beq
\bcen
\xymatrix@R=1pc{
&&T_k\ar[ld]
\ar@{~}[dd]
&&
\\
&T_i\ar[rd]\ar@{~}[l]|i
&&T_j\ar[lu]&\ar@{~}[l]|j
\\
&&T_k\ar[ru]&&
}
\ecen
=
-\frac{1}{n}
\xymatrix{i&\ar@{~}[l]|\bullet j}
\eeq
\end{claim}
\proof
\beq
\bcen
\xymatrix@R=1pc{
&&T_k\ar[ld]
\ar@{~}[dd]
&&
\\
&T_i\ar[rd]\ar@{~}[l]|i
&&T_j\ar[lu]&\ar@{~}[l]|j
\\
&&T_k\ar[ru]&&
}
\ecen
=
\underbrace
{\bcen
\xymatrix@C=1pc{
&
\ar[d]&\ar@/_1pc/[dd]
\\
&\ar@{~}[l]T_i
\ar[d]&T_j\ar[u]&
\ar@{~}[l]
\\
&\ar@/_1pc/[uu]
&\ar[u]
}
\ecen}_{=0}
-
\frac{1}{n}
\bcen
\xymatrix{
&T_i\ar@{~}[l]
\ar@/_1pc/[r]
&T_i\ar@/_1pc/[l]
&\ar@{~}[l]
}
\ecen
\eeq
\qed



\begin{claim}
\beq
\begin{array}{l}
\myboxed{
\delta(i,j) \Gamma_{adj}
=
-f_{imn}f_{jnm} = 2n\delta(i,j)}
\\ (-1)
\xymatrix{
&\ar@{~}@[green][l]|i f
&\ar@{~}@/_1pc/[l]|n
\ar@{~}@/^1pc/[l]|m
f
&\ar@{~}@[green][l]| j
}
= 2n\xymatrix{i&\ar@{~}[l]|\bullet j}
\end{array}
\eeq

\end{claim}
\proof

\beq
A=
\xymatrix{
&\ar@{~}@[green][l]|i f
&\ar@{~}@/_1pc/[l]|n
\ar@{~}@/^1pc/[l]|m
f
&\ar@{~}@[green][l]| j
}
=2
\bcen
\xymatrix@C=2pc{
&\ar@{~}[l]|i T_i
\ar[rd]
&\ar[l]T_n
&f
\ar@{~}[l]
\ar@{~}[ld]
&\ar@{~}@[green][l]|j
\\
&&T_m\ar[u]
}
\ecen
\eeq

\beq
\frac{1}{2}A=
\underbrace{\bcen
\xymatrix@C=2pc{
&&T_k\ar[ld]
\ar@{~}@/_1pc/[dd]
&
\\
&\ar@{~}[l]|i T_i
\ar[rd]
&\ar[u]T_n
&\ar@{~}@[green][l]|j
\\
&&T_m\ar[u]
}
\ecen}_{A_1}
-
\underbrace{\bcen
\xymatrix{
&&T_k\ar[ld]
&\ar@{~}@[green][l]|j
\\
&\ar@{~}[l]|i T_i
\ar[rd]
&\ar[u]T_n
\ar@{~}@/_1pc/[d]
\\
&&T_m\ar[u]
}
\ecen}_{A_2}
\eeq

\beq
A_1 =
\frac{n^2-1}{n}\delta(i,j)
\eeq

\beq
A_2 = -\frac{1}{n}\delta(i,j)
\eeq

\beq
A = 2 (A_1 - A_2)= 2n
\delta(i, j)
\eeq

\qed




\section{Differences Between $U(n)$ and $SU(n)$}

\begin{enumerate}
\item $SU(n)$ 

primitive invariants: Kronecker delta, Levi-Civita tensor


\beq
\bcen
\xymatrix@R=1pc@C=1pc{
&&&\ar@/_1pc/[ld]
\\
&T_i\ar@/_1pc/[lu]&T_i\ar@{~}[l]\ar@/_1pc/[rd]&
\\
\ar@/_1pc/[ru]&&&
}
\ecen
\eqdef P_{adj}
=
\bcen
\xymatrix@C=3pc{
&\ar[l]|\bullet
\\
\ar[r]|\bullet&}
\ecen
-
\frac{1}{n}
\bcen
\xymatrix@C=3pc{
&\ar@/_1pc/[d]|\bullet
\\
\ar@/_1pc/[u]|\bullet
&}
\ecen
\eeq

\beqa
dim(P_{adj})=\tr P_{adj} &=&
\bcen
\xymatrix@C=3pc{
&\ar[l]|\bullet\ar@[red]@{-}@/_.5pc/[l]
\\
\ar[r]|\bullet\ar@[red]@{-}@/_.5pc/[r]&
}
\ecen
-
\frac{1}{n}
\bcen
\xymatrix@C=3pc{
&\ar@/_1pc/[d]|\bullet
\ar@[red]@{-}@/_.5pc/[l]
\\
\ar@/_1pc/[u]|\bullet
&\ar@[red]@{-}@/^.5pc/[l]}
\ecen
\\
&=& n^2 -1
\eeqa

Since the Levi-Civita tensor
is an invariant matrix
for $SU(n)$,
we must have

\beq
0=
\bcen
\xymatrix@R=1pc@C=1pc{
&&\ar[ll]\ar@{~}[d]T_i
\ar[r]&
\\
\ar[r]
&\ar[r]\ar@2{-}[ddd]\cala_p
&\ar[r]T_i
&\cala^{\frac{1}{2}}_p\ar@2{-}[ddd]
\\
\ar[r]&\ar[rr]&&
\\
\ar[r]&\ar[rr]&&
\\
\ar[r]&\ar[rr]&&
}
\ecen
=
\bcen
\xymatrix@R=1pc@C=1pc{
&&\ar[ll]
\\
\ar[r]
&\ar[r]\ar@2{-}[ddd]\cala_p
&\cala^{\frac{1}{2}}_p\ar@2{-}[ddd]
\\
\ar[r]&\ar[r]&
\\
\ar[r]&\ar[r]&
\\
\ar[r]&\ar[r]&
}
\ecen
-\frac{1}{n}
\bcen
\xymatrix@R=1pc@C=1pc{
&&&
\ar`l[ld]`[d][d]
\\
\ar[r]
&\ar@2{-}[ddd]\cala_p
\ar`r[ru]`[ul][ul]
&
&\cala^{\frac{1}{2}}_p\ar@2{-}[ddd]
&
\\
\ar[r]&\ar[rr]&&
\\
\ar[r]&\ar[rr]&&
\\
\ar[r]&\ar[rr]&&
}
\ecen
\eeq


\item $U(n)$ 

primitive invariants: Kronecker delta

\beq
\bcen
\xymatrix@R=1pc@C=1pc{
&&&\ar@/_1pc/[ld]
\\
&T_i\ar@/_1pc/[lu]&T_i\ar@{~}[l]\ar@/_1pc/[rd]&
\\
\ar@/_1pc/[ru]&&&
}
\ecen
\eqdef P_{adj}
=
\bcen
\xymatrix@C=3pc{
&\ar[l]|\bullet
\\
\ar[r]|\bullet&}
\ecen
\eeq

\end{enumerate}


\beqa
dim(P_{adj})=\tr P_{adj} &=&
\bcen
\xymatrix@C=3pc{
&\ar[l]|\bullet\ar@[red]@{-}@/_.5pc/[l]
\\
\ar[r]|\bullet\ar@[red]@{-}@/_.5pc/[r]&
}
\ecen
\\
&=& n^2 
\eeqa

\section{$V_{def}\otimes V_{def}$ Decomposition}

Let 

$V_{def}=V=$ vector space 
in defining representation
$\{\ket{a}\}_{a=1}^n$.


\beq
\bcen
\xymatrix{
&\ar[l]
\\
&\ar[l]
}
\ecen
=
\bcen
\xymatrix{
&\ar[l] \cals_2
\ar@2{-}[d]
&\ar[l]
\\
&\ar[l]&\ar[l]
}
\ecen
+
\bcen
\xymatrix{
&\ar[l] \cala_2
\ar@2{-}[d]
&\ar[l]
\\
&\ar[l]&\ar[l]
}
\ecen
\eeq

\beq
\bcen
\xymatrix{
&\ar[l] \cals_2
\ar@2{-}[d]
&\ar[l]
\\
&\ar[l]&\ar[l]
}
\ecen
=
\frac{1}{2}
\left\{
\bcen
\xymatrix{
&
&\ar[ll]
\\
&&\ar[ll]
}
\ecen
+
\bcen
\xymatrix{
&\ar[l] \ar
@{<->}[d]
&\ar[l]
\\
&\ar[l]&\ar[l]
}
\ecen
\right\}
\eeq

\beq
\bcen
\xymatrix{
&\ar[l] \cala_2
\ar@2{-}[d]
&\ar[l]
\\
&\ar[l]&\ar[l]
}
\ecen
=
\frac{1}{2}
\left\{
\bcen
\xymatrix{
&
&\ar[ll]
\\
&&\ar[ll]
}
\ecen
-
\bcen
\xymatrix{
&\ar[l] \ar
@{<->}[d]
&\ar[l]
\\
&\ar[l]&\ar[l]
}
\ecen
\right\}
\eeq


\beqa
dim(\cals_2)
&=&
\frac{1}{2}
\left\{
\bcen
\xymatrix{
&
&\ar[ll]
\ar@[red]@/_1pc/@{-}[ll]
\\
&&\ar[ll]
\ar@[red]@/_1pc/@{-}[ll]
}
\ecen
+
\bcen
\xymatrix{
&\ar[l] \ar
@{<->}[d]
&\ar[l]
\ar@[red]@/_1pc/@{-}[ll]
\\
&\ar[l]&\ar[l]
\ar@[red]@/_1pc/@{-}[ll]
}
\ecen
\right\}
\\
&=&
\frac{n(n+1)}{2}
\eeqa

\beqa
dim(\cala_2)
&=&
\frac{1}{2}
\left\{
\bcen
\xymatrix{
&
&\ar[ll]
\ar@[red]@/_1pc/@{-}[ll]
\\
&&\ar[ll]
\ar@[red]@/_1pc/@{-}[ll]
}
\ecen
-
\bcen
\xymatrix{
&\ar[l] \ar
@{<->}[d]
&\ar[l]
\ar@[red]@/_1pc/@{-}[ll]
\\
&\ar[l]&\ar[l]
\ar@[red]@/_1pc/@{-}[ll]
}
\ecen
\right\}
\\
&=&
\frac{n(n-1)}{2}
\eeqa

The projection operator tree is
\begin{center}
\begin{minipage}{2cm}
\dirtree{%
.1 $\cala_2$.
.1 $\cals_2$.
}
\end{minipage}
\end{center}

\section{$V_{adj}\otimes V_{def}$
 Decomposition}
Let 

$V_{def}=V=$ vector space 
in defining representation
$\{\ket{a}\}_{a=1}^n$.

$V_{adj}=$ vector space 
in adjoint representation
$\{\ket{i}\}_{i=1}^N$.


$V_{adj}\otimes V
\cong (V\otimes V^\dagger)
\otimes V$

\beq
e=
\bcen
\xymatrix{
&&\ar@{~}[ll]
\\
&&\ar[ll]
}
\ecen
\cong
\bcen
\xymatrix{
&\ar@{~}[l]\ar@/_1pc/[r] T_i
&\ar@/_1pc/[l]T_j
&\ar@{~}[l]
\\
&&&\ar[lll]
}
\ecen
\eeq

\beq
R=
\bcen
\xymatrix{
&\ar@{~}[l] T_i
\ar@/^1pc/[ld]
&\ar@/_1pc/[l]T_j
&\ar@{~}[l]
\\
&&&\ar@/^1pc/[lu]
}
\ecen
=
\bcen
\xymatrix@R=1pc@C=1.5pc{
\ar@{~}[dr]&&&\ar@{~}[dl]
\\
&T_i\ar[ld]&T_j\ar[l]&
\\
&&&\ar[ul]
}
\ecen
\eeq

\beq
Q=
\bcen
\xymatrix{
&\ar@{~}[l] T_i
&\ar@/_1pc/@{<-}[l]T_j
\ar@/^1pc/[lld]
&\ar@{~}[l]
\\
&&&\ar@/^1pc/[llu]
}
\ecen
=
\bcen
\xymatrix{
\ar@{~}@/^1pc/[drr]
&&&\ar@{~}@/_1pc/[dll]
\\
&T_j\ar[l]&T_i\ar[l]&\ar[l]
}
\ecen
\eeq

Recall that for $SU(n)$,
the dimension $N$ of the adjoint rep is

\beq
N = n^2-1 = \xymatrix{&&\ar@{~}[ll]
\ar@{-}@[red]@/_1pc/[ll]}
\eeq
For example, for $SU(2)$, $N=3$ and for $SU(3)$,
$N=8$.

Note that

\beq
\tr(e)= 
\bcen
\xymatrix{
&&\ar@{~}[ll]
\ar@{-}@[red]@/_1pc/[ll]
\\
&&\ar[ll]
\ar@{-}@[red]@/_1pc/[ll]
}
\ecen
=Nn
\eeq

\beq
\tr(R) = 
\bcen
\xymatrix{
&\ar@{~}[l] T_i
\ar@/^1pc/[ld]
&\ar@/_1pc/[l]T_j
&\ar@{~}[l]
\ar@{-}@[red]@/^1pc/[lll]
\\
&&&\ar@/^1pc/[lu]
\ar@{-}@[red]@/^1pc/[lll]
}
\ecen
=N
\eeq

\beq
\tr(Q)=
\bcen
\xymatrix{
\ar@{~}@/^1pc/[drr]
&&&\ar@{~}@/_1pc/[dll]
\ar@{-}@[red]@/^1pc/[lll]
\\
&T_j\ar[l]&T_i\ar[l]&\ar[l]
\ar@{-}@[red]@/^1pc/[lll]
}
\ecen
=
N
\eeq


\begin{claim}

\beq
R^2 = \frac{n^2-1}{n}R
\eeq

\beq
QR = RQ= -\;\frac{1}{n}R
\eeq

\beq
Q^2 -e= - \;\frac{1}{n}R
\eeq
\end{claim}
\proof

\beqa
R^2&=&
\bcen
\xymatrix{
&&&&&\ar@{~}[dl]
\\
&\ar[dl]\ar@{~}[ul]T_i
&\ar[l]T_k&\ar@/^1.5pc/[l]T_k\ar@{~}@/_1.5pc/[l]
&\ar[l]T_j&
\\
&&&&&\ar[ul]
}
\ecen
\\
&=& \frac{n^2-1}{n}R
\quad\text{(by Eq.(\ref{eq-wavy-arc}))}
\eeqa

\beqa
QR &=&
\bcen
\xymatrix{
\ar@{~}@/^1pc/[drr]
&&&&&\ar@{~}[dl]
\\
&T_k\ar[l]
&T_i\ar[l]
&\ar[l]\ar@{~}@/_2pc/[ll]T_k
&\ar[l]T_j
&\ar[l]
}
\ecen
\eeqa

Define
\beq
X=
\bcen
\xymatrix{
\ar@{~}@/^1pc/[drr]
&&&&
\\
&T_k\ar[l]
&T_i\ar[l]
&\ar[l]\ar@{~}@/_2pc/[ll]T_k
&\ar[l]
}
\ecen
\eeq

\beqa
X
&=&
\bcen
\xymatrix@R=1pc@C=1pc{
\ar@{~}@/^2pc/[rrdddd]
&
&
&
&
\\
&
&
&
&\ar@/_1pc/[ld]
\\
&T_k\ar@/_1pc/[lu]
&
&T_k\ar@{~}[ll]\ar@/_1pc/[rd]
&
\\
\ar@/_1pc/[ru]
&
&
&
&\ar[dll]
\\
&&\ar[ull]T_i&&
}
\ecen
\\
&=&
\underbrace{\bcen
\xymatrix@R=1pc@C=1pc{
\ar@{~}@/^2pc/[rrddd]
&
&
&
&
\\
&
&
&
&\ar[llll]
\\
\ar[rrrr]
&
&
&
&\ar[dll]
\\
&&\ar[ull]T_i&&
}
\ecen}_{=0}
-
\frac{1}{n}
\bcen
\xymatrix@R=1pc@C=1pc{
\ar@{~}@/^2pc/[rrddd]
&
&
&
&
\\
&
&
&
&\ar@/_1pc/[d]
\\
\ar@/_1pc/[u]
&
&
&
&\ar[dll]
\\
&&\ar[ull]T_i&&
}
\ecen
\\
&=&-\frac{1}{n}
\bcen
\xymatrix{
\ar@/^1.5pc/@{~}[dr]&&
\\
&T_i\ar[l]
&\ar[l]
}
\ecen
\eeqa
so
\beq
QR=RQ=-\frac{1}{n}R
\eeq

\beqa
Q^2 &=&
\bcen
\xymatrix{
\ar@{~}@/^1pc/[drr]
&&&&&\ar@{~}@/_1pc/[dll]
\\
&T_k\ar[l]
&T_i\ar[l]
&\ar[l]T_j
&\ar[l]T_k\ar@{~}@/_2pc/[lll]
&\ar[l]
}
\ecen
\\
&=&
\bcen
\xymatrix@R=1pc@C=1pc{
\ar@{~}@/^2pc/[rrdddd]
&
&
&
&
&\ar@{~}@/_2pc/[ddddll]
\\
&
&
&
&
&\ar@/_1pc/[ld]
\\
&T_k\ar@/_1pc/[lu]
&
&
&T_k\ar@{~}[lll]\ar@/_1pc/[rd]
&
\\
\ar@/_1pc/[ru]
&
&
&
&
&\ar[dll]
\\
&
&\ar[ull]T_i
&T_j\ar[l]
&
&
}
\ecen
\\
&=&
\bcen
\xymatrix@R=1pc@C=1pc{
\ar@{~}@/^2pc/[rrddd]
&
&
&
&
&\ar@{~}@/_2pc/[dddll]
\\
&
&
&
&
&
\ar[lllll]
\\
\ar[rrrrr]
&
&
&
&
&\ar[dll]
\\
&
&\ar[ull]T_i
&T_j\ar[l]
&
&
}
\ecen
-\frac{1}{n}
\bcen
\xymatrix@R=1pc@C=1pc{
\ar@{~}@/^2pc/[rrddd]
&
&
&
&
&\ar@{~}@/_2pc/[dddll]
\\
&
&
&
&
&\ar@/_1pc/[d]
\\
\ar@/_1pc/[u]
&
&
&
&
&\ar[dll]
\\
&
&\ar[ull]T_i
&T_j\ar[l]
&
&
}
\ecen
\\
&=&
\bcen
\xymatrix{
&\ar@{~}[l]
\ar@/_1pc/@{<-}[r] T_i
&\ar@/_1pc/@{<-}[l]T_j
&\ar@{~}[l]
\\
&&&\ar[lll]
}
\ecen
-\frac{1}{n}
\bcen
\xymatrix@R=1pc@C=1.5pc{
\ar@{~}[dr]&&&\ar@{~}[dl]
\\
&T_i\ar[ld]&T_j\ar[l]&
\\
&&&\ar[ul]
}
\ecen
\\
&=&
e - \frac{1}{n}R
\eeqa

\qed


\begin{claim}

\beqa
P_1 &=& \frac{n}{n^2-1}R
\\
P_2&=& \frac{1}{2}P_4(1+Q)=
\frac{1}{2}
\left[e + Q -\frac{1}{n+1}R\right]
\\
P_3&=& \frac{1}{2}P_4(1-Q)=
\frac{1}{2}
\left[e - Q -\frac{1}{n-1}R\right]
\\
P_4 &=& 1-P_1
\eeqa
are projectors
for $SU(n)$. The $V_{adj}\otimes V
= \sum_\lam V_\lam$ Clebsch-Gordan series
is given by

\beq
\begin{array}{ccccccc}
\overbrace{V_{adj} \otimes V}^{\calv}&=&
P_1\calv &\oplus& P_2\calv &\oplus& P_3\calv
\\
\begin{ytableau}
\;&\;
\\
\;
\\
\none[\vdots]
\\
\;
\end{ytableau}
\otimes
\ydiagram{1}
&=&
\ydiagram{1}
&\oplus&
\begin{ytableau}
\;&\;&\;
\\
\;
\\
\none[\vdots]
\\
\;
\end{ytableau}
&\oplus&
\begin{ytableau}
\;&\;
\\
\;&\;
\\
\none[\vdots]
\\
\;
\end{ytableau}
\\
\\
(n^2-1)n 
&=&
n
&+&
\frac{n(n-1)(n+2)}{2}
&+&
\frac{n(n+1)(n-2)}{2}
\\
SU(3): 8(3)
&=&
3
&+&
15
&+&
6
\end{array}
\eeq
The projection operator  tree is
\begin{center}
\begin{minipage}{2cm}
\dirtree{%
.1 $P_1$.
.1 $P_4$.
.2 $P_2$.
.2 $P_3$.
}
\end{minipage}
\end{center}
\end{claim}
\proof

\beq
\tr(P_1)=
\frac{n}{n^2-1}N=n
\eeq

\beqa
\tr(P_2)&=&
\frac{N}{2}\left(n+ 1-\frac{1}{n+1}
\right)
\\
&=&
\frac{N}{2}\frac{n^2+2n}{n+1}
\\
&=&
\frac{N}{2}
\frac{n(n+2)}{n+1}
\\
&=&
\frac{(n-1)n(n+2)}{2}
\eeqa

\beqa
\tr(P_3)&=&
\frac{N}{2}\left(n- 1-\frac{1}{n-1}
\right)
\\
&=&
\frac{N}{2}\frac{n^2-2n}{n-1}
\\
&=&
\frac{N}{2}\frac{n(n-2)}{n-1}
\\
&=&
\frac{(n+1)n(n-2)}{2}
\eeqa
 
From $R^2 = \frac{n^2-1}{n}R$,

\beq
P_1 = \frac{n}{n^2-1}R
\eeq
Define

\beq
P_4 = e-P_1
\eeq
From $Q^2-e=-\frac{1}{n}R$, we get

\beq
P_4(Q^2-1)= 0
\eeq

Let

\beq
P_2 = \frac{1}{2}P_4(1+Q),\quad
P_3 = \frac{1}{2}P_4(1-Q)
\eeq
and

\beq
a=\frac{n}{n^2-1}
\eeq
Then

\beqa
P_2 &=& \frac{1}{2}P_4(1+Q)
\\
&=&\frac{1}{2}(e-aR)(1+Q)
\\
&=&
\frac{1}{2}(e-aR+Q-aRQ)
\\
&=&
\frac{1}{2}\left(e+ \left(\frac{1}{n}-1\right)aR+Q\right)
\quad\text{(use $QR= -\;\frac{1}{n}R$)}
\eeqa
where

\beqa
\left(\frac{1}{n}-1\right)a
&=&
\frac{1-n}{n}\frac{n}{n^2-1}
\\
&=&
-\frac{1}{n+1}
\eeqa

Furthermore

\beqa
P_3&=&
\frac{1}{2}P_4(1-Q)
\\
&=&
\frac{1}{2}(e-aR)(1-Q)
\\
&=&
\frac{1}{2}(e-aR-Q+aRQ)
\\
&=&
\frac{1}{2}\left(e- \left(\frac{1}{n}
+1\right)aR-Q\right)
\quad\text{(use $QR= -\;\frac{1}{n}R$)}
\eeqa
where

\beq\left(\frac{1}{n}+1\right)a =\frac{1}{n-1}
\eeq

\qed


Let $Q_1, Q_2, Q_3=e, R, Q$

\beq
Q_\lam\ket{Q_j}=\ket{Q_\lam Q_j}=
\sum_iA^\lam_{ij}\ket{Q_i}
\eeq

\beq
\av{Q_i|Q_\lam|Q_j}=
A^\lam_{ij}
\eeq
If $A^\lam$ are diagonalized  and divided by their eigenvalues, and they have a single non-zero eigenvalue, then they become
a complete set of projectors
with 1 or 0 along their diagonals.



