\chapter{Reducibility}
\label{ch-reducibility}

$M\in \CC^{d\times d }$

\beq
M\ket{v}=\lam \ket{v}
\eeq
If $M$ is Hermitian ($H^\dagger=H$), its eigenvalues are real. ( $\lam =
\av{\lam|M\lam}\in\RR$)


\beq
cp(\lam)\eqdef \det(M-\lam)=0
\eeq

If $M$ is a Hermitian  matrix, then there exists
a unitary matric ($CC^\dagger = C^\dagger C =1$)
such that

\beq
CMC^\dagger=
\left[
\begin{array}{cccc}
D_{\lam_1}
&0
&0
&0
\\
0
&D_{\lam_2}
&0
&0
\\
0
&0
&\ddots
&0
\\
0
&0
&0
&D_{\lam_r}
\end{array}
\right]
\eeq
where

\beq
D_{\lam_i} =
\text{diag}\underbrace{(\lam_i,\lam_i, \dots,\lam_i)}_{d_i\text{ times}}
\eeq

\beq
d=\sum_{i=1}^r d_i
\eeq


\beq
CMC^\dagger =
\left[
\begin{array}{cc}
\lam_1 &0
\\
0&\lam_2
\end{array}
\right]
\eeq

\beq
C P_1 C^\dagger=
\left[
\begin{array}{cc}
1&0
\\
0&0
\end{array}
\right]
=
\frac{CMC^\dagger -\lam_2}{\lam_1-\lam_2}
\eeq

\beq
CP_2 C^\dagger =
\left[
\begin{array}{cc}
0&0
\\
0&1
\end{array}
\right]
=
\frac{CMC^\dagger-\lam_1}{\lam_2-\lam_1}
\eeq

If $I^{d_i\times d_i}$
is the $d_i$
dimensional unit matrix,
\beqa
P_i &=&
C^\dagger
diag(0,\ldots,0, I^{d_i\times d_i},0, \dots, 0)C
\\
&=&
\prod_{j\neq i}
\frac{M -\lam_j}{\lam_i -\lam_j}
\eeqa

Note that $P_i$ are Hermitian
($P_i^\dagger = P_i$)
because $M$
is Hermitian and
its eigenvalues are real.)

Note that
$P_i$ and $M$
commute

\beq
[P_i, M]=
P_iM-MP_i=0
\eeq

orthogonal
\beq
P_i P_j =\delta(i,j)P_j
\eeq

complete
\beq
\sum_i P_i =1
\eeq

\beq
M= \sum_{i=1}^r
P_iM P_i
\eeq

\beq
d_i = \tr P_i
\eeq

\beqa
CMP_1C^\dagger &=&
\left[
\begin{array}{cc}
\lam_1&0
\\
0&\lam_2
\end{array}
\right] 
\left[
\begin{array}{cc}
1&0
\\
0&0
\end{array}
\right] 
\\
&=&
\lam_1
\left[
\begin{array}{cc}
1&0
\\
0&0
\end{array}
\right] 
\eeqa

\beq
MP_i = \lam_i P_i \;
\text{(no $i$ sum)}
\eeq

\beq
f(M) P_i = f(\lam_i)P_i \;
\text{(no $i$ sum)}
\eeq

$M^{(1)}, M^{(2)}$

\beq
[M^{(1)}, M^{(2)}]  =0
\eeq
Use $M^{(1)}$ to decompose $V$
into $\bigoplus_i V_i$.
Use  $M^{(2)}$ to decompose $V_i$ into
$\bigoplus_j V_{i,j}$. 
If $M^{(1)}$ and $M^{(2)}$ don't
commute, let $P^{(1)}_i$ be the eigenvalue 
projection operators of $M^{(1)}$. The replace $M^{(2)}$ by $P^{(1)}_i M^{(2)}P_i^{(1)}$

\beq
[M^{(1)}, P^{(1)}_iM^{(2)}P^{(1)}_i]  =0
\eeq

An invariant matrix (see Ch.\ref{ch-invariants}) commutes with 
all the elements $G$ of a group $\calg$

\beq
[G, M] =0
\eeq
for all $G\in\calg$.
If $P_i$ are 
the projection operators of $M$, then $P_i=f_i(M)$ so

\beq
[G, P_i]=0
\eeq
for all $G$ and $i$.
Hence,

\beq
G = 1G1 =\sum_i\sum_j P_i G P_j
=
\sum_i \underbrace{P_i G P_i}_
{\eqdef G_i}
\eeq

\beq
G = C^\dagger 
diag(G_1, G_2, \ldots)C
=
\sum_i
C^\dagger_i
G_i C_i
\eeq
where the matrices $C_i$
are the Clebsch Gordan 
coefficients (see Ch. \ref{ch-clebsch-gordan}).

A {\bf representation (rep)} $G_i$ acts only
on a $d_i$ dimensional vector space $V^{d_i}=P_i V^d$.
In this way, an invariant
matrix $M\in \CC^{d\times d}$
with $r$ 
distinct eigenvalues,
induces a decomposition of $V^d$
into a direct sum of vector spaces

\beq
V^d\xymatrix{\ar[r]_M&}
V_1^{d_1}
\oplus 
V_2^{d_2}
\oplus
\ldots
\oplus 
V_r^{d_r}
\eeq
If a representation $G_i$ cannot itself be
reduced further, it is said to 
be an {\bf irreducible representation (irrep)}.

Note that sometimes the term representation
is used to refer to the 
vector space $V_i^{d_i}$
instead of the matrix $G_i$.