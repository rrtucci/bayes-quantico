\chapter{Branching Rules and Symmetry Breaking}
\label{ch-branching}

\section{Block diagonalization}
\begin{figure}[h!]
\centering
\includegraphics[width=3in]
{branching/cgs-irreps.png}
\caption{Pictorial representation
of 
Clebsch-Gordan series}
\label{fig-cgs-irreps}
\end{figure}

For any square matrices $A$, $B$, $C$ (not necessarily of the same dimension)
\beq
A\oplus B\oplus C = \begin{pmatrix}
A&0&0
\\0&B&0
\\
0&0&C
\end{pmatrix}
\eeq
If $A_i$ for $i=1,2, \ldots, n$ are square matrices, $A_1\oplus A_2 \oplus \ldots \oplus A_n$ is called
a {\bf block diagonal matrix}.


An intuitive way of
discussing branching
rules is via block diagonalization (BD)
of matrices.  

Before discussing branching rules via BD,
it is a good idea to discuss
Clebsch-Gordan series via BD.
When we
discuss later on the branching rules via BD, we shall
point out the difference between the two.
The two are different 
applications of BD that are often confused.



Clebsch-Gordan series (see Chapter \ref{ch-clebsch-gordan}) via BD is
illustrated in Fig.\ref{fig-cgs-irreps}.
The figure shows a chain of BD steps.
The first BD step is

\beq
\rho(g)\xymatrix{\ar[r]_{BD}&}\rho_1(g)
\oplus \rho_2(g) \oplus \ldots \oplus \rho_n(g)
\eeq
where  
$\rho(), \rho_i()$ are reps of $G$.
The second BD step, each $\rho_i(g)$ 
is block diagonalized to produce

\beq
\bigoplus_{i=1}^n \rho_i(g)
\xymatrix{\ar[r]_{BD}&}
\bigoplus_{i=1}^n \bigoplus_{j=1}^m\rho_{i,j}(g)
\eeq
where  
the $\rho_{ij}()$ are reps of $G$.
In general, we can have any number of BD steps,
not just 2.

Usually, the beginning of the chain 
of BDs
is 

\beq
\rho(g)=\rho'(a)\otimes\rho''(b)
\eeq
and at the end  of chain,
the reps $\rho_{i,j, \ldots}(g)$ are all
irreducible.




\begin{figure}[h!]
\centering
\includegraphics[width=3in]
{branching/branching-irreps.png}
\caption{Pictorial representation
of Branching Rules.}
\label{fig-branching-irreps}
\end{figure}

Given a map $f:X\rarrow Y$, $x\mapsto f(x)$, 
a restriction $f\downarrow  X_1$ of the map to $X_1\subset X$ is the
same assignment rule but the domain restricted
to $X_1$. Hence, $f\downarrow  X_1: X_1\rarrow Y$, $x\mapsto 
f(x)$.


Branching rules via BD 
are illustrated in Fig.\ref{fig-branching-irreps}.
Assume $G, H_1, H_2$ are
groups and $G\supset H_1\supset H_2$
The first BD step takes
\beq
\{\rho(g)\mid g\in G\}\xymatrix{\ar[r]_{BD}&}\{\rho_1(h_1)
\oplus \rho_2(h_1) \oplus \ldots \oplus \rho_n(h_1)\mid h_1\in H_1\}
\eeq
where $\rho()$ is an irrep of $G$
and $\rho_i()$ for $i=1,2, \ldots, n$
are irreps of  $H_1$.
$\rho_i()$ can be obtained by
block diagonalizing the restriction
$\rho\downarrow  H_1(h_1)$ for each $h_1\in H_1$.

In the second BD step,

\beq
\left\{\bigoplus_{i=1}^n \rho_i(h_1)\mid h_1\in H_1\right\}
\xymatrix{\ar[r]_{BD}&}
\left\{\bigoplus_{i=1}^n \bigoplus_{j=1}^m\rho_{i,j}(h_2)\mid  h\in H_2\right\}
\eeq
where the 
$\rho_{ij}()$ are irreps of $H_2$.
$\rho_{ij}()$ can be obtained by
block diagonalizing the restriction
$\rho\downarrow  H_2(h_2)$ for each $h_2\in H_2$.
In general, we can have any number of BD steps,
not just 2.

Note that both
Clebsch-Gordan series and 
branching rules can both
be described in terms of vector spaces,
by the following equation, 
but the meaning of
the vector spaces is different

\beq
V\xymatrix{\ar[r]_{BD}&}\bigoplus_{i=1}^n V_i
\xymatrix{\ar[r]_{BD}&}
\bigoplus_{i=1}^n \bigoplus_{j=1}^m V_{ij}
\label{eq-bd-vectors}
\eeq

Now that we  have clarified the
difference between Clebsch-Gordan series and branching rules, let's say more
about the topic of this chapter, branching rules.

Define a
 {\bf tree of irreps (TOI)} as a tree wherein each node represents an irrep of $H_j$ and arrows point
from an irrep of $H_j$ to a direct sum of irreps 
of $H_{j+1}$ where $H_j\supset H_{j+1}$
and $H_j,  H_{j+1}$ are subgroups of $G$. 

As the tree branches out more and more, we get

\begin{itemize} 

\item less symmetry, 
fewer conserved quantities,
fewer unbroken generators, smaller multiplets, less degeneracy
\item more broken generators (get one Goldstone boson per broken generator if spontaneous 
symmetry breaking)
\end{itemize}




{\color{red}If
the root node irrep of a TOI is the adjoint representation of $G$, then the
dimension of each irrep in the tree is also the number of generators for that irrep.}
Hence, we will call
the TOI for the
adjoint rep of $G$, a {\bf tree of generators (TOG)}. For 
example, the octet ($\ul{8}$) irrep of $SU(3)$
is the adjoint irrep of $SU(3)$,
and $SU(3)$ has 8 generators.
If we draw  the 
weight diagram\footnote{Weight diagrams are
discussed in Chapter \ref{ch-weight-diagrams}} for $\ul{8}\mid _{SU(3)}$,
it contains 6 weights
of multiplicity 1 and one 
weight of multiplicity 2. Hence, if a weight of multiplicity $m$ is 
thought  of as $m$ dots, 
then {\color{red}there is a 1--1
correspondence
between the  dots of 
a weight diagram of 
the adjoint rep of $G$, and the generators of $G$}.
\section{Symmetry Breaking}
Explicit Symmetry breaking


Spontaneous symmetry breaking

$\{Q_u\mid u=1,2, \ldots, n_u\}=$ unbroken  generators $Q_u\ket{vac_0}=0$ for all $u$. If $h=e^{i\sum_u r_u Q_u}$ where $r_u\in \RR$, then $h\ket{vac_0}=\ket{vac_0}$


$\{X_b\mid b=1,2, \ldots n_b\}=$ broken generators. $X_b \ket{vac_0} \neq 0$ for all $b$.

$\ger{h}=$ stabilizer subalgebra. Algebra spanned by the $Q_u$

$\ger{g}/\ger{h}=$ orbit algebra. Algebra spanned by the $X_b$.


$\pi_a\in \RR$, coordinates of
Goldstone boson manifold

$\ket{vac}=e^{i\sum_b \pi_b X_b}\ket{vac_0}$= vacuum state, $n_b$ dimensional manifold.

$K.E.=$ Kinetic Energy

$V= $ Potential Energy

$\calh= K.E. + V$ Hamiltonian (total energy)

\beq
\left\{
\begin{array}{l}
\text{for all }u,\quad
[\calh, Q_u]=0, \quad Q_u\ket{vac_0}=0
\\
\text{for all } b,\quad [\calh, X_b]= 0, \quad X_b\ket{vac_0}\neq 0
\end{array}\right.
\eeq

\beq
[\calh, \ger{g}]=0, \quad\ger{h}\ket{vac_0}=0,
\quad \ger{g}/\ger{h}\ket{vac_0}\neq 0
\eeq

Example (Higgs Mechanism)
\beq
\calh=K.E. + \underbrace{V(\phi)}_{\lam(|\phi|^2-v^2)^2}
\eeq
where $\lam, v>0$, $\phi\in \CC$.

\begin{figure}[h!]
\centering
\includegraphics[width=3.5in]
{branching/mex-hat.png}
\caption{Mexican Hat potential.}
\label{fig-mex-hat}
\end{figure}


\section{Examples}
gen=generator, \redmark=broken gen, \greenmark=unbroken  gen,
GB= Goldstone Boson (massive particle, massive gauge boson)
\begin{enumerate}
\hrule
\item $SU(3)$ octet
(Spontaneous Symmetry Breaking)

\beq
\left\{
\begin{array}{lll}
SU(3)\supset 
&SU(2)\times U(1)\supset 
&U(1)
\\
8 \text{ gen}
& 3+ 1 \text{ gens}
& 1 \text{ gen}
\end{array}
\right.
\eeq

\begin{minipage}{10cm}
\dirtree{%
.1 $\ul{8}$, 8 gens \greenmark.
.2 $\ul{2}$, 2 gens \redmark, GB.
.3 $\ul{1}$ 1 gen \redmark.
.3 $\ul{1}$ 1 gen \redmark.
.2 $\ul{3}$, 3  gens \greenmark.
.3 $\ul{1}$, 1 gen \redmark.
.3 $\ul{1}$, 1 gen \redmark.
.3 $\ul{1}$, 1 gen \redmark.
.2 $\ul{1}$, 1 gen \greenmark.
.2 $\ul{2}$, 2 gen \redmark, GB.
.3 $\ul{1}$, 1 gen \redmark.
.3 $\ul{1}$, 1 gen \redmark.
}
\end{minipage}

\hrule
\item Zeeman splitting without electron spin (Explicit Symmetry Breaking)

\beq
\left\{
\begin{array}{lll}
SU(2)\supset 
&SO(2)
\\
3 \text{ gen}
& 1 \text{ gen}
\end{array}
\right.
\eeq

\begin{minipage}{10cm}
\dirtree{%
.1 $\ul{3}$, 3 gens \greenmark.
.2 $\ul{1}$, 1 gen  \greenmark.
.2 $\ul{1}$, 1 gen \redmark.
.2 $\ul{1}$, 1 gen \redmark.
}
\end{minipage}

\hrule
\item Zeeman splitting with electron spin (Explicit Symmetry Breaking)

$L_z=\ell_z\in\{-\ell, -\ell+1,\ldots, \ell-1, \ell\}=\ZZ_{[-\ell, \ell]}$

$S_z=s_z\in \{-\frac{1}{2}, +\frac{1}{2}\}$

$J_z=j_z=\ell_z + s_z\in 
\ZZ_{[-\ell, \ell]} \pm \frac{1}{2}$

\beq
\left\{
\begin{array}{lll}
SU(2)_{\text{orbital $\ell$} }\times SU(2)_{\text{spin $1/2$} }\supset 
&U(1)_{\text{orbital $\ell$} }\times SU(2)_{\text{spin $1/2$} }
\\
3+ 3 \text{ gen}
& 1+ 3 \text{ gen}
\end{array}
\right.
\eeq

\begin{minipage}{10cm}
\dirtree{%
.1 $L=\ell, S=\frac{1}{2}$, 6 gens \greenmark.
.2 $S=\frac{1}{2}$, 2 gen  \greenmark.
.2 $S=\frac{1}{2}$, 2 gen \redmark.
.2 $S=\frac{1}{2}$, 2 gen \redmark.
.2 \ldots.
}
\end{minipage}

\hrule
\item Electroweak symmetry breaking (Spontaneous Symmetry Breaking)

\beq
\left\{
\begin{array}{lll}
SU(2)_Y\times U(1)_Y\supset 
&U(1)_{em}
\\
3+1\text{ gens}
& 1 \text{ gen}
\end{array}
\right.
\eeq

\begin{minipage}{10cm}
\dirtree{%
.1 $\ul{3}\times \ul{1}$, 3+1 gen \greenmark.
.2 $\ul{1}$, 1 gen \greenmark, massless photon $A_\mu$ .
.2 $\ul{1}$, 1 gen \redmark, GB  $W^+$.
.2 $\ul{1}$, 1 gen \redmark, GB  $W^-$.
.2 $\ul{1}$, 1 gen \redmark, GB  $Z^0$.
}
\end{minipage}
\end{enumerate}