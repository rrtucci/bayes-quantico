\chapter{Orthogonal Groups}
\label{ch-ortho-groups}
This chapter is based on Cvitanovic's Birdtracks book Ref. \cite{birdtracks-book}.

$O(n)$ and $SO(n)$ are
defined as
the following groups of real matrices:

\beq
O(n)= \{G\in \RR^{n\times n}: G^T G= 1\}
\eeq

\beq
SO(n)= \{G\in O(n): \det G=1\}
\eeq

$O(n)$ contains orthogonal matrices $G$
($G^TG=GG^T=1$) with $\det G\in Z_2 = \{-1, 1\}$.
$SO(n)$ only contains those with $\det G= 1$.
$O(n)\cong Z_2 \times SO(n)$ where
$Z_2=\{-1, 1\}$ corresponds to the sign
of $\det G$.
Hence, $O(n)$ is a {\bf double cover} of $SO(n)$. $O(n)$
consists of 2 {\bf connected components} (CC),
whereas $SO(n)$ has only one CC.

An example of a $G\in O(n)$
that is not in $SO(n)$ is a reflection $Gx = -x$ for odd $n$.
Another example for $O(2)$ is

\beq
\text{rotation:}\quad 
G=\left(
\begin{array}{cc}
\cos\theta & \sin\theta
\\
-\sin\theta & \cos\theta
\end{array}
\right), \quad \det G =1
\eeq

\beq
\text{improper rotation:}\quad 
G=\left(
\begin{array}{cc}
\cos\theta & \sin\theta
\\
\sin\theta & -\cos\theta
\end{array}
\right), \quad \det G =-1
\eeq

The irreps of $SO(n)$
depend on whether $n$ is even or odd,
and whether the rep is spinor or non-spinor.
This chapter deals with {\bf non-spinor reps} (with either even or odd $n$.).
Chapter \ref{ch-spinors}
deals with the {\bf spinor reps}.

We assume that the {\bf metric tensor} $g_{\mu\nu}$
is a primitive invariant
that satisfies:

\beq
g_{\mu\nu}=g_{\nu\mu}=[g]_{\mu\nu},\quad
g^{\mu\nu}=g^{\nu\mu}=[g]_{\mu\nu},\quad
g\indices{_\mu^\nu}=g\indices{_\nu^\mu} =\delta_\mu^\nu
\eeq

\beq
g_{\nu\mu}x^\mu=x_\nu,\quad g^{\rho\nu}x_\nu=x^\rho\quad (\text{so }
g_{\nu\mu}g^{\rho\nu}=\delta^\rho_\mu)
\eeq
where $\mu, \nu, \rho\in \{1,2, \ldots, n\}$
and $x_\mu$ is any tensor.

In this section, we will
call {\bf orthogonal groups}
the group of
matrices
under which the following  symmetric quadratic form
is invariant

\beq
h(x)=
g_{\mu \nu}x^\mu x^\nu
\eeq
where $
\mu,\nu\in\{1, \ldots, n\}$.
Thus 

\beq
h(Gx)= h(x)
\eeq

\beq
 g_{\mu\nu}G\indices{^\mu_\alpha}
G\indices{^\nu_\beta}x^\alpha x^\beta =
g_{\alpha \beta}x^\alpha x^\beta
\implies 
g_{\mu\nu}G\indices{^\mu_\alpha}
G\indices{^\nu_\beta} = g_{\alpha \beta}
\implies
G^Tg G= g
\eeq

This condition guarantees that $G\in O(n)$ is orthogonal
for $g_{\mu\nu}=\delta_\mu^\nu$ but not that $\det G=1$.
When $g_{\mu\nu}$
is not the Kronecker delta function,
we get a  different group. For
example, if $g_{\mu\nu}=diag (1, -1, -1, -1)$ and $\det G=1$,
we get the {\bf Lorentz group} $SO(1,3)$
used in Special Relativity.


In this chapter (and in this book),
we will point the arrows in a birdtrack so that the birdtrack is a DAG. Cycles that make the birdtrack not acyclic will
have a segment in red. Without that
red segment, the birdtrack becomes  acyclic. The reason we follow this arrow convention is that it
promotes acyclic birdtracks which are more
akin to bnets. We will eschew undirected birdtracks for the same reason: bnets are directed.


Let

\beq
\myboxed{g\indices{_\mu^\nu}
 =\delta_\mu^\nu},\quad
\xymatrix{
&\ar[l]g
&\ar[l]
}=
\xymatrix{
&\ar[l]
}
\eeq

\beq
\myboxed{g\indices{^\mu_\nu}
 =\delta^\mu_\nu},\quad
\xymatrix{
\ar[r]
&\ar[r]g
&
}=
\xymatrix{
\ar[r]
&
}
\eeq



\beq
\myboxed{
g\indices{_\mu_\s} g\indices{^\s ^\nu}=\delta_\mu^\nu}
\quad
\xymatrix{
&\ul{g}
\ar[l]\ar[r]
&\ol{g}
&\ar[l]
}=
\xymatrix{&\ar[l]}
\label{eq-ol-ul-eg}
\eeq
Note that we used

\beq
\ul{g} = [g_{\mu\nu}],\quad
\ol{g} = [g^{\mu\nu}]
\eeq
We could write Eq.(\ref{eq-ol-ul-eg})
without the overline  and underline on
$g$. Those g-decorations are redundant as omitting them
would not introduce any
ambiguity. However, we will
use them  because
they make spotting errors
in the arrow directions easier.

The generators of 
orthogonal
 groups will be represented by:


\beq
(T_i)\indices{_\mu^\nu}=
\bcen
\xymatrix{
&\ar@{~}@[green][d]&&
\\
&\ar[l] T_i
&\ar[l]
}
\ecen
\eeq
We will also use

\beq
(T_i)\indices{^\mu_\nu}=
\bcen
\xymatrix{
&\ar@{~}@[green][d]&
\\
&\ar@{<-}[l] \ol{g}T_i\ul{g}
&\ar@{<-}[l]
}
\ecen
(T_i)\indices{_\mu_\nu}=
\bcen
\xymatrix{
&\ar@{~}@[green][d]&
\\
& T_i\ul{g}\ar[l]\ar[r]
&
}
\ecen
(T_i)\indices{^\mu^\nu}=
\bcen
\xymatrix{
&\ar@{~}@[green][d]&
\\
&\ar@{<-}[l] \ol{g}T_i
&\ar[l]
}
\ecen
\eeq

For $G\in O(n)$, $G^TG=1$ with
$G=e^{iT_i\eps_i}$ where $\eps_i\in\RR$.
Hence, the generators $T_i$ 
must be anti-symmetric ($T_i^T=-T_i$).

\beq
\myboxed{
(T_i)_{\mu\nu}
=-
(T_i)_{\nu\mu}
}
\bcen
\xymatrix{
&\ar@{~}[d]&
\\
\mu
&T_i\ar[l]\ar[r]
&\nu
}
\ecen
=
-
\bcen
\xymatrix{
\mu
&T_i\ar[l]\ar[r]
&\nu
\\
&\ar@{~}[u]
&
}
\ecen
\eeq

$g_\mu^\nu=\delta_\mu^\nu$
is obviously an invariant matrix. 
$g_{\mu\nu}$ must be invariant too, so

\beq
\begin{array}{l}
\myboxed{\underbrace{
(T_i)\indices{_\mu^\s}g\indices{_\s _\nu}
+
(T_i)\indices{_\nu^\s}
g\indices{_\mu_\s}=0}_
{
(T_i)_{\mu\nu}
=-
(T_i)_{\nu\mu}
}}
\\
\underbrace
{\bcen
\xymatrix{
&\ar@{~}[d]&&
\\
\mu&\ar[l] T_i
&\ar[l]
\ul{g}\ar[r]
&\nu
}
\ecen}_{(T_i)_{\mu\nu}}
+
\underbrace{\bcen
\xymatrix{
\mu&\ar[l] \ul{g}\ar[r]
&
T_i\ar[r]
&\nu
\\
&&\ar@{~}[u]
}
\ecen}_{(T_i)_{\nu\mu}}
=0
\end{array}
\label{eq-tf-is-sym-ortho}
\eeq
Hence, the invariance condition Eq.(\ref{eq-tf-is-sym-ortho}) reduces to
to the statement 
that $(T_i)_{\mu\nu}$
is antisymmetric.







The anti-symmetrizer
$\cala_2$ 
is an invariant tensor (see Section \ref{sec-invariance-sp-ap}).
Other projectors of the $V\otimes V$ are not 
invariant tensors.
Therefore, we must have

\beq
\bcen
\xymatrix@R=1pc@C=2pc{
&&&\ar@/_1pc/[dl]
\\
&T_i \ul{g}\ar@{~}[r]
\ar@/_1pc/[ul]
\ar@/^1pc/[dl]
&\ol{g}T_i
&
\\
&
&&\ar@/^1pc/[ul]
}
\ecen
=
\bcen
\xymatrix{
&\cala_2\ar@2{-}[d]
\ar[l]
&\ar[l]
\\
&\ar[l]&\ar[l]
}
\ecen
\eeq

For $SO(n)$ and $O(n)$,
the dimension $N$ of the adjoint rep 
(= number of generators) is

\beq
N = \frac{n(n-1)}{2} = \xymatrix{&&\ar@{~}[ll]
\ar@[red]@/_1pc/[ll]}
\eeq
If you take
an $n\times n$ matrix and remove
its diagonal, this $N$ is the
number of entries in the upper (or
lower) triangular sector
of the matrix. Recall that
for $U(n)$, $N=n^2$,
and for $SU(n)$, $N=n^2-1$.
So for $U(n)$ (or $SU(n)$),
there is a generator for each
entry
(or each entry minus one)
of an $n\times n$ matrix.


\begin{claim}
\beq
\begin{array}{l}
\myboxed{
\Gamma_{fun}\delta^\nu_\mu=
\sum_i
(T_iT_i)\indices{_\mu^\nu}
 = \frac{n-1}{2}\delta_\mu^\nu}
\\
\sum_i
\bcen
\xymatrix{
\mu
&T_i\ar[l]
&T_i\ar@{~}@/_2pc/[l]|i
\ar[l]
&\nu  \ar[l]
}\ecen
=
\left(\frac{n-1}{2}\right)
\xymatrix{
\mu&\ar[l]|\bullet \nu
}
\end{array}
\label{eq-wavy-arc-son}
\eeq
\end{claim}
\proof

\beqa
(T_iT_i)\indices{_\mu^\nu}
&=&
\bcen
\xymatrix{
\mu
&T_i\ul{g}\ar[l]
&\ol{g}T_i
\ar@{~}@/_2pc/[l]|i\ar@{<-}[l]
&\ar[l]\nu
}\ecen
\\
&=&
\bcen
\xymatrix@R=1pc@C=1pc{
\mu&&&\ar@/_1pc/[ld]\nu
\\
&T_i\ul{g}\ar@/_1pc/[lu]
&\ol{g}T_i\ar@{~}[l]
\ar@{<-}@/_1pc/[rd]&
\\
\ar@{<-}@/_1pc/[ru]&&&
\ar@/^1pc/[lll]
}
\ecen
\\
&&\nonumber
\\
&=&
\frac{1}{2}\left[
\bcen
\xymatrix@C=3pc{
&\ar[l]|\bullet
\\
&
\ar[l]|\bullet
\ar@{-}@/^1pc/[l]}
\ecen
-
\bcen
\xymatrix@C=3pc{
\ar@{<-}[dr]
&
\\
\ar@{<-}[ur]
&
\ar@{-}@/^1pc/[l]}
\ecen
\right]
\\
&&\nonumber
\\
&=&\left(\frac{n-1}{2}\right)
\xymatrix{
\mu&\ar[l]|\bullet \nu
}
\eeqa
\qed




\section{$V_{def}\otimes V_{def}$ Decomposition}

Let

$V_{def}=V=$ vector space 
in defining representation
$\{\ket{\mu}\}_{\mu=1}^n$.

Note that the symmetrizer
$\cals_2$ originaly
has two upper anf two
lower indices.
Its two 
upper indices can be
lowered using the metric tensor:

\beqa
(\cals_2)\indices{_{\mu\nu,\rho\s}}
&=&
g_{\rho \rho'}
g_{\s\s'}
(\cals_2)\indices{
_{\mu\nu}^{\rho'\s'}}
\\
&=&
\frac{1}{2}
\left(
g_{\mu\nu}g_{\nu\rho}+
g_{\mu,\rho}
g_{\mu\s}
\right)
\\
&=&
\bcen
\xymatrix@R=1pc@C=1pc{
&\cals_2\ar@2{-}[d]\ar[l]
&\ul{g} \ar[r]\ar[l]
&
\\
&\ar[l]
&\ul{g}\ar[r]\ar[l]
&
}
\ecen
\\
&=&
\bcen
\xymatrix@R=1pc@C=1pc{
&\cals_2\ar@2{-}[d]\ar[l]
&\ul{g} \ar[r]\ar[l]
&
\\
&\ar[l]
&\ul{g}\ar[r]\ar[l]
&
}
\ecen
\eeqa

Likewise
\beq
\bcen
\xymatrix@R=1pc@C=1pc{
&\ar[l]
\ul{g}
&\ar[l]
\\
&\ar[l]
\ul{g}
&\ar[l]
}
\ecen
=
\bcen
\xymatrix@R=1pc@C=1pc{
&\cals_2\ar@2{-}[d]\ar[l]
&\ul{g} \ar[r]\ar[l]
&
\\
&\ar[l]
&\ul{g}\ar[r]\ar[l]
&
}
\ecen
+
\bcen
\xymatrix@R=1pc@C=1pc{
&\cala_2\ar@2{-}[d]\ar[l]
&\ul{g} \ar[r]\ar[l]
&
\\
&\ar[l]
&\ul{g}\ar[r]\ar[l]
&
}
\ecen
\eeq

Define tensor $M$ by
\beq
M\indices{_\mu_\nu^\rho^\s}
=
g_{\mu\nu} g^{\rho \s}
=
\bcen
\xymatrix@R=1pc@C=1pc{
\mu&&&\rho
\\
&\ul{g}
\ar@/_1pc/[lu]
\ar@/^1pc/[ld]
&\ol{g}
\ar@{<-}@/^1pc/[ru]
\ar@{<-}@/_1pc/[rd]
&
\\
\nu&&&\s
}
\ecen
\eeq
Note that

\beq
M^2 =
\bcen
\xymatrix@R=1pc@C=1pc{
&&&
\\
&\ul{g}
\ar@/_1pc/[lu]
\ar@/^1pc/[ld]
&\ol{g}
\ar@{<-}@/^1pc/[ru]
\ar@{<-}@/_1pc/[rd]
&
\\
&&&
}
\xymatrix@R=1pc@C=1pc{
&&&
\\
&\ul{g}
\ar@/_1pc/[lu]
\ar@/^1pc/[ld]
&\ol{g}
\ar@{<-}@/^1pc/[ru]
\ar@{<-}@/_1pc/[rd]
&
\\
&&&
}
\ecen
=
n M
\eeq

Hence, $(M-n)M=0$
so $M$ has two eigenvalues $\lam=0,n$.

Next we will use the following equation from Chapter \ref{ch-reducibility2}
\footnote{Note that this equation
projects to zero all eigenvalues except one.}
to obtain a projection (PO)
operator for each eigenvalue

\beq 
P_i = \sum_{j\neq i}
\frac{M-\lam_j}{\lam_i-\lam_j}
\eeq
\begin{enumerate}
\item Singlet PO
\beq
(P_S)\indices{_\mu_\nu^\rho^\s}
=
\frac{1}{n}
\bcen
\xymatrix@R=1pc@C=1pc{
\mu&&&\rho
\\
&\ul{g}
\ar@/_1pc/[lu]
\ar@/^1pc/[ld]
&\ol{g}
\ar@{<-}@/^1pc/[ru]
\ar@{<-}@/_1pc/[rd]
&
\\
\nu&&&\s
}
\ecen
\eeq

\beqa
dim(P_S)&=&
\frac{1}{n}
\bcen
\xymatrix@R=1pc@C=1pc{
&&&\ar@[red]@/_.5pc/[lll]
\\
&\ul{g}
\ar@/_1pc/[lu]
\ar@/^1pc/[ld]
&\ol{g}
\ar@{<-}@/^1pc/[ru]
\ar@{<-}@/_1pc/[rd]
&
\\
&&&
\ar@[red]@/^.5pc/[lll]
}
\ecen
\\
&=& 1
\eeqa

\item Traceless Symmetric PO\footnote{Traceless here refers to 
$P\indices{_a^a_c^d}V\indices{_d^c}=(PV)\indices{_a^a}=0$ for any vector $V\indices{_d^c}$. It does not refer 
to
$P\indices{_a^b_b^a}=0$}



\beq
(P_{TS})\indices{_\mu_\nu^\rho^\s}
=
\bcen
\xymatrix@R=1pc@C=1pc{
&\cals_2\ar@2{-}[d]\ar[l]
&\ar[l]
\\
&\ar[l]
&\ar[l]
}
\ecen
-\frac{1}{n}
\bcen
\xymatrix@R=1pc@C=1pc{
&&&
\\
&\ul{g}
\ar@/_1pc/[lu]
\ar@/^1pc/[ld]
&\ol{g}
\ar@{<-}@/^1pc/[ru]
\ar@{<-}@/_1pc/[rd]
&
\\
&&&
}
\ecen
\eeq

\beqa
dim(P_{TS})
&=&
\bcen
\xymatrix@R=1pc@C=1pc{
&\cals_2\ar@2{-}[d]\ar[l]
&\ar[l]\ar@[red]@/_.5pc/[ll]
\\
&\ar[l]
&\ar[l]
\ar@[red]@/^.5pc/[ll]
}
\ecen
-\frac{1}{n}
\bcen
\xymatrix@R=1pc@C=1pc{
&&&\ar@[red]@/_.5pc/[lll]
\\
&\ul{g}
\ar@/_1pc/[lu]
\ar@/^1pc/[ld]
&\ol{g}
\ar@{<-}@/^1pc/[ru]
\ar@{<-}@/_1pc/[rd]
&
\\
&&&\ar@[red]@/^.5pc/[lll]
}
\ecen
\\
&=&
\frac{1}{2}(n^2+n)-1
\\
&=&
\frac{1}{2}(n+2)(n-1)
\eeqa


\item Anti-symmetric PO
\beq
(P_{A})\indices{_\mu_\nu^\rho^\s}
=
\bcen
\xymatrix@R=1pc@C=1pc{
&\cala_2\ar@2{-}[d]\ar[l]
&\ar[l]
\\
&\ar[l]
&\ar[l]
}
\ecen
\eeq

\beqa
dim(P_{A})
&=&
\bcen
\xymatrix@R=1pc@C=1pc{
&\cala_2\ar@2{-}[d]\ar[l]
&\ar[l]
\ar@[red]@/_.5pc/[ll]
\\
&\ar[l]
&\ar[l]
\ar@[red]@/^.5pc/[ll]
}
\ecen
\\
&=&
\frac{1}{2}(n^2 -n)
\\
&=&
\frac{1}{2}n(n-1)
\eeqa
\end{enumerate}


\begin{claim}
The Clebsch-Gordan series for $V\otimes V$ (i.e., decomposition of 
$V\otimes V$) is

\beq
\begin{array}{ccccccc}
\overbrace{V\otimes V}
^\calv
&=
&P_S\calv
&\oplus
&P_{TS}\calv
&\oplus
&P_A\calv
\\
\ydiagram{1}
\otimes\ydiagram{1}
&=
&\bullet
&\oplus
&\ydiagram{2}
&\oplus
&\ydiagram{1,1}
\\
n^2
&=
& 1
&+
&\frac{1}{2}(n+2)(n-1)
&+
&\frac{1}{2}n(n-1)
\end{array}
\eeq
The projection operator  tree is
\begin{center}
\begin{minipage}{2cm}
\dirtree{%
.1 $P_A$.
.1 $P_{SYM}$.
.2 $P_S$.
.2 $P_{TS}$.
}
\end{minipage}
\end{center}
where $P_{SYM}=\cals_2$.
\end{claim}
\proof
\qed

\section{$V_{adj}\otimes V_{def}$ Decomposition}

Let 

$V_{def}=V=$ vector space 
in defining representation
$\{\ket{\mu}\}_{\mu=1}^n$.

$V_{adj}=$ vector space 
in adjoint representation
$\{\ket{i}\}_{i=1}^N$.



$V_{adj}\otimes V
\cong (V\otimes V^\dagger)
\otimes V$

\beq
e=
\bcen
\xymatrix{
&&\ar@{~}[ll]
\\
&&\ar[ll]
}
\ecen
\cong
\bcen
\xymatrix{
&\ar@{~}[l]\ar@/_1pc/[r] T_i
&\ar@/_1pc/[l]T_j
&\ar@{~}[l]
\\
&&&\ar[lll]
}
\ecen
\eeq

\beq
R=
\bcen
\xymatrix{
&\ar@{~}[l] T_i
\ar@/^1pc/[ld]
&\ar@/_1pc/[l]T_j
&\ar@{~}[l]
\\
&&&\ar@/^1pc/[lu]
}
\ecen
=
\bcen
\xymatrix@R=1pc@C=1.5pc{
\ar@{~}[dr]&&&\ar@{~}[dl]
\\
&T_i\ar[ld]&T_j\ar[l]&
\\
&&&\ar[ul]
}
\ecen
\eeq

\beq
Q=
\bcen
\xymatrix{
&\ar@{~}[l] T_i
&\ar@/_1pc/[l]T_j
\ar@/^1pc/[lld]
&\ar@{~}[l]
\\
&&&\ar@/^1pc/[llu]
}
\ecen
=
\bcen
\xymatrix{
\ar@{~}@/^1pc/[drr]
&&&\ar@{~}@/_1pc/[dll]
\\
&T_j\ar[l]&T_i\ar[l]&\ar[l]
}
\ecen
\eeq

Recall that for $SO(n)$ and $O(n)$,
the dimension $N$ of the adjoint rep is

\beq
N = \frac{n(n-1)}{2} = \xymatrix{&&\ar@{~}[ll]
\ar@[red]@/_1pc/[ll]}
\eeq
For example, for $SO(3)$, $N=3$.

Note that


\beq
\tr(e)= 
\bcen
\xymatrix{
&&\ar@{~}[ll]
\ar@[red]@/_1pc/[ll]
\\
&&\ar[ll]
\ar@[red]@/_1pc/[ll]
}
\ecen
=Nn
\eeq

\beq
\tr(R) = 
\bcen
\xymatrix{
&\ar@{~}[l] T_i
\ar@/^1pc/[ld]
&\ar@/_1pc/[l]T_j
&\ar@{~}[l]
\ar@[red]@/^1pc/[lll]
\\
&&&\ar@/^1pc/[lu]
\ar@[red]@/^1pc/[lll]
}
\ecen
=N
\eeq


\beq
\tr(Q)=
\bcen
\xymatrix{
\ar@{~}@/^1pc/[drr]
&&&\ar@{~}@/_1pc/[dll]
\ar@[red]@/^1pc/[lll]
\\
&T_j\ar[l]&T_i\ar[l]&\ar[l]
\ar@[red]@/^1pc/[lll]
}
\ecen
=
N
\eeq


\begin{claim}

\beq
R^2 = \frac{n-1}{2}R
\eeq

\beq
QR = RQ= \;\frac{1}{2}R
\eeq

\beq
Q^2= \frac{e-Q}{2}
\eeq
\end{claim}
\proof

\beqa
R^2&=&
\bcen
\xymatrix{
&&&&&\ar@{~}[dl]
\\
&\ar[dl]\ar@{~}[ul]T_i
&\ar[l]T_k
&\ar@/^1.5pc/[l]T_k\ar@{~}@/_1.5pc/[l]
&\ar[l]T_j&
\\
&&&&&\ar[ul]
}
\ecen
\\
&=& \frac{n-1}{2}R
\quad\text{(by Eq.(\ref{eq-wavy-arc-son}))}
\eeqa

\beqa
QR &=&
\bcen
\xymatrix{
\ar@{~}@/^1pc/[drr]
&&&&&\ar@{~}[dl]
\\
&T_k\ar[l]
&T_i\ar[l]
&\ar[l]\ar@{~}@/_2pc/[ll]T_k
&\ar[l]T_j
&\ar[l]
}
\ecen
\eeqa

Define
\beq
X=
\bcen
\xymatrix{
\ar@{~}@/^1pc/[drr]
&&&&
\\
&T_k\ar[l]
&T_i\ar[l]
&\ar[l]\ar@{~}@/_2pc/[ll]T_k
&\ar[l]
}
\ecen
\eeq

\beqa
X
&=&
\bcen
\xymatrix@R=1pc@C=1pc{
\ar@{~}@/^2pc/[rrdddd]
&
&
&
&
\\
&
&
&
&\ar@/_1pc/[ld]
\\
&T_k\ul{g}\ar@/_1pc/[lu]
&
&\ol{g}T_k
\ar@{~}[ll]
\ar@{<-}@/_1pc/[rd]
&
\\
\ar@{<-}@/_1pc/[ru]
&
&
&
&\ar@{<-}[dll]
\\
&&\ar@{<-}[ull]
\ol{g}T_i\ul{g}&&
}
\ecen
\\
&=&
\frac{1}{2}
\left[
\underbrace{\bcen
\xymatrix@R=1pc@C=1pc{
\ar@{~}@/^2pc/[rrddd]
&
&
&
&
\\
&
&
&
&\ar[llll]
\\
&
&
&
&\ar@{<-}[dll]
\ar[llll]
\\
&&\ar@{<-}[ull]\ol{g}T_i\ul{g}
&&
}
\ecen}_{=0}
-
\bcen
\xymatrix@R=1pc@C=1pc{
\ar@{~}@/^2pc/[rrddd]
&
&
&
&
\\
\ar@{<-}[drrrr]
&
&
&
&\ar[dllll]
\\
&
&
&
&\ar@{<-}[dll]
\\
&&\ar@{<-}[ull]
\ol{g}T_i\ul{g}
&&
}
\ecen
\right]
\\
&=&\frac{1}{2}
\bcen
\xymatrix{
\ar@/^1.5pc/@{~}[dr]&&
\\
&T_i\ar[l]
&\ar[l]
}
\ecen
\eeqa
so
\beq
QR=RQ=\frac{1}{2}R
\eeq

\beqa
Q^2 &=&
\bcen
\xymatrix{
\ar@{~}@/^1pc/[drr]
&&&&&\ar@{~}@/_1pc/[dll]
\\
&T_k\ar[l]
&T_i\ar[l]
&\ar[l]T_j
&\ar[l]T_k\ar@{~}@/_2pc/[lll]
&\ar[l]
}
\ecen
\\
&=&
\bcen
\xymatrix@R=1pc@C=1pc{
\ar@{~}@/^2pc/[rrdddd]
&
&
&
&
&\ar@{~}@/_2pc/[ddddll]
\\
&
&
&
&
&\ar@/_1pc/[ld]
\\
&T_k\ul{g}
\ar@/_1pc/[lu]
&
&
&\ol{g}T_k
\ar@{~}[lll]
\ar@{<-}@/_1pc/[rd]
&
\\
\ar@{<-}@/_1pc/[ru]
&
&
&
&
&\ar@{<-}[dll]
\\
&
&\ar@{<-}[ull]
\ol{g}T_i\ul{g}
&\ol{g}T_j\ul{g}
\ar@{<-}[l]
&
&
}
\ecen
\\
&=&
\frac{1}{2}\left[
\bcen
\xymatrix@R=1pc@C=1pc{
\ar@{~}@/^2pc/[rrddd]
&
&
&
&
&\ar@{~}@/_2pc/[dddll]
\\
&
&
&
&
&
\ar[lllll]
\\
\ar@{<-}[rrrrr]
&
&
&
&
&\ar@{<-}[dll]
\\
&
&\ar@{<-}[ull]
\ol{g}T_i\ul{g}
&\ol{g}T_j\ul{g}
\ar@{<-}[l]
&
&
}
\ecen
-
\bcen
\xymatrix@R=1pc@C=1pc{
\ar@{~}@/^2pc/[rrddd]
&
&
&
&
&\ar@{~}@/_2pc/[dddll]
\\
\ar@{<-}[drrrrr]
&
&
&
&
&
\\
\ar@{<-}[urrrrr]
&
&
&
&
&\ar@{<-}[dll]
\\
&
&\ar@{<-}[ull]
\ol{g}T_i\ul{g}
&\ol{g}T_j\ul{g}
\ar@{<-}[l]
&
&
}
\ecen
\right]
\\
&=&
\frac{1}{2}
\left[
\bcen
\xymatrix{
&\ar@{~}[l]
\ar@/_1pc/@{<-}[r] \ol{g}T_i\ul{g}
&\ar@/_1pc/@{<-}[l]
\ol{g}T_j\ul{g}
&\ar@{~}[l]
\\
&&&\ar[lll]
}
\ecen
-
\bcen
\xymatrix@R=1pc@C=1.5pc{
\ar@{~}[drr]
&&&\ar@{~}[dll]
\\
&\ol{g}T_i\ul{g}
\ar[ld]
&\ol{g}T_j\ul{g}
\ar[l]&
\\
&&&\ar[ul]
}
\ecen
\right]
\\
&=&
\frac{e - Q}{2}
\eeqa

\qed


\begin{claim}

\beqa
P_1 &=& \frac{2}{n-1}R
\\
P_2&=& \frac{1}{3}P_4(1-2Q)=
\frac{1}{3}
\left[
e -2 Q
\right]
\\
P_3&=&  \frac{2}{3}P_4(1+Q)  =
\frac{2}{3}
\left[e + Q -\frac{3}{n-1}R\right]
\\
P_4 &=& e- P_1
\eeqa
are projectors
for $O(n)$ and $SO(n)$. The $V_{adj}\otimes V
= \sum_\lam V_\lam$ Clebsch-Gordan series
is given by

\beq
\begin{array}{ccccccc}
\overbrace{V_{adj} \otimes V}^{\calv}&=&
P_1\calv &\oplus& P_2\calv &\oplus& P_3\calv
\\
\ydiagram{1,1}
\otimes
\ydiagram{1}
&=&
\ydiagram{1}
&\oplus&
\ydiagram{1,1,1}
&\oplus&
\ydiagram{2,1}
\\
\\
\frac{1}{2}n^2(n-1)
&=&
n
&+&
\frac{1}{6}n(n-1)(n-2)
&+&
\frac{1}{3}n(n+2)(n-2)
\\
SO(3): 9
&=&
3
&+&
1
&+&
5
\\
SO(4): 24
&=&
4
&+&
4
&+&
16
\end{array}
\eeq
The projection operator  tree is
\begin{center}
\begin{minipage}{2cm}
\dirtree{%
.1 $P_1$.
.1 $P_4$.
.2 $P_2$.
.2 $P_3$.
}
\end{minipage}
\end{center}
\end{claim}
\proof

\beqa
\tr(P_1)&=&
\frac{2}{n-1}N
\\
&=&
\frac{1}{n-1}n(n-1)
\\
&=&
n
\eeqa

\beqa
\tr(P_2)&=&
\frac{N}{3}(n-2)
\\
&=&
\frac{n
(n-1)}{6}(n-2)
\eeqa

\beqa
\tr(P_3)&=&
\frac{2N}{3}\left(n + 1 - \frac{3} {n-1}\right)
\\
&=&
\frac{n}{3}\left(n^2 -4\right)
\eeqa
 
From $R^2 = \frac{2}{n-1}R$,

\beq
P_1 = \frac{2}{n-1}R
\eeq
Define

\beq
P_4 = e-P_1
\eeq
From $Q^2=\frac{1}{2}(1-Q)$, we get

\beq
2Q^2 +Q -1= (2Q-1)(Q+1)=0
\eeq

Let

\beq
P_2 =   \frac{1}{3}P_4(1-2Q)\quad
P_3 = \frac{2}{3}P_4(1+Q)
\eeq
and

\beq
a=\frac{2}{n-1}
\eeq
Then

\beqa
P_3 &=& \frac{2}{3}P_4(1+Q)
\\
&=&\frac{2}{3}(e-aR)(1+Q)
\\
&=&
\frac{2}{3}(e-aR+Q-aRQ)
\\
&=&
\frac{2}{3}\left(e-\frac{3}{2}aR+Q\right)
\quad\text{(use $QR= \frac{1}{2}R$)}
\\
&=&
\frac{2}{3}\left(e-\frac{3}{n-1}R+Q\right)
\eeqa

Furthermore

\beqa
P_2&=&
\frac{1}{3}P_4(1-2Q)
\\
&=&
\frac{1}{3}(e-aR)(1-2Q)
\\
&=&
\frac{1}{3}(e-aR-2Q+2aRQ)
\\
&=&
\frac{1}{3}\left(e-2Q\right)
\quad\text{(use $QR= \frac{1}{2}R$)}
\eeqa


\qed