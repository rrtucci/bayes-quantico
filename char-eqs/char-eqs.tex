\chapter{Characteristic Equations}
\label{ch-char-eqs}
This chapter is based on Cvitanovic's Birdtracks book Ref. \cite{birdtracks-book}.

Let
\beq
M\indices{_a^b}=
\xymatrix{
a&M\ar[l]&\ar[l]b
}
\eeq
for $a,b=1, 2, \ldots, n$. 

The goal of
this chapter is
to express the 
coefficients
of the characteristic 
equation (i.e., $det(M-\lam)=0$)
of $M$
as 
traces.
 
For starters, note the 
difference between 
birdtracks for
a matrix power and a tensor 
power of $M$.

\beq
M^2 = 
\xymatrix{
&\ar[l]M^2&\ar[l]
}=
\xymatrix{
&\ar[l]M&M\ar[l]&\ar[l]
}
\eeq

\beq
M\otimes M = M^{\otimes 2}
=
\bcen
\xymatrix@R=1pc{
&\ar[l]M&\ar[l]
\\
&\ar[l]M&\ar[l]
}
\ecen
\eeq

In general, $M^{\otimes p}$ is defined by
\beq
\begin{array}{l}
\myboxed{
(M^{\otimes p})\indices{
_\alp
^\beta
}
=
(M^{\otimes p})\indices{
_{a^{:p}}
^{rev(b^{^p})}
} =
M\indices{_{a_1}^{b_1}}
M\indices{_{a_2}^{b_2}}
\ldots
M\indices{_{a_p}^{b_p}}}
\\
\bcen
\xymatrix@R=1pc{
&M^{\otimes p}\ar[l]
\ar@2{-}[ddd]
&\ar[l]
\\
&\ar[l]&\ar[l]
\\
\vdots&&\vdots
\\
&\ar[l]&\ar[l]
}
\ecen
=
\bcen
\xymatrix@R=1pc{
&\ar[l]M&\ar[l]
\\
&\ar[l]M&\ar[l]
\\
\vdots & \vdots
\\
&\ar[l]M&\ar[l]
}
\ecen
\end{array}
\eeq
where $a_i, b_i\in \ZZ_{[1, n]}$,
and we define
the anti-symmetrized trace of $M^{\otimes p}$ by




\beqa
\tr_{1\ldots p}\cala[M^{\otimes p}]
&=&
\cala\indices{
_{a^{:p}}
^{rev(b^{:p})}
}
\prod_{i=1}^p
M\indices{
_{b_i}
^{a_i}
}
\\
&=&
\bcen
\xymatrix@R=1.5pc{
&\cala_p
\ar@2{-}[dd]
\ar@/_3pc/[ddddd]
&
\\
&\ar@/_2pc/[ddd]
&
\\
&\ar@/_1pc/[d]&
\\
&M
\ar@/_1pc/[u]&
\\
&M
\ar@/_2pc/[uuu]
&
\\
&M
\ar@/_3pc/[uuuuu]&
}
\ecen
\text{(Cvitanovic Drawing Style)}
\\
&=&
\bcen
\xymatrix@R=1.5pc{
&\cala_p
\ar@2{-}[dd]
\ar[l]
&M\ar[l]
&\ar[l]
\ar@/_1pc/@[red]@{-}[lll]
\\
&\ar[l]
&M\ar[l]
&\ar[l]
\ar@/_1pc/@[red]@{-}[lll]
\\
&\ar[l]
&M\ar[l]
&\ar[l]
\ar@/_1pc/@[red]@{-}[lll]
}
\ecen
\text{(This book's drawing style)}
\eeqa

Note that the
determinant of $M$
is one of
those traces

\beq
det M = 
\tr_{1\ldots n} \cala[ M^{\otimes n}]
\eeq


\begin{claim}
\beq
\bcen
\xymatrix@C=1pc@R=1.5pc{
&\cala_p
\ar@2{-}[dd]
\ar[l]
&
&\ar[ll]
\\
&\ar[l]
&M\ar[l]
&\ar[l]
\ar@/_1pc/@[red]@{-}[lll]
\\
&\ar[l]
&M\ar[l]
&\ar[l]
\ar@/_1pc/@[red]@{-}[lll]
}
\ecen
=
\frac{1}{p}
\left[
\bcen
\xymatrix@C=1pc@R=1.5pc{
&
&
&\ar[lll]
\\
&\ar[l]\cala_{p-1}
\ar@2{-}[d]
&M\ar[l]
&\ar[l]
\ar@/_1pc/@[red]@{-}[lll]
\\
&\ar[l]
&M\ar[l]
&\ar[l]
\ar@/_1pc/@[red]@{-}[lll]
}
\ecen
-(p-1)
\bcen
\xymatrix@C=1pc@R=1.5pc{
&\ar[l]\cala_{p-1}
\ar@2{-}[d]
&M\ar[l]
&\ar[l]
\\
&\ar[l]
&M\ar[l]
&\ar[l]
\ar@/_1pc/@[red]@{-}[lll]
}
\ecen
\right]
\eeq
\end{claim}
\proof

See Chapter \ref{ch-sym}.
\qed

Consider the above
claim for $p=2,3$.
\beq
\left\{
\begin{array}{l}
\bcen
\xymatrix@C=1pc@R=1.5pc{
&\cala_3
\ar@2{-}[dd]
\ar[l]
&
&\ar[ll]
\\
&\ar[l]
&M\ar[l]
&\ar[l]
\ar@/_1pc/@[red]@{-}[lll]
\\
&\ar[l]
&M\ar[l]
&\ar[l]
\ar@/_1pc/@[red]@{-}[lll]
}
\ecen
=
\frac{1}{3}
\left[
\bcen
\xymatrix@C=1pc@R=1.5pc{
&
&
&\ar[lll]
\\
&\ar[l]\cala_{2}
\ar@2{-}[d]
&M\ar[l]
&\ar[l]
\ar@/_1pc/@[red]@{-}[lll]
\\
&\ar[l]
&M\ar[l]
&\ar[l]
\ar@/_1pc/@[red]@{-}[lll]
}
\ecen
-2
\bcen
\xymatrix@C=1pc@R=1.5pc{
&\ar[l]\cala_{2}
\ar@2{-}[d]
&M\ar[l]
&\ar[l]
\\
&\ar[l]
&M\ar[l]
&\ar[l]
\ar@/_1pc/@[red]@{-}[lll]
}
\ecen
\right]
\\
\\
\bcen
\xymatrix@C=1pc@R=1.5pc{
&\cala_2
\ar@2{-}[d]
\ar[l]
&
&\ar[ll]
\\
&\ar[l]
&M\ar[l]
&\ar[l]
\ar@/_1pc/@[red]@{-}[lll]
}
\ecen
=
\frac{1}{2}
\left[
\bcen
\xymatrix@C=1pc@R=1.5pc{
&
&
&\ar[lll]
\\
&
&M\ar[ll]
&\ar[l]
\ar@/_1pc/@[red]@{-}[lll]
}
\ecen
-
\bcen
\xymatrix@C=1pc@R=1.5pc{
&
&M\ar[ll]
&\ar[l]
}
\ecen
\right]
\end{array}\right.
\label{eq-recursion-p-23}
\eeq
If we multiply from the right, by $M^d$
for $d=1,2$, 
the first row
of Eq.(\ref{eq-recursion-p-23}) and
then take the trace 
of that row,
we get

\beq
\left\{
\begin{array}{l}
\bcen
\xymatrix@C=1pc@R=1.5pc{
&\cala_3
\ar@2{-}[dd]
\ar[l]
&M\ar[l]
&\ar[l]\ar@/_1pc/@[red]@{-}[lll]
\\
&\ar[l]
&M\ar[l]
&\ar[l]
\ar@/_1pc/@[red]@{-}[lll]
\\
&\ar[l]
&M\ar[l]
&\ar[l]
\ar@/_1pc/@[red]@{-}[lll]
}
\ecen
=
\frac{1}{3}
\left[
\bcen
\xymatrix@C=1pc@R=1.5pc{
&
&M\ar[ll]
&\ar[l]
\ar@/_1pc/@[red]@{-}[lll]
\\
&\ar[l]\cala_{2}
\ar@2{-}[d]
&M\ar[l]
&\ar[l]
\ar@/_1pc/@[red]@{-}[lll]
\\
&\ar[l]
&M\ar[l]
&\ar[l]
\ar@/_1pc/@[red]@{-}[lll]
}
\ecen
-2
\bcen
\xymatrix@C=1pc@R=1.5pc{
&\ar[l]\cala_{2}
\ar@2{-}[d]
&M^2\ar[l]
&\ar[l]
\ar@/_1pc/@[red]@{-}[lll]
\\
&\ar[l]
&M\ar[l]
&\ar[l]
\ar@/_1pc/@[red]@{-}[lll]
}
\ecen
\right]
\\
\\
\bcen
\xymatrix@C=1pc@R=1.5pc{
&\cala_2
\ar@2{-}[d]
\ar[l]
&M^2\ar[l]
&\ar[l]
\ar@/_1pc/@[red]@{-}[lll]
\\
&\ar[l]
&M\ar[l]
&\ar[l]
\ar@/_1pc/@[red]@{-}[lll]
}
\ecen
=
\frac{1}{2}
\left[
\bcen
\xymatrix@C=1pc@R=1.5pc{
&
&M^2\ar[ll]
&\ar[l]
\ar@/_1pc/@[red]@{-}[lll]
\\
&
&M\ar[ll]
&\ar[l]
\ar@/_1pc/@[red]@{-}[lll]
}
\ecen
-
\bcen
\xymatrix@C=1pc@R=1.5pc{
&
&M^3\ar[ll]
&\ar[l]\ar@/_1pc/@[red]@{-}[lll]
}
\ecen
\right]
\end{array}\right.
\label{eq-2eqs-m3}
\eeq

Let

\beq
\tau = \tr(M)
\eeq
Then Eqs.(\ref{eq-2eqs-m3})
can be expressed
algebraically by

\beq
\tr_{1,2,3}\cala_3(M^{\otimes 3})=
\frac{1}{3}
\left[
\tau  \tr_{1,2}\cala_2(M^{\otimes 2})
-\tr(M^2)\tau 
+\tr M^3
\right]
\eeq
and

\beq
\tr_{1,2}
\cala_2 M^{\otimes 2}=
\frac{1}{2}
\left[
\tau^2-\tr(M^2)
\right]
\eeq
Therefore,

\beqa
\tr_{1,2,3}\cala_3(M^{\otimes 3})
&=&
\frac{1}{3}
\left[
\frac{1}{2}\tau^3 
-\frac{3}{2}\tr(M^2)\tau 
+\tr M^3
\right]
\\
&=&
\frac{1}{3!}
\left[
\tau^3
-3\tr(M^2)\tau 
+2\tr M^3
\right]
\eeqa
In general,

\beq
\tr_{1\ldots p}\cala_p M=
\frac{1}{p}
\sum_{k=1}^p
(-1)^{k-1}
\left(\tr_{1\ldots p-k}\cala_{p-k}M^{\otimes p-k}
\right)
\tr(M^k)
\label{eq-tr-am-expansion}
\eeq

Next note that
\beq
\cala_{p}=0
\quad \text{if $p>n$}
\eeq
This follows because the Levi Civita tensor
with more than $n$
indices is zero.; i.e., 

\beq
\eps_{a_1, a_2, \ldots, a_{n+1}} =0
\eeq
Indeed, two of the $a_i$ must be
equal, so
that element of the
$\eps$ tensor is zero
 
Let $I$ be the $n\times n$
identity matrix. Then,
since $\cala_{n+1}=0$,
the following is 
true

\beq
\begin{array}{cc}
\myboxed{0=
\tr_{2\ldots n+1}
\cala_{n+1}I\otimes M^{\otimes n}}
&\quad\quad
0=
\bcen
\xymatrix@R=1.5pc{
&\cala_{n+1}\ar[l]
\ar@2{-}[ddd]&&\ar[ll]
\\
&
\ar[l]
&M\ar[l]
&\ar[l]
\ar@/_1pc/@[red]@{-}[lll]
\\
&\ar[l]
&M\ar[l]
&\ar[l]
\ar@/_1pc/@[red]@{-}[lll]
\\
&\ar[l]
&M\ar[l]
&\ar[l]
\ar@/_1pc/@[red]@{-}[lll]
}
\ecen
\end{array}
\label{eq-a-n-plus-1}
\eeq
We can
now expand
the right hand
side of Eq.(\ref{eq-a-n-plus-1}) using
identity Eq.(\ref{eq-tr-am-expansion})

\beqa
0&=&
\sum_{k=0}^n
(-1)^k \left(
\tr_{1\ldots n-k}
\cala_{n-k}
M^{\otimes n-k}
\right)
M^k
\\
&=&
\left\{
\begin{array}{l}
M^n
\\
- M^{n-1}(\tr M)
\\
+M^{n-2}
(\tr_{1\ldots 2}
\cala_2 M^{\otimes 2})
\\
\ldots
\\
(-1)^n det(M)
\end{array}
\right.
\label{eq-char-eq-gen}
\eeqa
Viola. The last equation is none other
than the characteristic equation
of $M$. As promised,
the coefficients
of this polynomial in $M$,
are expressed as traces. 