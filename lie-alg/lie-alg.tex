\chapter{Lie Algebras}
\label{ch-lie-alg}

This chapter is based on 
Ref.\cite{birdtracks-book}.
\section{Generators of infinitesimal transformations}

For some group
$\calg$, assume that any group element $G\in\calg$
that is infinitesimal 
close to the identity
$1$ can be parametrized by


\beq
G = 1 + i \sum_i 
\eps_i T_i
\eeq
where $T_i\in \CC^{n\times n}$
for $i=1,2, \ldots, N$,
$\eps_i\in\RR $
and $|\eps_i|<<1$
\footnote{Note that the $\eps_i$ are real, not complex.}.

The $T_i$ matrices are called
the {\bf generators
of infinitesimal transformations}
for group $\calg$.
The generators of a group $\calg$ span a vector space 
called a Lie algebra $\ger{g}$.\footnote{See Sec.\ref{sec-algebra-over-f}
for the definition of an algebra over a field.
} For example,
the generators of the group SU(2) 
span the {\bf Lie algebra} $\ger{su(2)}$.

The tensor
\beq
g_{ij}=\tr(T^\dagger_i T_j)
\eeq
is called the {\bf Cartan-Killing form}. This tensor  can
be used to raise and lower the 
the adjoint rep indices $i, j, k$
in a tensor such as $M_{ijk}$:


\beq
M\indices{^{i}_{jk}}=
g^{ii'}
M\indices{_{i'jk}}
\eeq


Assume that the $T_i$ matrices are Hermitian and
that they satisfy

\beq
g_{ij}=\tr(T_i T_j)=\kappa\delta(i, j)
\label{eq-gluon-mass}
\eeq
A Lie algebra that satisfies Eq.(\ref{eq-gluon-mass})
is called a {\bf simple Lie algebra}. 

A {\bf semi-simple Lie algebra} is a direct
sum of simple Lie algebras.


It's customary to choose 
generators so that  $\kappa=\frac{1}{2}$.\footnote{For $SU(2)$,
it is customary to
choose $T^i =\frac{1}{2}\s_i$,
where $\s_i$ for $i=1,2,3$ are the Pauli matrices.
For $SU(3)$,
it is customary to choose $T^i=
\frac{1}{2}\lam_i$
where $\lam_i$
for $i=1,2, \ldots, 8$ are the Gell-Mann matrices.
For both of these choices,
$\kappa=\frac{1}{2}$.}
However, we will often set $\kappa=1$
for intermediate calculations
and restore $\kappa\neq 1$ at the end by dimensional analysis.
Just remember that each $T^j$ scales as $\sqrt{\kappa}$.
For example, given
the equation 
$\tr(T^iT^j)=\delta(i,j)$,
we know that when $\kappa\neq 1$,
$\tr(T^iT^j)=\kappa\delta(i,j)$
so both sides of the equation scale as $\kappa$.

We will use
the following
scaled version of $T^j$
as a birdtrack. Define

\beq
(C_{Adj}^i)\indices{_b^a}=
\frac{1}{\sqrt{\kappa}}
(T^i)\indices{_b^a}=
\frac{1}{\sqrt{\kappa}}
\bcen
\xymatrix{
&a\ar[d]
\\
i\ar@{~}@[green][r]
&T^i
\ar[d]
\\
&b
}
\ecen
\eeq
In the CC convention, we will always
start reading the indices 
of this node at the wavy undirected green leg.
$Adj$ stands the 
Adjoint. In this node (vertex), an adj-rep particle
(wavy line, gluon) is generated (released) by
a def-rep
particle 
(straight solid line, arrow).



In terms of
birdtracks, Eq.(\ref{eq-gluon-mass})
becomes


\beq
\begin{array}{ll}
\myboxed{
(T^i)\indices{
^b_a}
(T^j)\indices{
^a_b}=
\tr(T^iT^j)=
\delta(i,j)
}
&
\xymatrix{
i
&T^i\ar@{~}[l]
\ar@/^2pc/[r]|{\sum a}
&T^j
\ar@/^2pc/[l]|{\sum b}
&j\ar@{~}[l]
}
=
\xymatrix{&\ar[l]|\bullet}
\end{array}
\eeq

We can now define the projection operator
for the adj-rep. This  projection operator represent a
gluon exchange between 2 def-rep particles.
\beq
\myboxed
{(P_{Adj})\indices{
_b^a_d^c
}
=
\sum_i
(T^i)\indices{_b^a}
(T^i)\indices{_d^c}}
\bcen
\xymatrix@R=1pc@C=1pc{
b
&&c
\\
&P_{Adj}
\ar@{<-}[ur]
\ar@[green][ul]
\ar@{<-}[dl]
\ar[dr]
\\
a
&&d
}
\ecen
=
\bcen
\xymatrix@R=1pc@C=2pc{
b
&
&c\ar[d]
\\
T^i\ar[u]&
&\ar@{~}@[green][ll]|{\sum i} T^i\ar[d]
\\
a\ar[u]
&
&d}
\ecen
\eeq
The 
green arrow  is the first index in the CC
convention.

Note that if
$x\in V^n\otimes \dual{V}^n$,
then

\beq
(P_{Adj})\indices{
_b^a_d^c
}
x\indices{_c^d}
=
\sum_i (T^i)\indices{_b^a}
\underbrace{
\left[(T^i)\indices{_d^c}
 x\indices{_c^d}
\right]}_{\eps_i\in\RR}
\eeq

\section{Tensor Invariance Conditions}

Recall Eq.(\ref{eq-xprime-eq-gg-x}).
If $x\in V^{n^p}\otimes \dual{V}^{n^q}$, and $\GG\in \calg\subset GL(n^{p+q}, \CC)$,

\beq
(x')\indices{
_{a^{:p}}
^{b^{:q}}
}
=
\GG\indices{
_{a^{:p}}
^{b^{:q}}
_{rev(c^{:q})}
^{rev(d^{:p})}
}
x\indices{
_{d^{:p}}
^{c^{:q}}
},
\quad
x'_\alp=\GG\indices{_\alp^\beta}x_\beta
\eeq
where we define

\beq
\GG\indices{
_\alp
^\beta
}
\eqdef
\prod_{i=1}^p
G\indices{
_{a_i}
^{d_i}
}
\prod_{i=1}^q
\dual{G}\indices{
^{b_i}
_{c_i}
}
\eeq
If $\GG$
is infinitesimally
close to the identity,
then we can parametrize it as

\beqa
\GG\indices{
_\alp
^\beta}
&=&
 1+i\sum_j\eps_j(M^j)
\indices{_\alp^\beta}
\\
G\indices{_{a_i}^{d_i}}
&=&
1+i\sum_j \eps_j 
(T^j)\indices{_{a_i}^{d_i}}
\\
\dual{G}\indices{^{b_i}_{c_i}}
&=&
1-i\sum_j\eps_j
(T^j)\indices{^{b_i}_{c_i}}
\eeqa



Define

\beq
(M^j)
\indices{_\alp^\beta}
=
\left[
(T^j)\indices{_{a_i}^{d_i}}
\frac{1}{\delta_{a_i}^{d_i}}
-
(T^j)\indices{^{b_i}_{c_i}}
\frac{1}{\delta^{b_i}_{c_i}}
\right]
\delta
^{d^{:p}}
_{a^{:p}}
\delta
^{b^{:q}}
_{c^{:q}}
\eeq
When $x_\alp' =x_\alp$, 
to first order in $\eps_i$,

\beq
0=(M^j)\indices{_\alp^\beta}x_\beta=
\left[
(T^j)\indices{_{a_i}^{d_i}}
\frac{1}{\delta_{a_i}^{d_i}}
-
(T^j)\indices{^{b_i}_{c_i}}
\frac{1}{\delta^{b_i}_{c_i}}
\right]
\delta
^{d^{:p}}
_{a^{:p}}
\delta
^{b^{:q}}
_{c^{:q}}
x\indices{
_{d^{:p}}
^{c^{:q}}
}
\eeq
For example,
if we define


\beq
\begin{array}{l}
\myboxed{
(M^j)\indices{_{a_1a_2}^{b_1}_{c_1}^{d_2d_1}
}
=
(T^j)\indices{_{a_1}^{d_1}}
\delta{_{a_2}^{d_2}}
\delta{^{b_1}_{c_1}}
+
\delta{_{a_1}^{d_1}}
(T^j)\indices{_{a_2}^{d_2}}
\delta{^{b_1}_{c_1}}
-
\delta{_{a_1}^{d_1}}
\delta{_{a_2}^{d_2}}
(T^j)\indices{^{b_1}_{c_1}}
}
\\
\bcen
\xymatrix@R=1pc@C=1pc{
&j\ar@{~}@[red][dd]
\\
a_1
&
&d_1\ar[ld]
\\
a_2
&M^j
\ar[lu]
\ar[l]
\ar@{<-}[ld]
&d_2\ar[l]
\\
b_1
&
&c_1\ar@{<-}[lu]
}
\ecen
=
\bcen
\xymatrix@R=1pc@C=1pc{
&j\ar@{~}@[red][d]
\\
a_1
&T^j\ar[l]
&d_1\ar[l]
\\
a_2
&
&d_2\ar[ll]
\\
b_1
&
&c_1\ar@{<-}[ll]
}
\ecen
+
\bcen
\xymatrix@R=1pc@C=1pc{
&j\ar@{~}@[red][dd]
\\
a_1
&
&d_1\ar[ll]
\\
a_2
&T^j\ar[l]
&d_2\ar[l]
\\
b_1
&
&c_1\ar@{<-}[ll]
}
\ecen
-
\bcen
\xymatrix@R=1pc@C=1pc{
&j\ar@{~}@[red][ddd]
\\
a_1
&
&d_1\ar[ll]
\\
a_2
&
&d_2\ar[ll]
\\
b_1
&T^j\ar@{<-}[l]
&c_1\ar@{<-}[l]
}
\ecen
\end{array}
\label{eq-tensor-inv-1}
\eeq
then

\beq
\begin{array}{l}
\myboxed{
0=(M^jx)\indices{_{a_1a_2}^{b_1}}
=
\left[(T^j)\indices{_{a_1}^{d_1}}
\delta{_{a_2}^{d_2}}
\delta{^{b_1}_{c_1}}
+
\delta{_{a_1}^{d_1}}
(T^j)\indices{_{a_2}^{d_2}}
\delta{^{b_1}_{c_1}}
-
\delta{_{a_1}^{d_1}}
\delta{_{a_2}^{d_2}}
(T^j)\indices{^{b_1}_{c_1}}
\right]
x\indices{_{d_1d_2}^{c_1}}
}
\\
0=
\bcen
\xymatrix@R=1pc@C=1pc{
&j\ar@{~}@[red][dd]
\\
a_1
&
&x\ar@2{-}[dd]
\ar[ld]
\\
a_2
&M^j
\ar[lu]
\ar[l]
\ar@{<-}[ld]
&\ar[l]
\\
b_1
&
&\ar@{<-}[lu]
}
\ecen
=
\bcen
\xymatrix@R=1pc@C=1pc{
&j\ar@{~}@[red][d]
\\
a_1
&T^j\ar[l]
&x\ar@2{-}[dd]
\ar[l]
\\
a_2
&
&\ar[ll]
\\
b_1
&
&\ar@{<-}[ll]
}
\ecen
+
\bcen
\xymatrix@R=1pc@C=1pc{
&j\ar@{~}@[red][dd]
\\
a_1
&
&x\ar@2{-}[dd]
\ar[ll]
\\
a_2
&T^j\ar[l]
&\ar[l]
\\
b_1
&
&\ar@{<-}[ll]
}
\ecen
-
\bcen
\xymatrix@R=1pc@C=1pc{
&j\ar@{~}@[red][ddd]
\\
a_1
&
&x\ar@2{-}[dd]
\ar[ll]
\\
a_2
&
&\ar[ll]
\\
b_1
&T^j\ar@{<-}[l]
&\ar@{<-}[l]
}
\ecen
\end{array}
\label{eq-tensor-inv-2}
\eeq
We will refer to
identities such as
Eq.(\ref{eq-tensor-inv-1}) 
and (\ref{eq-tensor-inv-2})
as {\bf tensor invariance conditions}.

\section{Clebsch-Gordan Coefficients}
The Clebsch Gordan (CG) coefficients are
introduced in 
Ch.\ref{ch-clebsch-gordan}.
Note
that 
the generators
$(T^i)\indices{_a^b}$
are a simple
kind of CG coefficient,
one with 
\begin{itemize}
\item a gluon
(adj-rep) particle
instead of
a general $\lam$ rep
particle emanating 
from the $i$
index,
\item 
a particle
of the def-rep
entering
and another leaving
the node,
instead of 
any number of
def-rep particles entering and leaving.
\end{itemize}



Since $\GG= 1 +i\sum_j \eps_j M^j$,
generators decompose in the same way as
the group elements

\beq
\begin{array}{l}
\myboxed
{M^j
=
\sum_\lam C_\lam ^\dagger
T^j_ \lam
C_\lam}
\\
\bcen
\xymatrix@R=1pc@C=1pc{
&j\ar@{~}[dd]
\\
&
&\ar[ld]
\\
&M^j
\ar[lu]
\ar[l]
\ar@{<-}[ld]
&\ar[l]
\\
&
&\ar@{<-}[lu]
}
\ecen
=
\sum_\lam\bcen
\xymatrix@R=1pc@C=1pc{
&
&j\ar@{~}[dd]
&
&
\\
&
&
&
&
\\
&C^\dagger_\lam
\ar[lu]
\ar[l]
\ar@{<-}[ld]
&T^j_\lam\ar[l]
&C_\lam\ar[l]
\ar@{<-}[ru]
\ar@{<-}[r]
\ar[rd]
&
\\
&
&
&
&
}
\ecen
\end{array}
\eeq

The CG coefficients
are invariant tensors.

\beq
C_\lam =
 G_\lam^\dagger C_\lam
 G
\eeq
Hence,

\beq
0 = -T_\lam^j C_\lam
+
C_\lam T^j
\eeq
Note that in the last equation,
$T_\lam^j$ and $T^j$
are different.
In terms of birdtracks, we might have, for example,


\beq
0=
\left\{
\begin{array}{l}
-
\bcen
\xymatrix@R=1pc@C=1.5pc{
&j\ar@{~}@[red][d]
&
&c_1\ar[ld]
\\
a&T_\lam^j
\ar@{.>}[l]
&C_\lam\ar@{.>}[l]
&c_2\ar[l]
\\
&
&
&b_1\ar@{<-}[lu]
}
\ecen
+
\bcen
\xymatrix@R=1pc@C=1.5pc{
&&j\ar@/^1pc/@{~}@[red][d]
\\
&
&T^j\ar[ld]
&c_1\ar[l]
\\
a&C_\lam\ar@{.>}[l]
&c_2\ar[l]
&
\\
&
&b_1\ar@{<-}[lu]
&
}
\ecen
\\
+
\bcen
\xymatrix@R=1pc@C=1.5pc{
&&j\ar@/^1pc/@{~}@[red][dd]
\\
&
&c_1\ar[ld]
&
\\
a&C_\lam\ar@{.>}[l]
&T^j\ar[l]
&c_2\ar[l]
\\
&
&b_1\ar@{<-}[lu]
&
}
\ecen
-
\bcen
\xymatrix@R=1pc@C=1.5pc{
&&j\ar@/^1pc/@{~}@[red][ddd]
\\
&
&c_1\ar[ld]
&
\\
a&C_\lam\ar@{.>}[l]
&c_2\ar[l]
&
\\
&
&T^j\ar@{<-}[lu]
&b_1\ar@{<-}[l]
}
\ecen
\end{array}
\right\}
\label{eq-tlamj-tj}
\eeq
Multiplying 
Eq.(\ref{eq-tlamj-tj})
on the left by $C^\dagger_\lam$,
and moving the first tem to
the right side, we obtain
an expression
for the generator  $T_\lam^i$
in term
the generators $T^j$
(and $C_\lam$ CG coefficients).

\beq
\begin{array}{l}
\bcen
\xymatrix@R=1pc@C=1.5pc{
&j\ar@{~}@[red][d]
&
\\
a
&T_\lam^j
\ar@{.>}[l]
&a'\ar@{.>}[l]
}
\ecen
=
\underbrace{
\bcen
\xymatrix@C=1pc@R=1pc{
&\ar@{~}@[red][d] j
&
&
&
&
\\
a
&\ar@{.>}[l]T_j
&C_\lam\ar[l]
\ar@{<-}[ru]
\ar@{<-}[r]
\ar[rd]
&
&C^\dagger_\lam
\ar[lu]
\ar[l]
\ar@{<-}[ld]
&a'\ar@{.>}[l]
\\
&
&
&
&
&
}
\ecen
-
\bcen
\xymatrix@C=1pc@R=1pc{
&
&
&
&\ar@{~}@[red][d] j
&
\\
a
&C_\lam\ar@{.>}[l]
\ar@{<-}[ru]
\ar@{<-}[r]
\ar[rd]
&
&C^\dagger_\lam
\ar[lu]
\ar[l]
\ar@{<-}[ld]
&\ar[l]T_j
&a'\ar@{.>}[l]
\\
&
&
&
&
&
}
\ecen}_{=0}
\\
+
\bcen
\xymatrix@C=1pc@R=1pc{
&&j\ar@{~}@[red][d]
\\
&
&T^j
&
&
\\
a
&C_\lam\ar@{.>}[l]
\ar@{<-}[ru]
\ar@{<-}[r]
\ar[rd]
&
&C^\dagger_\lam
\ar[lu]
\ar[l]
\ar@{<-}[ld]
&a'\ar@{.>}[l]
\\
&
&
&
&
}
\ecen
+
\bcen
\xymatrix@C=1pc@R=1pc{
&&j\ar@/_1pc/@{~}@[red][dd]
\\
&
&
&
&
\\
a
&C_\lam\ar@{.>}[l]
\ar@{<-}[ru]
\ar@{<-}[r]
\ar[rd]
&T^j
&C^\dagger_\lam
\ar[lu]
\ar[l]
\ar@{<-}[ld]
&a'\ar@{.>}[l]
\\
&
&
&
&
}
\ecen
-
\bcen
\xymatrix@C=.6pc@R=1pc{
&&j\ar@/_1pc/@{~}@[red][ddd]
\\
&
&
&
&
\\
a
&C_\lam\ar@{.>}[l]
\ar@{<-}[ru]
\ar@{<-}[r]
\ar[rd]
&
&C^\dagger_\lam
\ar[lu]
\ar[l]
\ar@{<-}[ld]
&a'\ar@{.>}[l]
\\
&
&T^j
&
&
}
\ecen
\end{array}
\label{eq-inv-3pt-vertex}
\eeq
The term with the underbrace in Eq.(\ref{eq-inv-3pt-vertex})
does not come from
Eq.(\ref{eq-tlamj-tj}).
I included it to demonstrate to
 the reader
that 
Eq.(\ref{eq-inv-3pt-vertex})
is just another
tensor
invariance condition that
touches all the incoming
and outgoing arrows.



\section{Structure Constants
(3 gluon vertex)}

A {\bf Lie Algebra} is an
algebra over the field $\CC$ 
such that its vector product is the matrix commutator (see Section \ref{sec-algebra-over-f}).
Simply put, a Lie Algebra 
is a set 
of square Hermitian matrices $\{T^i\}_{i=1}^N$ that satisfy

\beq
\begin{array}{l}
\myboxed{
\underbrace{T^i T^j-T^j T^i}_{[T^i, T^j]}
= i f_{ijk}T^k
\quad
\text{(\bf Lie Algebra commutation relations)}
}
\\
\bcen
\xymatrix@R=2pc@C=1pc{
a
&T^i\ar[l]
&T^j\ar[l]
&c\ar[l]
\\
&i\ar@{~}[u]
&j\ar@{~}[u]
&
&
}
\ecen
-
\bcen
\xymatrix@R=2pc@C=1pc{
a
&T^j\ar[l]
&T^i\ar[l]
&c\ar[l]
\\
&i\ar@{~}[ur]
&j\ar@{~}[ul]
&
&
}
\ecen
=
i
\bcen
\xymatrix@R=2pc@C=1pc
{
a
&T^k\ar[l]\ar@{~}[d]
&c\ar[l]
\\
&f_{ijk}
\ar@{~}[dl]
\ar@{~}[dr]
&
\\
i
&
&j
\\
}
\ecen
\end{array}
\label{eq-lie-alg-com-rels}
\eeq
The $f_{ijk}$ tensors are called the {\bf structure constants} of the Lie Algebra. They define
a 3 gluon vertex
in term of the generators
$T^i$.\footnote{It's possible
to distinguish between upper and lower gluon indices (i.e., to give the gluon arrows a direction. In that case, the Lie Algebra commutation relations would be $[T^i, T^j]= f\indices{^{ij}_k}T^k$
and the gluon
indices could be lowered
and raised using the metric
(called the {\bf Cartan-Killing form})
$g_{ij}=\tr((T^i)^\dagger T^j)$. 
But since we are assuming 
$g_{ij}=\kappa\delta_i^j$,
there is no need to do
this.
}


If $(T^j)\indices{_a^b}$ are the rep-matrices (in the def-rep) of the
generators
of a group $\calg$, then Eq.(\ref{eq-lie-alg-com-rels}
) shows that
the matrices $(M^k)_{ij}=
if_{ijk}$
are also a rep-matrix (in the adj-rep) of
the generators of $\calg$.

Since $\tr(T^k T^{k'})=\delta(k, k')$,
Eq.(\ref{eq-lie-alg-com-rels}) implies


\beq
\begin{array}{l}
\myboxed{
if_{ijk}
=\tr([T^i, T^j]T^k)
=
(T^i)\indices{_a^c}
(T^k)\indices{_c^b}
(T^j)\indices{_b^a}
-
(T^i)\indices{_a^c}
(T^j)\indices{_c^b}
(T^k)\indices{_b^a}
}
\\
i
\bcen
\xymatrix@R=1pc@C=1pc{
&i\ar@{~}[d]
\\
&f_{ijk}
&
\\
j\ar@{~}[ur]
&
&k\ar@{~}[lu]
}
\ecen
=
\bcen
\xymatrix@R=2pc@C=1pc{
&
&i\ar@{~}[d]
&
&
\\
&
&T^i\ar@{<-}[ld]|{\sum a}
&
&
\\
&T^j\ar@{<-}[r]
&\sum b\ar@{<-}[r]
&T^k\ar@{<-}[lu]|{\sum c}
&
\\
j\ar@{~}[ru]
&
&
&
&k\ar@{~}[lu]
}
\ecen
-
\bcen
\xymatrix@R=2pc@C=1pc{
&
&i\ar@{~}[d]
&
&
\\
&
&T^i\ar@{<-}[ld]|{\sum a}
&
&
\\
&T^k\ar@{<-}[r]
&\sum b\ar[r]
&T^j\ar@{<-}[lu]|{\sum c}
&
\\
j\ar@{~}[urrr]
&
&
&
&k\ar@{~}[ulll]
}
\ecen
\end{array}
\eeq
Note that

\beq
\begin{array}{l}
\myboxed{
f_{ijk}=-f_{jik}
}
\\
\bcen
\xymatrix@R=1pc@C=1pc{
&k\ar@{~}[d]
\\
&f_{ijk}
&
\\
i\ar@{~}[ur]
&
&j\ar@{~}[lu]
}
\ecen
=
-
\bcen
\xymatrix@R=1pc@C=1pc{
&k\ar@{~}[d]
\\
&f_{ijk}
\ar@{~}[dl]
\ar@{~}[dr]
&
\\
&
&
\\
i\ar@{~}[urr]
&
&j\ar@{~}[ull]
}
\ecen
\text{(Convention CC)}
\\
\bcen
\xymatrix@R=1pc@C=1pc{
&k\ar@{~}[d]
\\
&f_{\rvi\rvj \rvk}
&
\\
i\ar@{~}[ur]
&
&j\ar@{~}[lu]
}
\ecen
=
-
\bcen
\xymatrix@R=1pc@C=1pc{
&k\ar@{~}[d]
\\
&f_{\rvj\rvi \rvk}
&
\\
i\ar@{~}[ur]
&
&j\ar@{~}[lu]
}
\ecen
\text{(Convention FL)}
\end{array}
\eeq
In fact, the tensor $f_{ijk}$
is {\bf totally antisymmetric} (i.e., it changes
sign under a transposition
of any two indices).

\begin{claim}
$f_{ijk}$ is 
a real number.
\end{claim}
\proof

\beqa
\left[i\tr([T^i, T^j]T^k)\right]^\dagger
&=&
(-i)\tr(T^k[T^j, T^i])
\\
&=&
(-i)\tr([T^j, T^i]T^k)
\\
&=&
i\tr([T^i, T^j]T^k)
\eeqa
\qed

Note that the birdtrack for the Lie Algebra 
commutation
relations Eq.(\ref{eq-lie-alg-com-rels})
can be understood as 
the statement 
that the generators $T^j$
are invariant matrices.
Below we restate 
Eq.(\ref{eq-lie-alg-com-rels}) to make that obvious

\beq
0=
\bcen
\xymatrix@R=2pc@C=1pc{
a
&T^i\ar[l]
&T^j\ar[l]
&c\ar[l]
\\
&i\ar@{~}@[red][u]
&j\ar@{~}[u]
&
&
}
\ecen
-
\bcen
\xymatrix@R=2pc@C=1pc{
a
&T^j\ar[l]
&T^i\ar[l]
&c\ar[l]
\\
&i\ar@{~}@[red][ur]
&j\ar@{~}[ul]
&
&
}
\ecen
-i
\bcen
\xymatrix@R=2pc@C=1pc
{
a
&T^k\ar[l]\ar@{~}[d]
&c\ar[l]
\\
&f_{ijk}
\ar@{~}@[red][dl]
\ar@{~}[dr]
&
\\
i
&
&j
\\
}
\ecen
\eeq




\begin{claim}
\label{cl-cijk-is-invariant}
\beq
\begin{array}{l}
\myboxed{
f_{ijm}f_{mkl}
-
f_{ljm}f_{mki}
=
f_{iml}f_{jkm}\quad
\text{\bf(Jacobi identity)}} 
\\
\bcen
\xymatrix@C=3pc{
i\ar@{~}[d]
&l\ar@{~}[d]
\\
f_{ljm}
&f_{mki}\ar@{~}[l]|{\sum m}
\\
j\ar@{~}[u]
&
k\ar@{~}[u]
}
\ecen
-
\bcen
\xymatrix@C=3pc{
i\ar@{~}[dr]
&l\ar@{~}[dl]
\\
f_{ljm}
&f_{mki}\ar@{~}[l]|{\sum m}
\\
j\ar@{~}[u]
&
k\ar@{~}[u]
}
\ecen
=
\bcen
\xymatrix{
i\ar@{~}[rd]
&
&l\ar@{~}[ld]
\\
&f_{iml}\ar@{~}[d]|{\sum m}
\\
&f_{jkm}
\\
j\ar@{~}[ur]
&&k\ar@{~}[ul]
}
\ecen
\end{array}
\eeq
\end{claim}
\proof

Note that

\beqa
\tr\left(
[[T^i, T^j],T^k]T^l
\right)
&=&
\tr\left(
f_{ijm}[T^m, T^k]T^l
\right)
\\
&=&
\tr\left(
f_{ijm}f_{mkl'}T^{l'}T^l
\right)
\\
&=&
f_{ijm}f_{mkl}
\eeqa
so
the Jacobi identity 
can be restated as

\beq
\tr\left(
\left\{
[[T^i, T^j], T^k]
+
[[T^j, T^k], T^i]
+
[[T^k, T^i], T^j]
\right\}T^l 
\right)=0
\eeq
Hence,
the claim follows if we can prove that

\beq
\underbrace{[[T^i, T^j], T^k]
+
[[T^j, T^k], T^i]
+
[[T^k, T^i], T^j]}_{
\text{cyclic permutations of $ijk$}}
=0
\label{eq-ttt-cyclic}
\eeq
If we expand
the left hand side on Eq.(\ref{eq-ttt-cyclic}),
we find 6 terms that cancel
in pairs.
\qed

Note Claim
\ref{cl-cijk-is-invariant}
can be undertood
as the Lie Algebra commutation relations
Eq.(\ref{eq-lie-alg-com-rels}), but stated in the adj-rep
instead of the def-rep. Indeed,
if 

\beq
M^i_{jk} = if_{ijk}
\eeq
then Claim
\ref{cl-cijk-is-invariant}
becomes

\beq
(M^i M^l - M^l M^i)_{jk}
=
if_{ilm}(M^m)_{jk}
\eeq


Note that Claim
\ref{cl-cijk-is-invariant}
can be understood as a statement of the fact that $f_{ijk}$ is an invariant
tensor.

\beq
\begin{array}{l}
\myboxed{
0=
f_{ijm}f_{mkl}
-
f_{ljm}f_{mki}
-
f_{iml}f_{jkm}
}
\\
0=
\bcen
\xymatrix@C=3pc{
i\ar@{~}@[red][d]
&l\ar@{~}[d]
\\
f_{ijm}
&f_{mkl}\ar@{~}[l]|{\sum m}
\\
j\ar@{~}[u]
&
k\ar@{~}[u]
}
\ecen
-
\bcen
\xymatrix@C=3pc{
i\ar@{~}@[red][dr]
&l\ar@{~}[dl]
\\
f_{ljm}
&f_{mki}\ar@{~}[l]|{\sum m}
\\
j\ar@{~}[u]
&
k\ar@{~}[u]
}
\ecen
-
\bcen
\xymatrix{
i\ar@{~}@[red][rd]
&
&l\ar@{~}[ld]
\\
&f_{iml}\ar@{~}[d]|{\sum m}
\\
&f_{jkm}
\\
j\ar@{~}[ur]
&&k\ar@{~}[ul]
}
\ecen
\end{array}
\eeq


\section{Other Forms of Lie Algebra Commutators}

Consider the following two gluon exchange
operators. Note that
$\PP^2=\PP$,  but $\QQ^2\neq \QQ$,
so $\PP$ is a bonafide projection operator
but  $\QQ$
isn't. $\QQ\QQ^\dagger =\PP$ so
$\QQ$ behaves like half of a projection operator.
 

\beq
\myboxed
{{\PP}\indices{_a^b_c^d}
=
\sum_i (T^i)\indices{_a^b}
(T^i)\indices{_c^d}}
\bcen
\xymatrix{
a
&
&d
\\
&\PP\ar@[green][ul]
\ar@{<-}[ur]
\ar[dr]
&
\\
b\ar[ur]
&
&c
}
\ecen
=
\bcen
\xymatrix@C=3pc{
a
&d
\ar[d]
\\
T^i\ar[u]
\ar@{~}@[green][r]|{\sum i}
&T^i
\ar[d]
\\
b\ar[u]
&c
}
\ecen
\eeq




\beq
\myboxed{\QQ\indices{_a^b_Y^X}
=
\sum_i
(T^i)\indices{_a^b}
(T^i_\lam)\indices{_Y^X}
}
\bcen
\xymatrix{
a
&
&X
\\
&\QQ\ar@[green][ul]
\ar@2{<-}[ur]
\ar@2{->}[dr]
&
\\
b\ar[ur]
&
&Y
}
\ecen
=
\bcen
\xymatrix@C=3pc{
a
&X
\ar@2{->}[d]
\\
T^i\ar[u]
\ar@{~}@[green][r]|{\sum i}
&T_\lam^i
\ar@2{->}[d]
\\
b\ar[u]
&Y
}
\ecen
\eeq

\begin{claim}
If $\QQ\indices{_b^a}$ is
the matrix with $(Z, X)$ entries 
$\QQ\indices{_b^a_Z^X}$, then
 \beq
 [
 \QQ\indices{_b^a},
 \QQ\indices{_d^c}
 ]=
\PP\indices
{
_{c'}
^c
^a
_b
}
\QQ\indices{
_d
^{c'}
}
-
\QQ\indices{
_{d'}
^c
}
\PP\indices{
_d
^{d'}
^a
_b
}
\eeq

\end{claim}
\proof
 

This claim can be visualized as follows.
$\QQ$ is an invariant tensor so


\beq
0=
\left\{
\begin{array}{l}
\bcen
\xymatrix{
c\ar[ddrr]
&
&
&d
\\
i\ar@{~}@[red][dr]
&
&
&
\\
Y
&T_\lam^i\ar@2{->}[l]
&\QQ\ar@2{->}[l]
\ar@[green][ruu]
&X\ar@2{->}[l]
}
\ecen
-
\bcen
\xymatrix{
c\ar[ddr]
&
&
&d
\\
i\ar@{~}@[red][drr]
&
&
&
\\
Y
&\QQ\ar@2{->}[l]
\ar@[green][rruu]
&T_\lam^i\ar@2{->}[l]
&X\ar@2{->}[l]
}
\ecen
\\
-
\bcen
\xymatrix{
c\ar[r]
&T^i
\ar[r]
&\QQ\ar@[green][r]
\ar@2{->}[ddll]
&d
\\
i\ar@[red]@{~}[ur]
&
&
&
\\
Y
&
&
&X\ar@2{->}[uul]
}
\ecen
+
\bcen
\xymatrix{
c\ar[r]
&\QQ
\ar[r]
\ar@2{->}[ddl]
&T^i\ar[r]
&d
\\
i\ar@{~}@[red][urr]
&
&
&
\\
Y
&
&
&X\ar@2{->}[uull]
}
\ecen
\end{array}
\right\}
\eeq
Now multiplying by $(T^i)\indices{_a^b}$, we get

\beq
\begin{array}{l}
\myboxed{
\QQ\indices{
_b
^a
_Y
^{Z}
}
\QQ\indices{
_d
^c
_{Z}
^X
}
-
\QQ\indices{
_d
^c
_Y
^{Z}
}
\QQ\indices{
_b
^a
_Z
^X
}
=
\PP\indices
{
_{c'}
^c
^a
_b
}
\QQ\indices{
_d
^{c'}
_Y
^X
}
-
\QQ\indices{
_{d'}
^c
_Y
^X
}
\PP\indices{
_d
^{d'}
^a
_b
}
}
\\
\begin{array}{l}
\bcen
\xymatrix{
c\ar[ddrr]
&
&
&d
\\
a\ar[dr]
&
&
&b
\\
Y
&\QQ\ar@2{->}[l]
\ar@[green][rru]
&\QQ\ar@2{->}[l]
\ar@[green][ruu]
&X\ar@2{->}[l]
}
\ecen
-
\bcen
\xymatrix{
c\ar[ddr]
&
&
&d
\\
a\ar[drr]
&
&
&b
\\
Y
&\QQ\ar@2{->}[l]
\ar@[green][rruu]
&\QQ\ar@2{->}[l]
\ar@[green][ur]
&X\ar@2{->}[l]
}
\ecen
=
\\
\quad\quad\quad
\bcen
\xymatrix{
c\ar[r]
&T^i
\ar[r]
\ar@{~}[d]
&\QQ\ar@[green][r]
\ar@2{->}@/_1pc/[ddll]
&d
\\
a\ar[r]
&T^i\ar[rr]
&
&b
\\
Y
&
&
&X\ar@2{->}[uul]
}
\ecen
-
\bcen
\xymatrix{
c\ar[r]
&\QQ
\ar@[green][r]
\ar@2{->}[ddl]
&T^j\ar[r]
\ar@{~}[d]
&d
\\
a\ar[rr]
&\ar[r]
&T^j\ar[r]
&b
\\
Y
&
&
&X\ar@2{->}@/_1pc/[uull]
}
\ecen
\end{array}
\end{array}
\eeq
Finally, if we hide the 
capital letter indices
to  obtain a statement
about matrices with capital
letter indices, we get

\beq
\QQ\indices{
_b
^a
}
\QQ\indices{
_d
^c
}
-
\QQ\indices{
_d
^c
}
\QQ\indices{
_b
^a
}
=
\PP\indices
{
_{c'}
^c
^a
_b
}
\QQ\indices{
_d
^{c'}
}
-
\QQ\indices{
_{d'}
^c
}
\PP\indices{
_d
^{d'}
^a
_b
}
\eeq

\qed

