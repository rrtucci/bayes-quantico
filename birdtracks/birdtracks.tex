\chapter{Birdtracks: COMING SOON}
\label{ch-birdtracks}

Cvitanovic Birdtracks book \cite{birdtracks-book}


Elliott-Dawber book \cite{eli-daw-book}

My paper \qt{Quantum Bayesian Nets} \cite{tucci-qbnets}

\section{Classical Bayesian Networks and their Instantiations}

TPM (Transition Probability Matrix) $P(y|x)\in [0,1]$
where  $x\in val(\rvx)$ and $y\in val(\rvy)$

\beq
\sum_{y\in val(\rvy)}P(y|x) = 1
\eeq

\beq
\calc=
\bcen
\xymatrix{
&\rvb\ar[ld]
\\
\rvc
&&\rva\ar[ll]\ar[lu]
}
\ecen
\eeq

\beq
\calc(a,b,c)
=
P(c|b,a)P(b|a)P(a)
=
\bcen
\xymatrix{
&b\ar[ld]
\\
c
&&a\ar[ll]\ar[lu]
}
\ecen
P(a)
\eeq

$a^{:2} = (a_1, a_2)$
\beq
\calc'=
\bcen
\xymatrix{
&\rvb\ar[ld]
\\
\rvc
&&\rva^{:2}\ar[ll]|{\rva_2}\ar[lu]|{\rva_1}
}
\ecen
\eeq

\beq
\calc'(a^{:2},b,c)
=
P(c|b,a_2)P(a_2|a^{:2})P(b|a_1)P(a_1|a^{:2})P(a^{:2}
)
=
\bcen
\xymatrix{
&b\ar[ld]
\\
c
&&a^{:2}
\ar[ll]|{a_2}\ar[lu]|{a_1}
}
\ecen
P(a^{:2})
\eeq

Marginalizer nodes  $\rva_1$ and $\rva_2$
have the TPMs
\beq \color{blue}
P(a'_i|\rva^{:2}=(a_1,a_2)) = \delta(a'_i, a_i)
\eeq
for $i=1,2$


\section{Quantum Bayesian Networks and
their Instantiations}

TPM (Transition Probability Matrix) 
$A(y|x)\in\CC$
where  $x\in val(\rvx)$ and $y\in val(\rvy)$

\beq
\sum_{y\in val(\rvy)}|A(y|x)|^2 = 1
\eeq

\beq
\calq=
\bcen
\xymatrix{
&\rvb\ar[ld]
\\
\rvc
&&\rva\ar[ll]\ar[lu]
}
\ecen
\eeq

\beq
\calq(a,b,c)
=
A(c|b,a)A(b|a)A(a)
=
\bcen
\xymatrix{
&b\ar[ld]
\\
c
&&a\ar[ll]\ar[lu]
}
\ecen
A(a)
\eeq

$a^{:2} = (a_1, a_2)$
\beq
\calq'=
\bcen
\xymatrix{
&\rvb\ar[ld]
\\
\rvc
&&\rva^{:2}\ar[ll]|{\rva_2}\ar[lu]|{\rva_1}
}
\ecen
\eeq

\beq
\calq'(a^{:2},b,c)
=
A(c|b,a_2)A(a_2|a^{:2})A(b|a_1)A(a_1|a^{:2})A(a^{:2})
=
\bcen
\xymatrix{
&b\ar[ld]
\\
c
&&a^{:2}
\ar[ll]|{a_2}\ar[lu]|{a_1}
}
\ecen
A(a^{:2})
\eeq

Marginalizer nodes  $\rva_1$ and $\rva_2$
have the TAMs
\beq \color{blue}
A(a'_i|\rva^{:2}=(a_1,a_2)) = \delta(a'_i, a_i)
\eeq
for $i=1,2$

\section{Birdtracks}



\beq
\delta(b,a)=\indi(a=b)=
\delta^b_a =
\xymatrix{a&\ar[l]|\bullet b}
\eeq


\beq
\bra{a,b}
X\indices{_\rva_\rvb^\rvc^\rvd}
\ket{c,d}
=
X\indices{_a_b^c^d}
=
\bcen
\xymatrix@R=1pc{
\rva=a
&X\indices{_\rva_\rvb^\rvc^\rvd}
\ar[dl]\ar[l]
\\
\rvb=b
\\
\rvc=c\ar[ruu]
\\
\rvd=d\ar[ruuu]
}\ecen
\eeq

\beq
\bcen
\xymatrix@R=1pc{
a
&X\indices{_\rva_\rvb^\rvc^\rvd}
\ar[dl]\ar[l]
\\
b
\\
c\ar[ruu]
\\
d\ar[ruuu]
}\ecen
\rarrow
\bcen
\xymatrix@R=1pc{
a,b
&X\indices{_\rva_\rvb^\rvc^\rvd}
\ar[dl]\ar[l]
\\
a,b
\\
c\ar[ruu]
\\
d\ar[ruuu]
}\ecen
\eeq
$X\indices{_\rva_\rvb^\rvc^\rvd}\in V^2 \otimes V_2$.
Sometimes, 
we will omit denote
this node simply by $X$.
This if okay as long as
we are not using,
$X$ to also denote
a different version of $X\indices{_\rva_\rvb^\rvc^\rvd}$
with some of the indices
raised or lowered or 
their order has been changed.
\footnote{For matrices,
$(A^\dagger)_{i,j} = (A_{j, i})^*$
so
taking a Hermitian conjugate
involves both taking
the complex conjugate of
the matrix element and reversing the left-to-right (L2R) order of its indices.
This generalizes to 
$(X^\dagger)\indices{_d_c^b^a}=
(X\indices{_a_b^c^d})^*$.
Besides raising and lowering indices, we reverse their L2R order.
}

\beq
(X^\dagger)\indices{_d_c^b^a}
=
\bcen
\xymatrix@R=1pc{
(X^\dagger)\indices{_\rvd_\rvc^\rvb^\rva}
&\rva=a\ar[l]
\\
&\rvb=b\ar[lu]
\\
&\rvc=c\ar[luu]
\\
&\rvd=d\ar[luuu]
}\ecen
\eeq


\beqa
(X^\dagger)\indices{_d_c^b^a}
X\indices{_a_b^c^d}
&=&
\bcen
\xymatrix@R=1pc{
(X^\dagger)\indices{_\rvd_\rvc^\rvb^\rva}
&\sum a\ar[l]\ar@{<-}[r]
&
X\indices{_\rva_\rvb^\rvc^\rvd}
\\
&\sum b\ar[ul]\ar@{<-}[ur]
&
\\
&\sum c\ar@{<-}[luu]
\ar[ruu]
&
\\
&\sum d\ar@{<-}[luuu]
\ar[ruuu]
&
}
\ecen
\\
&=&
\bcen
\xymatrix@R=1pc{
X^\dagger
&\ar[l]\ar@{<-}[r]
&
X
\\
&\ar[ul]\ar@{<-}[ur]
&
\\
&\ar@{<-}[luu]
\ar[ruu]
&
\\
&\ar@{<-}[luuu]
\ar[ruuu]
&
}
\ecen
\eeqa

Birdtracks originated as a graphical
way to represent the tensors in General Relativity (Gravitation). In General Relativity, one deals with tensors such as
$T\indices{_a^b_c}$ which have some indices raised
and some lowered. One can use the metric 
$g^{a,b}$ to raise all the lowered indices
to get $T^{abc}$. If we represent this
graphically as a node with incoming arrows 
$a,b,c$, we need to 
follow one of the following
2 conventions: either
\begin{enumerate}
\item
label the arrows 
as $\rva$, $\rvb, \rvc$, 
and define the node as
$T^{\rva\rvb\rvc}$,
or
\item
instead of labelling the
arrows explicitly $\rva, \rvb, \rvc$, 
 indicate in the node
where is the first arrow
$\rva$, and draw the
arrows $\rva, \rvb, \rvc$
so that they enter the node
in {\bf counterclockwise} (CC) order.
The {\bf left-to-right} (L2R) order
of the indices on $T$ corresponds
the CC order of the arrows.
\end{enumerate}
If we don't do either 1 or 2, we won't
be able to distinguish between
the graphical
representations of $T^{1,2,3}$
and $T^{2,1,3}$, for example.
Cvitanovic's Birdtracks book
Ref.\cite{birdtracks-book} follows Convention 2, but
most of the time, in this book, we will follow
Convention 1 \footnote{If we follow Convention 1,
we don't need to reverse the L2R order of the indices
when taking a Hermitian conjugate. Thus,
$(X^\dagger)\indices{^\rva^\rvb_\rvc_\rvd}=
X\indices{_\rva_\rvb^\rvc^\rvd}=
X\indices{^\rvd^\rvc_\rvb_\rva}$.
As long as $\rva, \rvb$ are lower indices and $\rvc,\rvd$ are upper
indices of $X$, any L2R
order of $\rva, \rvb, \rvc, \rvd$ 
is equivalent
under Convention 1.}
The reason I chose to do so is for the sake of consistency:
Convention 2 
is closer to the quantum bnet conventions. 


Another issue that arises in using  birdtracks is this.
When is it permissible to represent 
a tensor by $T_{ab}^{cd}$?
If we define
$T_{ab}^{cd}$  by
\beq
T_{ab}^{cd} = T\indices{_a_b^c^d}
\eeq
then it's always permissible.
Then one can define
tensors like
$T\indices{_a^b^c^d}$
as 

\beq
T\indices{_a^b^c^d}=
g^{bb'}T\indices{_a_{b'}^c^d}
=
g^{bb'}T_{ab'}^{cd}
\eeq
Hence, one drawback of
using the notation
$T_{ab}^{cd}$
is that if one is interested 
in using versions of
$T_{ab}^{cd}$ with
some indices raised or 
lowered, one has to 
write down explicitly the metric tensors 
that do the lowering and
raising.
Instead of writing
$T\indices{_a^b^c^d}$,
you'll have to write
$g^{bb'}T_{ab'}^{cd}$.
This is not very onerous when 
explaining a topic
in which not much
lowering and raising of indices is
done. But in topics like
General Relativity that do
use a lot of raising and lowering of indices, it might not be 
too elegantly concise.


$a^{:m}\in \ZZ_+^m$

\beq
R^{a_3^{:m_3}, b_2^{:n_2}}
_{b_3^{:n_3}, a_2^{:m_2}}
S^{a_2^{:m_2}, b_1^{:n_1}}
_{b_2^{:n_2}, a_1^{:m_1}} =
\bcen
\xymatrix{
b_3^{:n_3}
&R\ar[l]\ar@{<-}[ld]
&\sum  b_2^{:n_2}\ar[l]
&S\ar[l]\ar@{<-}[ld]
&b_1^{:n_1}\ar[l]
\\
a_3^{:m_3}
&
&\sum a_2^{:m_2}\ar@{<-}[lu]
&
&a_1^{:m_1}\ar@{<-}[lu]
}
\ecen
\eeq

\beq
\tr_\rvb X\indices{_a_\rvb^\rvb^d}
=
\sum_b X\indices{_a_b^b^d}
=
\bcen
\xymatrix@R=1pc{
a
&X\indices{_\rva_\rvb^\rvc^\rvd}
\ar[dl]\ar[l]
\\
\ar@[red]@{-}[d]
&
\\
\ar[ruu]
\\
d\ar[ruuu]
}\ecen
\eeq


\beq
\xymatrix{
&\ar[d]
&
&\ar@{<-}[d]\ar@[red]@{-}[ll]
&
\\
&R\ar[l]\ar@{<-}[ld]
&
&S\ar@{<-}[ld]
&\ar[l]
\\
&
&\ar@{<-}[lu]
&
&\ar@{<-}[lu]
}
\eeq