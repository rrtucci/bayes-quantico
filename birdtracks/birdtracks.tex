\chapter{Birdtracks}
\label{ch-birdtracks}

This chapter is based on Cvitanovic Birdtracks book Ref. \cite{birdtracks-book}
and my paper Ref. \cite{tucci-qbnets}


The tensor notation 
(see Sec.\ref{sec-tensors})
 is succinct and easy to follow,
but it's not
visually
illuminating. The birdtrack notation is not as succinct
as the tensor notation and can lead to sign 
errors if you are careless,
but it is very visually illuminating. Thus, the tensor notation
and birdtrack notation complement each other well.
We will often display results
using both, side by side.

\section{Classical Bayesian Networks and their Instantiations}

Classical Bayesian Networks (bnets)
are discussed exhaustively
in the first book of this 
series, Ref.\cite{bayesuvius}.
This is a brief section
to remind the reader
of how they are defined.

Let PD stand for probability distribution.
We call $P_{\rvy|\rvx}:val(\rvy)
\times val(\rvx)
\rarrow  [0,1]$ a
{\bf Transition Probability Matrix} (TPM)\footnote{A TPM is also
known as a Conditional Probability Table (CPT).} if 

\beq
\sum_{y\in val(\rvy)}P_{\rvy|\rvx}(y|x) = 1
\eeq
In other words,
a TPM is a conditional PD. A TPM of the form

\beq
P(y|x)= 
\delta(y, f(x))
\eeq
for some function
$f:val(\rvx)
\rarrow val(\rvy)$
is said to be {\bf deterministic}.



A bnet is a 
{\bf Directed Acyclic Graph} (DAG) 
with the nodes labelled by
random variables\footnote{As in
the first volume of this series, 
we indicate random variables by underlined letters}. Each
bnet stands for a full 
PD  of the node random variables expressed
as a product of a TPM for each node.
For example, the bnet

\beq
\calc=
\bcen
\xymatrix{
&\rvb\ar[ld]
\\
\rvc
&&\rva\ar[ll]\ar[lu]
}
\ecen
\label{eq-c-bnet-def}
\eeq
stands for the full 
PD

\beq
P(a,b,c)=
P(c|b,a)P(b|a)P(a)
\eeq
Bnets 
do not have free
indices
because 
their nodes are labelled by random
variables. It is convenient
to use the DAG for a 
bnet but with the
underlining
removed from the random variables,
and
assign a numerical value to this new DAG.
The resultant DAG now
has free indices. We call it an
{\bf instantiation of the 
bnet}.
For example, from the
bnet $\calc$ 
of Eq.(\ref{eq-c-bnet-def}),
we get the
instantiation\footnote{
Note that we don't
include the root node
probabilities as part of 
the graph value. Thus,
 
$P(a,b) =
\underbrace{\xymatrix{ b\rarrow a}}_{P(b|a)}
P(a)$}


\beq
P(a,b,c)
=
P(c|b,a)P(b|a)P(a)
=
\bcen
\xymatrix{
&b\ar[ld]
\\
c
&&a\ar[ll]\ar[lu]
}
\ecen
P(a)
\eeq

Let $a^{:2} = (a_1, a_2)$.
Based on
the bnet $\calc$ of
Eq.(\ref{eq-c-bnet-def}),
define a new bnet $\calc'$
as follows

\beq
P'=
\bcen
\xymatrix{
&\rvb\ar[ld]
\\
\rvc
&&\rva^{:2}\ar[ll]|{\rva_2}\ar[lu]|{\rva_1}
}
\ecen
\eeq
$\calc'$ represents the 
the full PD 

\beq
P(a^{:2}, b, c)=
P(c|b,a_2)P(a_2|a^{:2})P(b|a_1)P(a_1|a^{:2})P(a^{:2}
)
\eeq
The 2 new nodes
$\rva_1$ and $\rva_2$
of bnet $\calc'$
are called 
{\bf marginalizer nodes}.
We assign to them
the following TPMs (printed in blue):

\beq \color{blue}
P[a'_i|\rva^{:2}=(a_1,a_2)] = \delta(a'_i, a_i)
\eeq
for $i=1,2$.
We can also
define an instantiation of $\calc'$ as follows:

\beq
P'(a^{:2}, b, c)
=
\bcen
\xymatrix{
&b\ar[ld]
\\
c
&&a^{:2}
\ar[ll]|{a_2}\ar[lu]|{a_1}
}
\ecen
P(a^{:2})
\eeq




\section{Quantum Bayesian Networks and
their Instantiations}

As far as I know,
Quantum Bayesian Networks
(qbnets) were invented by me in Ref.\cite{tucci-qbnets}.

qbnets are closely
analogous to classical
bnets, but the TPM
are replaced by Transition Probability 
Amplitudes (TPA).

Let PA stand for probability amplitude. We call $A_{\rvy|\rvx}:val(\rvy)
\times val(\rvx)
\rarrow  \CC$ a
TPA if

\beq
\sum_{y\in val(\rvy)}|A(y|x)|^2 = 1
\label{eq-bnet-normalization}
\eeq
Note that if $A$ is the matrix with entries
$\av{y|A|x}=A(y|x)$, then

\beq
\av{y|A^\dagger A|x}=\sum_{y\in val(\rvy)}|A(y|x)|^2 = 1
\eeq
If $A$ is a unitary matrix, then $A^\dagger A= AA^\dagger =1$ so 
\qt{half} ($A^\dagger A=1$) of
the definition
of unitary matrix is satisfied.
If both parts were 
satisfied, $A$ would have to be a square matrix.

A qbnet is a 
DAG
with the nodes labelled by
random variables. Each
qbnet stands for a full 
PA  of the node random variables expressed
as a product of a TPA for each node.
For example, the qbnet


\beq
\calq=
\bcen
\xymatrix{
&\rvb\ar[ld]
\\
\rvc
&&\rva\ar[ll]\ar[lu]
}
\ecen
\label{eq-q-bnet-def}
\eeq
stands for the full PA

\beq
A(a,b,c)
=
A(c|b,a)A(b|a)A(a)
\eeq
Qbnets 
do not have free
indices
because 
their nodes are labelled by random
variables. It is convenient
to use the DAG for a 
qbnet but with the
underlining
removed from the random variables,
and
assign a numerical value to this new DAG.
The resultant DAG now
has free indices. We call it an
{\bf instantiation of the 
qbnet}.
For example, from the
bnet $\calq$ 
of Eq.(\ref{eq-q-bnet-def}),
we get the
instantiation

\beq
A(a,b,c)
=
A(c|b,a)A(b|a)A(a)
=
\bcen
\xymatrix{
&b\ar[ld]
\\
c
&&a\ar[ll]\ar[lu]
}
\ecen
A(a)
\eeq

Let $a^{:2} = (a_1, a_2)$.
Based on
the qbnet $\calq$ of
Eq.(\ref{eq-q-bnet-def}),
define a new qbnet $\calq'$
as follows

\beq
\calq'=
\bcen
\xymatrix{
&\rvb\ar[ld]
\\
\rvc
&&\rva^{:2}\ar[ll]|{\rva_2}\ar[lu]|{\rva_1}
}
\ecen
\eeq
$\calq'$  represents the 
the full PA 

\beq
A(a^{:2}, b, c)=
A(c|b,a_2)A(a_2|a^{:2})A(b|a_1)A(a_1|a^{:2})A(a^{:2}
)
\eeq
The 2 new nodes
$\rva_1$ and $\rva_2$
of qbnet $\calq'$
are called 
{\bf marginalizer nodes}.
We assign to them
the following TPAs (printed in blue):

\beq \color{blue}
A[a'_i|\rva^{:2}=(a_1,a_2)] = \delta(a'_i, a_i)
\eeq
for $i=1,2$.
We can also
define an instantiation of $\calq'$ as follows:

\beq
A(a^{:2}, b, c)
=
\bcen
\xymatrix{
&b\ar[ld]
\\
c
&&a^{:2}
\ar[ll]|{a_2}\ar[lu]|{a_1}
}
\ecen
A(a^{:2})
\eeq




\section{Birdtracks}




\beq
\delta(b,a)=\indi(a=b)=
\delta^b_a =
\xymatrix{a&\ar[l]|\bullet b}
\eeq


\beq
\bra{a,b}
X\indices{_\rva_\rvb^\rvc^\rvd}
\ket{c,d}
=
X\indices{_a_b^c^d}
=
\bcen
\xymatrix@R=1pc{
\rva=a
&X\indices{_\rva_\rvb^\rvc^\rvd}
\ar[dl]\ar[l]
\\
\rvb=b
\\
\rvc=c\ar[ruu]
\\
\rvd=d\ar[ruuu]
}\ecen
\eeq

\beq
\bcen
\xymatrix@R=1pc{
a
&X\indices{_\rva_\rvb^\rvc^\rvd}
\ar[dl]\ar[l]
\\
b
\\
c\ar[ruu]
\\
d\ar[ruuu]
}\ecen
\rarrow
\bcen
\xymatrix@R=1pc{
a,b
&X\indices{_\rva_\rvb^\rvc^\rvd}
\ar[dl]\ar[l]
\\
a,b
\\
c\ar[ruu]
\\
d\ar[ruuu]
}\ecen
\eeq
$X\indices{_\rva_\rvb^\rvc^\rvd}\in V^2 \otimes V_2$.
Sometimes, 
we will omit denote
this node simply by $X$.
This if okay as long as
we are not using,
$X$ to also denote
a different version of $X\indices{_\rva_\rvb^\rvc^\rvd}$
with some of the indices
raised or lowered or 
their order has been changed.
\footnote{For matrices,
$(A^\dagger)_{i,j} = (A_{j, i})^*$
so
taking a Hermitian conjugate
involves both taking
the complex conjugate of
the matrix element and reversing the left-to-right (L2R) order of its indices.
This generalizes to 
$(X^\dagger)\indices{_d_c^b^a}=
(X\indices{_a_b^c^d})^*$.
Besides raising and lowering indices, we reverse their L2R order.
}

\beq
(X^\dagger)\indices{_d_c^b^a}
=
\bcen
\xymatrix@R=1pc{
(X^\dagger)\indices{_\rvd_\rvc^\rvb^\rva}
&\rva=a\ar[l]
\\
&\rvb=b\ar[lu]
\\
&\rvc=c\ar[luu]
\\
&\rvd=d\ar[luuu]
}\ecen
\eeq


\beqa
(X^\dagger)\indices{_d_c^b^a}
X\indices{_a_b^c^d}
&=&
\bcen
\xymatrix@R=1pc{
(X^\dagger)\indices{_\rvd_\rvc^\rvb^\rva}
&\sum a\ar[l]\ar@{<-}[r]
&
X\indices{_\rva_\rvb^\rvc^\rvd}
\\
&\sum b\ar[ul]\ar@{<-}[ur]
&
\\
&\sum c\ar@{<-}[luu]
\ar[ruu]
&
\\
&\sum d\ar@{<-}[luuu]
\ar[ruuu]
&
}
\ecen
\\
&=&
\bcen
\xymatrix@R=1pc{
X^\dagger
&\ar[l]\ar@{<-}[r]
&
X
\\
&\ar[ul]\ar@{<-}[ur]
&
\\
&\ar@{<-}[luu]
\ar[ruu]
&
\\
&\ar@{<-}[luuu]
\ar[ruuu]
&
}
\ecen
\eeqa

Birdtracks originated as a graphical
way to represent the tensors in General Relativity (Gravitation). In General Relativity, one deals with tensors such as
$T\indices{_a^b_c}$ which have some indices raised
and some lowered. One can use the metric 
$g^{a,b}$ to raise all the lowered indices
to get $T^{abc}$. If we represent this
graphically as a node with incoming arrows 
$a,b,c$, we need to 
follow one of the following
2 conventions: either
\begin{enumerate}
\item
label the arrows 
as $\rva$, $\rvb, \rvc$, 
and define the node as
$T^{\rva\rvb\rvc}$,
or
\item
instead of labelling the
arrows explicitly $\rva, \rvb, \rvc$, 
 indicate in the node
where is the first arrow
$\rva$, and draw the
arrows $\rva, \rvb, \rvc$
so that they enter the node
in {\bf counterclockwise} (CC) order.
The {\bf left-to-right} (L2R) order
of the indices on $T$ corresponds
the CC order of the arrows.
\end{enumerate}
If we don't do either 1 or 2, we won't
be able to distinguish between
the graphical
representations of $T^{1,2,3}$
and $T^{2,1,3}$, for example.
Cvitanovic's Birdtracks book
Ref.\cite{birdtracks-book} follows Convention 2, but
most of the time, in this book, we will follow
Convention 1 \footnote{If we follow Convention 1,
we don't need to reverse the L2R order of the indices
when taking a Hermitian conjugate. Thus,
$(X^\dagger)\indices{^\rva^\rvb_\rvc_\rvd}=
X\indices{_\rva_\rvb^\rvc^\rvd}=
X\indices{^\rvd^\rvc_\rvb_\rva}$.
As long as $\rva, \rvb$ are lower indices and $\rvc,\rvd$ are upper
indices of $X$, any L2R
order of $\rva, \rvb, \rvc, \rvd$ 
is equivalent
under Convention 1.}
The reason I chose to do so is for the sake of consistency:
Convention 2 
is closer to the quantum bnet conventions. 





$a^{:m}\in \ZZ_+^m$

\beq
R^{a_3^{:m_3}, b_2^{:n_2}}
_{b_3^{:n_3}, a_2^{:m_2}}
S^{a_2^{:m_2}, b_1^{:n_1}}
_{b_2^{:n_2}, a_1^{:m_1}} =
\bcen
\xymatrix{
b_3^{:n_3}
&R\ar[l]\ar@{<-}[ld]
&\sum  b_2^{:n_2}\ar[l]
&S\ar[l]\ar@{<-}[ld]
&b_1^{:n_1}\ar[l]
\\
a_3^{:m_3}
&
&\sum a_2^{:m_2}\ar@{<-}[lu]
&
&a_1^{:m_1}\ar@{<-}[lu]
}
\ecen
\eeq

\beq
\tr_\rvb X\indices{_a_\rvb^\rvb^d}
=
\sum_b X\indices{_a_b^b^d}
=
\bcen
\xymatrix@R=1pc{
a
&X\indices{_\rva_\rvb^\rvc^\rvd}
\ar[dl]\ar[l]
\\
\ar@[red]@{-}[d]
&
\\
\ar[ruu]
\\
d\ar[ruuu]
}\ecen
\eeq


\beq
\xymatrix{
&\ar[d]
&
&\ar@{<-}[d]\ar@[red]@{-}[ll]
&
\\
&R\ar[l]\ar@{<-}[ld]
&
&S\ar@{<-}[ld]
&\ar[l]
\\
&
&\ar@{<-}[lu]
&
&\ar@{<-}[lu]
}
\eeq