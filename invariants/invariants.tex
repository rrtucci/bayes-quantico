\chapter{Invariant Tensors}
\label{ch-invariants}
This chapter is based on Cvitanovic's Birdtracks book
 Ref.\cite{birdtracks-book}.

A {\bf bilinear form}
is a linear function $m:\dual{V}^n\times V^n\rarrow \CC$ usually with
$\dual{V}^n, V^n=\CC^n$.
For example, 

\beq
m(\dual{x}^{:n}, y^{:n})=
\dual{x}^a M\indices{
_a^b
} y_b
\xymatrix@R=1pc@C=1pc{
&M\ar[dl]
\ar@{<-}[dr]
\\
a&&b
}
\eeq
$m()$ is said to be invariant if 

\beq
m(\dual{x}^{:n}, y^{:n})
=
m(\dual{x}^{:n}G^\dagger, Gy^{:n})
\eeq
$m()$ is invariant iff matrix $M$ is an
{\bf invariant matrix}; i.e., iff

\beq
\myboxed
{
M\indices{
_a^b
} =
(G^\dagger)\indices{_a^{a'}}
G\indices{
_{b'}^b
}
M\indices{
_{a'}^{b'}
}}
\bcen
\xymatrix@R=1pc@C=1pc{
&M\ar[dl]
\ar@{<-}[dr]
\\
a&&b
}
\ecen
=
\bcen
\xymatrix@R=2pc@C=1pc{
&M\ar[dl]|{G^\dagger}
\ar@{<-}[dr]|G
\\
a&&b
}
\ecen
\eeq

\beq
M= G^\dagger  M G
\eeq
If $G$ is unitary,
\beq
GM=MG, \quad [G, M] =0
\eeq

A {\bf multilinear form}
is a linear function $h:\dual{V}^{n^p}\times V^{n^q}\rarrow \CC$, usually with
$\dual{V}, V=\CC$.
For example,


\beq
h(\dual{w}, \dual{x}, y, z)
=
h\indices{_a_b^c^d}
\dual{w}^a
\dual{x}^b
y_c
z_d
\quad\quad
\bcen
\xymatrix@R=1pc@C=1pc{
h
\ar[d]
\ar[dr]
\ar@{<-}[drr]
\ar@{<-}[drrr]
\\
a
&b
&c
&d
}
\ecen
\eeq
$h()$ is said to be
invariant if

\beq
h(\dual{w}, \dual{x}, y, z)=
h(\dual{w}G^\dagger, \dual{x}G^\dagger, Gy, Gz)
\eeq
$h()$ is invariant iff tensor $h\indices{_a_b^c^d}$
is a
{\bf invariant tensor}  (IT); i.e., iff

\beq
\myboxed{
h\indices{_a_b^c^d}
=
(G^\dagger)\indices{
_a^{a'}
}
(G^\dagger)\indices{
_b^{b'}
}
h\indices{_{a'}_{b'}^{c'}^{d'}}
G\indices{
_{c'}^c
}
G\indices{
_{d'}^d
}}
\bcen
\xymatrix@R=1pc@C=1pc{
h
\ar[d]
\ar[dr]
\ar@{<-}[drr]
\ar@{<-}[drrr]
\\
a
&b
&c
&d
}
\ecen
=
\bcen
\xymatrix@R=2pc@C=1pc{
h
\ar[d]|{G^\dagger}
\ar[dr]|{G^\dagger}
\ar@{<-}[drr]|G
\ar@{<-}[drrr]|G
\\
a
&b
&c
&d
}
\ecen
\eeq

A {\bf composed IT} is an IT that can
be written as a product or contraction
of ITs.

A {\bf tree IT}
is a composed IT
without any loops.

A {\bf primitive IT}
is an IT that can be expressed as a linear
combination of
a finite number of tree ITs.


The {\bf primitiveness assumption}: All IT are primitive.
\hrule
Examples. Suppose $x, y, z \in\RR^3$
and $i,j,k\in\{1,2,3\}$.
\begin{itemize}
\item Primitive ITs
\beq
length(x)
=
\delta_{ij}x_i x_i
\;
\quad
volume(x,y,z)
=
\eps_{ijk}x_iy_jz_k
\eeq


\beq
\delta_{ij}=
\xymatrix{
i
&j\ar@{-}[l]
}
\;,\quad
\eps_{ijk}=
\bcen
\xymatrix{
&\eps
\ar@{-}[dl]
\ar@{-}[d]
\ar@{-}[dr]
\\
i
&j
&k
}
\ecen
\eeq


\item Tree ITs


\beq
\delta_{ij}\eps_{klm}=
\bcen
\xymatrix{
i\ar@{-}[d]
&& \eps
\ar@{-}[dl]
\ar@{-}[d]
\ar@{-}[dr]
\\
j
&k
&l
&m
}
\ecen
\eeq

\beq
\eps_{ijm}\delta_{mn}\eps_{nkl}
=
\bcen
\xymatrix{
&\eps_{ijm}\ar@{-}[dl] \ar@{-}[d]
&\eps_{mkl}\ar@{-}[l]|{\sum m}
\ar@{-}[d]
\ar@{-}[dr]
\\
i
&j
&k
&l
}
\ecen
\eeq

\item Non-tree IT

\beq
\eps_{ims}
\eps_{jnm}
\eps_{krn}
\eps_{lsr}
=
\bcen
\xymatrix{
i
&\eps_{ims}\ar@{-}[l]
\ar@{-}[d]|{\sum m}
&\eps_{lsr}\ar@{-}[l]|{\sum s}
\ar@{-}[d]|{\sum r}
&l\ar@{-}[l]
\\
j
&\eps_{jnm}\ar@{-}[l]
&\eps_{krn}\ar@{-}[l]|{\sum n}
&k\ar@{-}[l]
}
\ecen
\eeq

\beq
\begin{array}{l}
\myboxed{
\eps_{ims}
\eps_{jnm}
\eps_{krn}
\eps_{lsr}=
\delta_{ij}
\delta_{kl}
+
\delta_{il}
\delta_{jk}
}
\\
\bcen
\xymatrix{
i
&\eps_{ims}\ar@{-}[l]
\ar@{-}[d]|{\sum m}
&\eps_{lsr}\ar@{-}[l]|{\sum s}
\ar@{-}[d]|{\sum r}
&l\ar@{-}[l]
\\
j
&\eps_{jnm}\ar@{-}[l]
&\eps_{krn}\ar@{-}[l]|{\sum n}
&k\ar@{-}[l]
}
\ecen
=
\bcen
\xymatrix{
i\ar@{-}[d]
&l\ar@{-}[d]
\\
j
&k
}
\ecen
+
\bcen
\xymatrix{
i\ar@{-}[r]
&l
\\
j\ar@{-}[r]
&k
}
\ecen
\end{array}
\eeq

\item Primitiveness Assumption

Suppose $\calp=\{\delta_{ij}, f_{ijk}\}$
where $f_{ijk}$ is not $\eps_{ijk}$. For 
some $A, B, C, \ldots H\in\CC$, one has

\beq
\xymatrix@C=1pc@R=1pc{
&\Circle{.}\ar@{-}[l]
&\ar@{-}[l]
}
=
A\xymatrix@C=1pc@R=1pc{
&&\ar@{-}[ll]
}
\eeq


\beq
\bcen
\xymatrix@C=1pc@R=1pc{
&\ar@{-}[d]
\\
&\Circle{.}\ar@{-}[l]
&\ar@{-}[l]
}
\ecen
=
B
\bcen\xymatrix@C=1pc@R=1pc{
&\ar@{-}[d]
\\
&\bullet
&\ar@{-}[ll]
}
\ecen
\eeq

\beq
\bcen
\xymatrix@C=1pc@R=1pc{
&\ar@{-}[d]
\\
&\Circle{.}\ar@{-}[l]
&\ar@{-}[l]
\\
&\ar@{-}[u]
}
\ecen
=
\left\{
\begin{array}{l}
C
\bcen
\xymatrix@C=1pc@R=1pc{
&&\ar@{-}[ll]
\\
&&\ar@{-}[ll]
}
\ecen
+D
\bcen
\xymatrix{
\ar@{-}[dr]&\ar@{-}[dl]
\\
&
}
\ecen
+E
\bcen
\xymatrix@C=1pc@R=1pc{
&\bullet \ar@{-}[l]
\ar@{-}[d]
&\ar@{-}[l]
\\
&\bullet \ar@{-}[l]
&\ar@{-}[l]
}
\ecen
\\
+F
\bcen
\xymatrix@C=1pc@R=1pc{
\ar@{-}[dd]&\ar@{-}[dd]
\\
\\
&
}
\ecen
+G
\bcen
\xymatrix@C=1pc@R=1pc{
\ar@{-}[d]&\ar@{-}[d]
\\
\bullet\ar@{-}[d]
\ar@{-}[r]
&\bullet\ar@{-}[d]
\\
&
}
\ecen
+H
\bcen
\xymatrix@C=1pc@R=1pc{
&&
\\
\bullet\ar@{-}[ur]
\ar@{-}[d]
\ar@{-}[r]
&\bullet\ar@{-}[ul]
\ar@{-}[d]
\\
&
}
\ecen
\end{array}
\right\}
\eeq
\end{itemize}

\hrule


Let $\calp=(p_1, p_2, \ldots, p_k)$ be a {\bf full set of primitive ITs}. By \qt{full}, we mean no others exist.
$\calp$ is 
a basis for an {\bf algebra of invariants}.\footnote{An algebra over a field
is defined in Sec.\ref{sec-algebra-over-f}}

An {\bf invariance group} $\calg$ with
a full set of primitive ITs $\calp=\{p_1, p_2, \ldots, p_k\}$ is the set of all linear transformation $G\in \calg$ such that

\beqa
p_1(\dual{x}, y)&=&
p_1(\dual{x}G^\dagger, Gy)
\\
p_2(\dual{w}, \dual{x}, {y}, {z})&=&
p_2(\dual{w}\dual{G}, 
\dual{x}\dual{G}, Gy,
Gz)
\\
&&\text{etc.}
\eeqa


Example. Consider
an invariance group
with a single
primitive IT $p()$ defined by

\beq
p(\dual{x}, y) = \delta_a^b \dual{x}^a y_b=\dual{x}^by_b
\eeq
Then

\beq
\dual{(x')}^a (y')_a=
\dual{x}^b (G^\dagger G)\indices{_b^c}y_c = \dual{x}^by_b
\eeq
so $G$ must be unitary
\beq
G^\dagger G=1
\eeq
 
The group of $n$ dimensional unitary matrices is called $U(n)$



