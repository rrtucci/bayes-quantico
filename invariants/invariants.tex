\chapter{Invariants}
\label{ch-invariants}

Given a {\bf bilinear form} 

\beq
m(\bar{x}^{:n}, y^{:n})=
x^a M\indices{
_a^b
} y_b
\xymatrix@R=1pc@C=1pc{
&M\ar[dl]
\ar@{<-}[dr]
\\
a&&b
}
\eeq
is invariant if 

\beq
m(\bar{x}^{:n}, y^{:n})
=
m(\bar{x}^{:n}G^\dagger, Gy^{:n})
\eeq

{\bf matrix invariant}

\beq
\boxed
{
M\indices{
_a^b
} =
(G^\dagger)\indices{_a^{a'}}
G\indices{
_{b'}^b
}
M\indices{
_{a'}^{b'}
}}
\bcen
\xymatrix@R=1pc@C=1pc{
&M\ar[dl]
\ar@{<-}[dr]
\\
a&&b
}
\ecen
=
\bcen
\xymatrix@R=2pc@C=1pc{
&M\ar[dl]|{G^\dagger}
\ar@{<-}[dr]|G
\\
a&&b
}
\ecen
\eeq

\beq
M= G^\dagger  M G
\eeq

\beq
GM=MG, \quad [G, M] =0
\eeq

{\bf multilinear form}

\beq
h(\bar{w}, \bar{x}, y, z)
=
h\indices{_a_b^c^d}
w^a
x^b
y_c
z_d
\quad\quad
\bcen
\xymatrix@R=1pc@C=1pc{
h
\ar[d]
\ar[dr]
\ar@{<-}[drr]
\ar@{<-}[drrr]
\\
a
&b
&c
&d
}
\ecen
\eeq
is invariant if

\beq
h(\bar{w}, \bar{x}, y, z)=
h(\bar{w}G^\dagger, \bar{x}G^\dagger, Gy, Gz)
\eeq


{\bf tensor invariant}  (TI)

\beq
\boxed{
h\indices{_a_b^c^d}
=
(G^\dagger)\indices{
_a^{a'}
}
(G^\dagger)\indices{
_b^{b'}
}
h\indices{_{a'}_{b'}^{c'}^{d'}}
G\indices{
_{c'}^c
}
G\indices{
_{d'}^d
}}
\bcen
\xymatrix@R=1pc@C=1pc{
h
\ar[d]
\ar[dr]
\ar@{<-}[drr]
\ar@{<-}[drrr]
\\
a
&b
&c
&d
}
\ecen
=
\bcen
\xymatrix@R=2pc@C=1pc{
h
\ar[d]|{G^\dagger}
\ar[dr]|{G^\dagger}
\ar@{<-}[drr]|G
\ar@{<-}[drrr]|G
\\
a
&b
&c
&d
}
\ecen
\eeq

A {\bf composed TI} is a TI that can
be written as a product or contraction
of TIs.

A {\bf tree TI}
is a composed TIs
without any loops.

A {\bf primitive TI}
is a TI that can be expressed as a linear
combination of
a finite number of tree TIs.


The {\bf primitiveness assumption}: All TI are primitive.
\hrule
Consider $\RR^3$ vector space.

\beq
length(x)
=
\delta_{ij}x_i x_i
\;
\quad
volume(x,y,z)
=
\eps_{ijk}x_iy_jz_k
\eeq

Primitive TIs

\beq
\delta_{ij}=
\xymatrix{
i
&j\ar@{-}[l]
}
\;,\quad
\eps_{ijk}=
\bcen
\xymatrix{
&\eps
\ar@{-}[dl]
\ar@{-}[d]
\ar@{-}[dr]
\\
i
&j
&k
}
\ecen
\eeq
Tree TIs


\beq
\delta_{ij}\eps_{klm}=
\bcen
\xymatrix{
i\ar@{-}[d]
&& \eps
\ar@{-}[dl]
\ar@{-}[d]
\ar@{-}[dr]
\\
j
&k
&l
&m
}
\ecen
\eeq

\beq
\eps_{ijm}\delta_{mn}\eps_{nkl}
=
\bcen
\xymatrix{
&\eps_{ijm}\ar@{-}[dl] \ar@{-}[d]
&\eps_{nkl}\ar@{-}[l]
\ar@{-}[d]
\ar@{-}[dr]
\\
i
&j
&k
&l
}
\ecen
\eeq
Non-tree TI

\beq
\eps_{ims}
\eps_{jnm}
\eps_{krn}
\eps_{lsr}
=
\bcen
\xymatrix{
i
&\eps_{ims}\ar@{-}[l]
\ar@{-}[d]|{\sum m}
&\eps_{lsr}\ar@{-}[l]|{\sum s}
\ar@{-}[d]|{\sum r}
&l\ar@{-}[l]
\\
j
&\eps_{jnm}\ar@{-}[l]
&\eps_{krn}\ar@{-}[l]|{\sum n}
&k\ar@{-}[l]
}
\ecen
\eeq

\beq
\begin{array}{l}
\boxed{
\eps_{ims}
\eps_{jnm}
\eps_{krn}
\eps_{lsr}=
\delta_{ij}
\delta_{kl}
+
\delta_{il}
\delta_{jk}
}
\\
\bcen
\xymatrix{
i
&\eps_{ims}\ar@{-}[l]
\ar@{-}[d]|{\sum m}
&\eps_{lsr}\ar@{-}[l]|{\sum s}
\ar@{-}[d]|{\sum r}
&l\ar@{-}[l]
\\
j
&\eps_{jnm}\ar@{-}[l]
&\eps_{krn}\ar@{-}[l]|{\sum n}
&k\ar@{-}[l]
}
\ecen
=
\bcen
\xymatrix{
i\ar@{-}[d]
&l\ar@{-}[d]
\\
j
&k
}
\ecen
+
\bcen
\xymatrix{
i\ar@{-}[r]
&l
\\
j\ar@{-}[r]
&k
}
\ecen
\end{array}
\eeq

\hrule
An {\bf algebra of invariants}

Let $\calp=(p_1, p_2, \ldots, p_k)$ be a full set of primitives. By \qt{full}, we mean no others exist.

An {\bf invariance group} $\calg$ is the set of all linear transformation $G\in \calg$ such that

\beqa
p_1(x, \bar{y})&=&
p_1(Gx, \bar{y}G^\dagger)
\\
p_2(w, x, \bar{y}, \bar{z})&=&
p_2(Gw, Gx, \bar{y}G^\dagger,
\bar{z}G^\dagger)
\\
&&\text{etc.}
\eeqa

Example

\beq
p(\bar{x}, y) = \delta_a^b x^a y_b=x^by_b
\eeq

\beq
(x')^a (y')_a=
x^b (G^\dagger G)\indices{_b^c}y_c = x^by_b
\eeq
So $G$ must be unitary
\beq
G^\dagger G=1
\eeq
 
The group of $n$ dimensional unitary matrices is called $U(n)$



